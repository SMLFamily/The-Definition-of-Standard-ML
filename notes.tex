\documentstyle[12pt]{article}
\title{The Semantics of Standard ML\\RECORD OF CHANGES AND VERSIONS}
\author{Mads Tofte}
\begin{document}
\maketitle
\begin{center} \bf 1987 \end{center}
\vspace{1cm}
\begin{description}
\item{13 August 87} ``The Semantics of Standard ML, Version 1'' 
by Robert Harper, Robin Milner and Mads Tofte. ECS-LFCR-87-36,
August 87.

\item{13 August 87 (annotated).} A copy of [13 August 87]
annotated by RM. There are annotations in blue (regarding
exceptions) and other corrections (marked in red).  

\item{25 November 87} ``The Semantics of Standard ML''. The semantics
of 13 August changes so as to accomodate exception constructors  (MacQueen's
proposal).  
All changes from [13 August 87] are taken from [13 August 87,
annotated] (blue corrections only) and marked in the margin.
Does not include polymorphic exceptions or references. 
Copy sent of to MacQueen for approval 18 Jan 88. 
Original with me.
\end{description}

\newpage
\begin{center} \bf 1988 \end{center}
\vspace{1cm}
\begin{description}
\item{21 June 88} ``The definition of Standard ML, Version 2''.
I have done and checked Robin's corrections to [25 Nov 87] 
as listed on my ``cover letter'' [ad 25 Nov 87].
Also, I have done and checked the red corrections 
of [13 August 87, annotated]. In addition the following changes.

\begin{enumerate}
\item
Exceptions have monotypes only. Rule (30) and restriction 1
in Section 4.9 have been modified accordingly. 

\item
The type constructor {\tt excn} has been changed to {\tt exn}.
(I have maintained {\it exn} to range over exeption names.)

\item
The type {\tt exn} does not admit equality. This has been
made explicit on the first page of appendix C.

\item
The third example of a real constant in Section 2.2 has been 
corrected (RM had earlier corrected master copy by hand).

\item
Abstype should hide equality outside the with part. Changes to
Section~4.7 and the present rule (19), formerly (21).

\item
The rule for constants in atomic patterns in the dynamic semantics
of the core (rule 36 in Version 1) changed so as to make the
types of constants more obvious.

\item
I have inserted sentences in Section 4.4 (Version 1) to the effect
that the equality attribute of a bound type variable in a type
function does not matter. Also, I have inserted sentences in Section 4.5
(Version 1) to the effect that the equality attributes of bound type
variables in type schemes do matter.

\item
My scheme for polymorphic references and exceptions tentatively
included. See ``Polymorphic References and Exceptions in the
ML Semantics'', 20 May 88, Mads Tofte, for a detailed list of
changes. In particular, I have undone the above change (that
exceptions have monotypes only) to allow exceptions
to have imperative types. (30 May 88)

\item
Basic exception constructors changed to start with a capital
letter (Bind, Match etc) in accordance with ``Unifying 
Exceptions with Constructors in Standard ML (30 May)''.
(30 May 88, mt)

\item
Semantics of programs, henceforth Section 8, included.
Corrections to the semantics of modules as detailed in
``Changes to ML Semantics to deal with Programs properly'',
RM 17/5/88 with my added handwritten notes.(31 May 88, mt)

\item
Previously ``applied'' functor forms have been turned into
derived forms and moved to Appendix A. 
The previously ``pure'' forms new appear directly in the grammar. (1 June 88,mt)

\item
Comments added in the section ``Closure Restrictions'' to clarify
the closure restrictions.  Comments also added on the rules for
signature declaration, functor specification and functor declaration
in the static semantics for modules for the same purpose.
(2 June 88, mt)

\item
The sentence stating that there are no derived forms form modules (first page
in the introduction) has been replaced by a reference to Appendix A. 
(6 June 88)

\item
In Figure 2, Pats has been changed to Pat. (6 June  88)

\item
A section ``Signature Matching'' has been inserted in the static
semantics for modules, after Section 5.10 (Enrichment). This defines
matching as a combination of instantiation and enrichment.
Also, the definition of functor matching in Section 5.13 has been
simplified using the notion of matching. (8 June 88, mt)

\item
Section 6.2 (Simple Objects), below figure. The following has been deleted:
`` or of a failing attempt to handle an exception with a handle rule''.
(9 June 88)

\item
Section 6.3, last paragraph. ``completely separately'' replaced by ``separately'', because of the new Section 8. (9 June 88)

\item
Page 55, Version 1. Wrong heading ``Labelled Patterns'' replaced by
``Pattern Rows''. (9 June 88)

\item
Page 40, Version 1. Heading ``Structure-level Bindings'' replace by
``Structure Bindings'' in accordance with page 15 and page 60.
(9 June 88)

\item
Page, 64, Figure 18. The derived form for raise expression has been 
deleted. (9 June 88)

\item
Page 65, Figure 19. Wrong heading ``LABELLED PATTERNS'' has been replaced by
``PATTERN ROWS''. Moreove, in the same figure, ``TYPES'' has been
replaced by ``TYPE EXPRESSIONS''.(9 June 88)

\item
In the display in Section 6.4 (Basic Values), the \verb+\+ has
been replace by \verb+/+ (division of reals). (13 June 88)

\item
Section 2.5, the first sentence. A reference to Section 2.7 has been
added, since \verb+#+{\it lab} introduces a new lexical item. This has
been added to Section 2.7. (13 June 88)

\item
Section 3. Since there are at present derived forms for functors,
the introduction to Section 3 has been changed. (13 June 88)

\item
Rule 91 (Version 1) has been corrected. (The result part of
a functor signature resulting from the elaboration of a 
functor signature expression must be principal).(14 June 88)

\item
The heading ``Functor Signatures'' in Figure 20 (current version)
concerning derived forms has been replace by ``Functor Signature
Expressions''. Also, the compound ``functor signatures'' has
been replaced by ``functor signature expressions'' on the first
page of Appendix A (Derived Forms). (15 June 88)

\item
We use the term ``constructor binding'' instead of  ``datatype construction''.
The syntactic classes are renamed accordingly. 

\item
The title of the document has been changed to ``The definition
of Standard ML''. (15 June 88)

\item
Appendix C, bottom of first page: \verb+io_failure+ has been changed to
\verb+Io+. (15 June 88)

\item
Section 3.1 (Reseved Words, Syntax of Modules). The {\tt open}
has been removed since it is already in the core langugae.
(15 June 88)

\item
Section 8. Changed ``store'' to ``state'' throughout. (16 June 88).

\item
Syntax of the Core, Figure 2. Replaced ``type expression rows''
by ``type-expression rows''. (17 June 88).

\item
Page 75 (current version). Templates added at the top of Figure 26 and 27.
(17 June 88).

\item
Section 7.1 (Reduced Syntax, Dynamic semantics of modules), first
item. The ``are omitted from exception bindings'' has been replaced
by ``are omitted from exception descriptions''. (20 June 88)
\end{enumerate}

\item{24 June 88.}``The definition of Standard ML, Version 2''.
Most of the following changes are prompted by Robin's reading
of [21 June]. 
\begin{enumerate}
\item 
The preface to Version 2 has been revised following RM's notes (not
followed to the letter, however). (23 June 88)

The term ``functor matching'' has been replaced by ``functor signature
matching'' throughout. (23 June 88)

\item
The term ``store attribute'' has been replaced by ``imperative attribute''.
(23 June 88)

\item
Page 44, rule 85 regarding type sharing. The premise corrected. $\theta$s
have been inserted. A comment to the rule has been inserted. This change
is also mentioned in the preface to Version 2.
(23 June 88)

\item
The varible {\it exn}, which ranges over exception names, has been
renamed to {\it en}. (23 June 88)

\item
The class Exn of exception names has been renamed to ExName.
 (23 June 88)
\item
The variable {\it exns}, which ranges over finite sets of exception
names, has been renamed to {\it ens}.
 (23 June 88)
\item 
The class ExnSet of finite sets of exception names has been renamed
to ExNameSet in accordance with TyNameSet , NameSet and StrNameSet.
 (23 June 88)
\item
The class BasExn of basic exception names has been renamed
to BasExName.
 (23 June 88)
\end{enumerate}

\item{July 19} Several small changes to June 24.
\begin{enumerate}
\item The ``definition'' changed to ``Definition'' in the title of the
document (also in the preface to Version 2). The title has been moved
to the front. Page numbering is changed, page 1 is the first page of
Section 1.

\item \# introduced as keyword and allowed in symbolic identifiers.

\item A comment about consistency added for the type sharing rule.

\item Subscripts 1 and 2 interchanged in the last paragraph of Section 
5.11.
\end{enumerate}

\item{1 August 88.} Several minor changes to July 19, as listed in
the annotated copy of the latter as detailed in 4 handwritten sheets
of paper. 

Two copies printed on the Plw16 3 August 88.
\end{description}

\newpage
\begin{center} \bf 1989 \end{center}
\vspace{1cm}
\begin{description}
\item{4 April 89} Began changes towards ``The definition of Standard ML, Version 3'',
starting from Version 2. (Page refs to version 2)
\begin{enumerate}
\item Changed version number in title to 3 and date to 1 May 89.
\item Did the corrections of the errata with modifications indicated in 
``MORE CORRECTIONS''.
\item Several changes to the two first paragraphs of page 81.
\item Small changes on pages iv, v, 4, 5 ( * is excluded from TyCon ), 7,
11, 14, 19, 31 (remark to the effect that change of bound names does
not change arity or equality attribute), 32 (change of the definition
of consistency), 36, 39, 41, 48, 61, 62, 63, 65, 66, 67, 68, 72, 73, 74, 77,
79, 80, 81, 88 as marked in ``MORE CORRECTIONS''.
\end{enumerate}
\item{5 April 89} More corrections:
\begin{enumerate}
\item Introduced the basic exceptions Abs and Neg, pages 46, 74, 87 and 92.
\item Replaced all occurrences of `keyword' by `reserved word', pages 6 and 10.
\item To avoid that the parsing of \quad   local in end \quad declarations
      interact with the requirement that there be no free imperative type
      variables, the side condition on rule 55 (structure-level declarations)
      has been moved to the rules for top-level declarations (rules 96--98).
      See pages 37, 42, 43, 91, 93 and 96.
\item Clarification concerning the disambiguation in parsing inserted
      in Appendix B, page 66.
\item The rules for the use of the reserved word {\tt op} have
      been clarified, see page 6.
\item A new section ``Core language Programs'' has been added at the
      end of Section 8, see page 62.
\end{enumerate}
\item{6 April '89} More corrections:
\begin{enumerate}
\item Introduced syntax for long and short identifiers. Pages
      4, 91, 92, 94
\item Type constraints on non-expansive expressions are allowed
      and do not make the expression expansive (page 20).
\item Section 2.9 (Syntactic Restrictions), page 8, last bullet:
      The function expression on the right-hand side of a recursive
      value binding is allowed to be constrained by type constraints.
\item A new subsection ``Type Explication'' has been inserted in the
      static semantics of modules, after Section 5.7, page 33.
\item Other changes concerning type explication: Comment added in
      Section 5.8, page 33; comment added in Section 5.11, page 34.
\item Small correction in the second and third paragraph of the comment
      on the rule for functor application (page 37).
\item New inference rule for signature expressions added (page 38).
\item Rule 60 concerning structure bindings (page 38) altered so
      as to make sure that the signature expression elaborates to
      a type-explicit signature.
\item Similar change to rule 66 concerning signature bindings (page 39).
      The comment adjusted accordingly.
\item Similar change to rule 91 concerning functor signature 
      expressions (page 42). Comment added.
\item Similar change to rule 95 concerning functor bindings (page 42).
      The comment adjusted accordingly.
\item Several changes to make sure that {\tt op} does not occur in
      signatures.  Details are as follows:
\item A new syntactic class ConDesc of constructor descriptions has
      been introduced (page 11).
\item The syntax for specifications (Figure 7, page 13) has been modified
      by changing the syntax of data descriptions and introducing a
      rule for constructor descriptions.
\item The syntactic restriction (page 12, second bullet) has been
      extended to include constructor descriptions.
\item An optional {\tt op} has been included in the syntax for
      exception bindings, page 8 and 68. 
\item Rule 81 concerning Datatype Descriptions has been adjusted
      to the new syntax. Page 40.
\item New rule for static semantics of constructor descriptions has
      been added after rule 81.
\item Constructor descriptions added to the list in the third bullet 
      of Section 7.1 (page 55).
\item Introduction of special constants (scon and Scon); many changes:
\item Indtroduce scon and Scon (pages 3, 88, 93, 94).
\item Syntax for scon as atomic expression introduced (page 8).
\item Syntax for scon as atomic pattern introduced (page 9).
\item Added paragraph at the end of 4.1 (page 16) giving types
      to special constants.
\item Added new rule for scon as atomic expression before rule 1
      (page 23)
\item Added new rule for scon as atomic pattern after rule 32
      (page 28).
\item Introduced special values (sv and SVal), page 44 and Figure 13,
      page 45.
\end{enumerate}
\item{7 April '89,} More changes:
\begin{enumerate}
\item New rule for special constants as atomic expressions added, page 48,
      before rule 99.
\item Two new rules for special constants as atomic pattern added, page 52,
      after rule 135.
\item scon included in the grammar for atomic expressions (page 67) and
      atomic patterns (page 69).
\item Last blob of Figure 23, page71, concerning the types of special 
      constants has been deleted. (These types are given by the function
      called type).
\item Figure 25, page 72, modified so that the constructor environments
      of int, real and string become empty.
\item Small insertion on page 74, firt para., first line.
\item The blob on equality, page 75, rewritten.
\item Page 46, several replacements of mathematical = sign by ML = sign.
\item page 2, l.3,  small correction.
\item page 5, Section 2.5, line 2, small correction.
\item page 5, Section 2.6, l -2: insert : ``(Note that qualified identifiers
      never have infix status.)''
\end{enumerate}
\item{10 Apr 89} Late 7 Apr I printed a version with the corrections
so far (including many changes to the index) all of which are marked
in ``MORE CORRECTIONS''.  Checking the changes, led to more corrections:
(references now to the version printed 7 April).
\begin{enumerate}
\item Several changes in index.ml (the ML program) so that entries
      are collapsed to intervals, when possible and so that intervals
      will not be trivial.
\item page 4, Sec. 3.4, line 7 : ``The qualified identifiers''
\item page 30, Sec. 4.11, bullet 2, line 1: {\tt |}, twice
\item page 33, line -3, ``is type-explicit and every name in $N_1$ occurs
      in $S_1$''
\item page 34, line -2, same change
\item page 37, line 4, ``is type-explicit and N subseteq names $S_f$''
\item page 39, rule for signature bindings, conclusion: repl. 
      $(N)S$ by $\Sigma$.
\item page 40, rule for seq. spec, second premise: change subscript
      from 1 to 2.
\item page 43, comment on rule for Functor Bindings, l4 ``can match (N)S 
      (assuming N subseteq names S).''
\item page 45, Sec. 6.1, last bullet: change variables to classes (also
      in index)
\item page 54, comments: ``because of rule 142, rule ''
\item page 55, comments: ``because of rule 142, rule ''
\item page 56, Sec. 7.1, last bullet, change variables to classes (also
      in index).
\item page 60, rule for local spec, second premise: change 1 to 2
\item page 64, bullet 3: ``syntax class of declarations {\it dec}''
\item page 89, entry for $\{\ \}$ in pattern: change 53 to 54.
\item page 91, entry for condesc: add 56
\item page 91, entry for ConDesc: add 56
\item page 91, entry for constructor description: add 56
\item page 91, entry for core language programs: move below subitems
\item page 92, entry for equality: change basics value and identifier
      to of values
\item page 92, entry for equality of type function: insert 41
\item page 92, entry for exception constant as atomic pattern: change
      53 to 53-54.
\item page 93, entry for FAIL: add 54
\item page 94, entry for identifier, long: insert 64
\item page 94, entry for imptyvars: replace 43-43 by 43
\item page 95, entry for open: add 40
\item page 96, entry for polymorphic references: repl. 43-43 by 43
\item page 96, entry for polymorphic exceptions: repl. 43-43 by 43
\item page 96, entry for (r) record : repl. 53 by 54
\item page 96, entry for  record (r): repl. 53 by 54
\item page 96, entry for record as atomic pattern: repl. 53 by 54
\item page 97, entry for signature (Sigma), delete extra 35
\item page 99, entry for type variable, imperative: replace 43-43 by 43
\end{enumerate}
\item{11 April 89} More changes
\begin{enumerate}
\item Changed index.ml so that an interval of pages from i to i+1 is
      printed i, i+1 (rather than i--i+1).
\item page 93, entry for FAIL: add 52
\item page 96, entry for polymorphic exceptions: repl. 43-41 by 41, 43
\item page 96, entry for pattern matching: repl. 45, 47-30 by 30, 45,47
\end{enumerate}
\item{14 April '89} Changes to the version printed 7--11 April, following
Robin's annotations.
\begin{enumerate}
\item page iii, several minor changes.
\item page vi, item 15: yet an attempt to get rid of blanks
\item page 1, several minor changes.
\item page 3, Section 2.2 (Special Constants), third paragraph, after first
sentence; insert:
``We assume an underlying alphabet of 256 different characters (numbered 
0 to 255) which is such that the characters with numbers 0 to 127 coincide
with the ASCII character set.''
\item page 3, Section 2.2 (Special Constants), fourth line of display; replace
	line by:
\begin{quote}
	ddd The single character with number ddd (3 decimal digits
             denoting an integer in the interval [0,255]).
\end{quote}
\item page 4, removed the recent definition of short identifiers
\item page 6, line 2 and 4: similar adjustments
\item page 94, entry for identifier: replace ``short'' by ``qualified''
\item page 12, Section 3.5, second bullet: replace condesc by datdesc.
\item page 30, Section 4.11, second bulled, line 1: use cdots
\item page 32, Section 5.3: change the definition of well-formed
      signature.
\item page 33, Section 5.9, line -4: delete ``and every name in N1
      occurs in S1''
\item page 34, Section 5.12, line -2: same change
\item page 37, comment (55): delete ``and  N subseteq names Sf''



\item page 38: replace ``must elaborate'' by ``is elaborated''
\item page 43, comment on  (99): delete ``(assuming N subseteq names S)''
\item page 45, line -6: replace second ``denoted by'' by ``written''
\item page 47, Section 6.5 : more space in display
\item page 54, rule 149 (paraenthesised pattern): change VE to VE/FAIL
      both in the premise and in the conclusion.
\item page 62, line -9: replace ``ranged over BDyn'' by ``and let us
      use BDyn to range over DynamicBasis.''
\item page 63, line 9: replace ``).'' by ``.)''
\item page 64, bullet 4.: delete ``short'' and end the sentence
      ``, i.e. omit qualified identifiers''.
\item page 66, Figure 15, entry for {\tt if}: number subsexpression 1, 2 and 3.
\item page 68, line -2: ``Suppose the lexical sequence''
\item page 68--89: use cdots
\item page 69, line 1: ``where $\exp$ stands for a lexical sequence which 
      is already determined as a subphrase (if necessary by applying the 
      precedence rule).''
\item page 78, bullet for equality: small adjustments
\item page 89, entry for = : first ref made into an entry by itself
\item page vi, item 15: yet another attempt to get rid of blanks!!!!
\item the index: replace ``coersion'' by ``coercion'' (twice)


\item page 77, bullet for chr(i); replace by:
\begin{quote}        
        chr(i) returns the character numbered i (see Section 2.2) 
	if i is in the interval [0, 255], and the packet [Chr] otherwise.
\end{quote}
\item page 78, bullet for ord(s); replace by:
\begin{quote}
        ord(s) returns the number of the first character in s (an integer
	in the interval [0,255], see Section 2.2), or the packet [Ord]
	if s is empty.
\end{quote}
\end{enumerate}
\end{description}
\begin{description}
\item{8 May 89} Tiny corrections:
\begin{enumerate}
\item page iii, middle: replace ``overriden'' by ``overridden''.
\item page 1, par. beginning ``A further factoring'', second sentence:
new punctuation.
\item page 4, Sect. 2.4: replace ``identifers'' by ``identifiers''.
\item page 32, Sect. 5.2: replace reference to the concept of disjoining
names (which was deleted earlier) by its definition.
\item page 47, Sect. 6.5, third line after display: ``The exception
(Io,s)''. Similar on page 79.
\item page 54, topmost comments: start with ``(142),(145),(147)''
\item page 66, Fig. 15, entry for {\tt if}: correct indices on the
right-hand side {\tt case} form.
\item page 79, entry for end of stream: slightly revised description.
\end{enumerate}
\end{description}
\begin{center}
\large Version 4 (for publication)
\end{center}
\begin{description}
\item{25 July 89} 
\begin{enumerate}
\item Page 3. Several small changes to clarify what printable characters
are and what \verb+\^+$c$ means.
\item Page5, last par. of  Sect. 2.4. The word ``parser'' has been
replace by ``compiler'' as the classification of variables,
value constructors and exception constructors cannot be done
in normal context free parsing (the scope of constructors is more 
conveniently dealt with during type checking).
\item Page 9, Fig. 4.  For uniformity, <op> has been allowed in front
of longcon and longexcon in atpat.
\item Page 21--22. Section 4.9 has been substantially revised
in reaction to Simon Finn's comments on the well-formedness
of type structures.  There are new concepts of type structures
being well-formed and of type structures and type environments
respecting equality.  Rules 19 and 20 have been given as side
condition that TE respect equality. The comments to these
rules have been revised accordingly.
\item Page 33, Sect. 5.3. The two first lines, which referred to
the well-formedness of type environenments, have been deleted.
item Page 35, Sect.5.13.  The definition of principal signatures
has been modified to allude to structures respecting equality.
\item Page 72. Inserted <op> in front of constants and exception 
constants in atomic patterns.
\item Index: inserted entries for \verb;^;, control charater and printable
character. Also entries for respecting equality. Entry for well-formed
type environments deleted.
\end{enumerate}
\item{27 July} More changes
\begin{enumerate}
\item Page 2. Last paragraph of the Introduction has been rewritten.
\item Pages 3--4. Slanted font for defined terms in Section 2 (integer 
constant, real constant,  string constant, comment and identifiers).
\item Page 21. The title of Sect. 4.9 has been shortened to
``Type Structures and Type Environments''.
\item Page 33. Sect. 5.5 has been modified to require the {\sl
combined} admissibility of the basis and the result of the elaboration.
\item Page 35. Section on Principal Signatures has been extended to
cover the notion of a basis {\sl covering} a structure.  This is 
needed in order to get signatures. In Version 3 there are {\sl not}
principal signatures, due to examples like
\begin{quote}
\begin{verbatim}
structure A =
  struct
    structure B1 = struct end
    structure B2 = struct end
  end 
structure A: sig structure B1: sig end end = A;
signature Sig = 
 sig
  structure A1 : sig structure B1: sig end
                     structure B2: sig end
                 end
  sharing A1 = A
 end;
\end{verbatim}
\end{quote}
\end{enumerate}
\item{3 August, more changes:}
\begin{enumerate}
\item Page 2. `in the basis' repl. by `against the background provided'
\item Page 2, last paragraph about `claims' adjusted
\item Page 23, side conditions in slant.
\item Page 17, Remark about $\cup$ before figure
\item Page 21, $con(v)$ repl. by $(con,v)$. Definition of `respecting equality' added.
Definition of what it means for $TE$ to maximise equality added. 
\item Page 25: repl. `respects' by `maximises' in rules 19,20
\item Page 34, Sect. 5.9: repl. `we claim' by `it can be shown'
\item Page 35, Sect. 5.12: repl. `we claim' by `it can be shown'
\item Page 35, Sect. 5.13: {\bf Principal signatures rewritten}
\item Page 38, rule 65, repl. `principal' by `equality-principal'
\item Page 30, end of Sect. 4.12, slight rephrasing
\item Page 47, line -5, `nil' made teletype
\item Page 39, rule 69: slight change to mention equality principal
\item Page 43, rule 99: slight change to mention equality principal
\end{enumerate}
\item{4 August, more changes:}
\begin{enumerate}
\item End of Sect. 4.9: Replace Abs(E,TE) by Abs(TE,E) twice. Replace E+Abs(TE)
by Abs(TE)+E. Sililar change in rule 20.
\item replace `implementor' by `implementer' throughout.
\item page 26, comment on rule 20:  page reference changed from 21 to 22.
\item page 35, small corrections to `Principal Signatures'
\item Index: several changes.
\end{enumerate}
\item{10 August 1989, More corrections; page numbers refer to version 3}
\begin{enumerate}
\item Robin's new preface has replaced the previous three prefaces.
\item page 30, Section 4.12. Local definition of enrichment for
the core language has been inserted. Also two new index entries.
Also shortening of Sect. 4.11 to fit it all in one page.
\item Page 37, inserted `for dec' in rule 57
\item page 40, repl. hypotesis by hypotesis.
\item page 40, insert $m_i$ in comment to rule 79.
\item page 51, comment on rules 112-118: deleted last line ', so v will be
$\ldots$  '
\item Sect. 5.5 about admissibility has been strengthened to require
the admissibility of entire proofs, as far as signature elaborations
are concerned.
\item Sect. 5.13 about Principal signatures: items 1 and 2 have
been shortened.
\end{enumerate}
\item 21 August 89 (after camera-ready had been sent)
\begin{enumerate}Changed Sigma_0 to Sigma on page 36 in the
definition of equality-principal signature. Robin sent that page
to Prior.
\end{enumerate}
\end{description}
\begin{description}
\item{4 Sept 89}, page 3, section 2.2, second par. line 4.  Removed the 
+ in front of 3.32E5. Also decreased hoffset by 8mm to get centering
on page.
\end{document}
