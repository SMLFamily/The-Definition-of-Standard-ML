% this is file `root', the root of the whole semantics
\documentstyle[a4,12pt,twoside]{article}
\title{The Definition of Standard ML\\ Version 3}
\author{Robert Harper \and Robin Milner \and Mads Tofte \\
Laboratory for Foundations of Computer Science\\
Department of Computer Science\\
University of Edinburgh\\
Edinburgh EH9 3JZ, Scotland}
\date{1 May 1989}
\newcommand{\ml}[1]{{\tt #1}}
\renewcommand{\cdots}{\mbox{$\cdot\!\cdot\!\cdot$}}

        % Core Language
\newcommand{\ttlbrace}{\mbox{\tt\char'173}}
\newcommand{\ttrbrace}{\mbox{\tt\char'175}}
\newcommand{\lttbrace}{\mbox{\tt\char'173}}
\newcommand{\rttbrace}{\mbox{\tt\char'175}}
\newcommand{\ttprime}{\mbox{\tt\char'047}}
\newcommand{\ABSTYPE}{{\tt abstype}}
\newcommand{\AND }{{\tt and}}
\newcommand{\ANDALSO }{{\tt andalso}}
\newcommand{\AS}{{\tt as}}
\newcommand{\CASE}{{\tt case}}
\newcommand{\DO}{{\tt do}}
\newcommand{\DATATYPE}{{\tt datatype}}
\newcommand{\ELSE}{{\tt else}}
\newcommand{\END}{{\tt end}}
\newcommand{\EXCEPTION}{{\tt exception}}
\newcommand{\FUN}{{\tt fun}}
\newcommand{\FN}{{\tt fn}}
\newcommand{\HANDLE}{{\tt handle}}
\newcommand{\IF}{{\tt if}}
\newcommand{\IN}{{\tt in}}
\newcommand{\INFIX}{{\tt infix}}
\newcommand{\INFIXR}{{\tt infixr}}
\newcommand{\LET}{{\tt let}}
\newcommand{\LOCAL}{{\tt local}}
\newcommand{\NONFIX}{{\tt nonfix}}
\newcommand{\OF}{{\tt of}}
\newcommand{\OP}{{\tt op}}
\newcommand{\ORELSE}{{\tt orelse}}
\newcommand{\RAISE}{{\tt raise}}
\newcommand{\REC}{{\tt rec}}
\newcommand{\THEN}{{\tt then}}
\newcommand{\TYPE}{{\tt type}}
\newcommand{\VAL}{{\tt val}}
\newcommand{\WITH}{{\tt with}}
\newcommand{\WITHTYPE}{{\tt withtype}}
\newcommand{\WHILE}{{\tt while}}

        % Modules
\newcommand{\EQTYPE}{\ml{eqtype}}
\newcommand{\FUNCTOR}{\ml{functor}}
\newcommand{\INCLUDE}{\ml{include}}
\newcommand{\SIG}{\ml{sig}}
\newcommand{\OPEN}{\ml{open}}
\newcommand{\SHARING}{\ml{sharing}}
\newcommand{\SIGNATURE}{\ml{signature}}
\newcommand{\STRUCT}{\ml{struct}}
\newcommand{\STRUCTURE}{\ml{structure}}
          % Identifier class names
\newcommand{\Var}{{\rm Var}}
\newcommand{\Con}{{\rm Con}}
\newcommand{\SCon}{{\rm SCon}}
\newcommand{\Exn}{{\rm ExCon}}
\newcommand{\TyVar}{{\rm TyVar}}
\newcommand{\ImpTyVar}{{\rm ImpTyVar}}
\newcommand{\AppTyVar}{{\rm AppTyVar}}
\newcommand{\TyCon}{{\rm TyCon}}
\newcommand{\Lab}{{\rm Lab}}
\newcommand{\StrId}{{\rm StrId}}
\newcommand{\FunId}{{\rm FunId}}
\newcommand{\SigId}{{\rm SigId}}

          % Identifier class variables
\newcommand{\id}{{\it id}}
\newcommand{\var}{{\it var}}
\newcommand{\con}{{\it con}}
\newcommand{\scon}{{\it scon}}
\newcommand{\exn}{{\it excon}}
\newcommand{\tyvar}{{\it tyvar}}
\newcommand{\atyvar}{\mbox{\tt \ttprime a}}
\newcommand{\btyvar}{\mbox{\tt \ttprime b}}
\newcommand{\ctyvar}{\mbox{\tt \ttprime c}}
\newcommand{\aityvar}{\mbox{\tt \ttprime\_a}}
\newcommand{\aetyvar}{\mbox{\tt \ttprime\ttprime a}}
\newcommand{\tycon}{{\it tycon}}
\newcommand{\lab}{{\it lab}}
\newcommand{\strid}{{\it strid}}

\newcommand{\longvar}{{\it longvar}}
\newcommand{\longcon}{{\it longcon}}
\newcommand{\longexn}{{\it longexcon}}
\newcommand{\longtycon}{{\it longtycon}}
\newcommand{\longstrid}{{\it longstrid}}

          % General classes
\newcommand{\phrase}{{\it phrase}}

          % Core Syntax Classes
\newcommand{\apexp}{{\it appexp}}
\newcommand{\atpat}{{\it atpat}}
\newcommand{\atexp}{{\it atexp}}
\newcommand{\bind}{{\it bind}}
\newcommand{\constrs}{{\it conbind}}
\newcommand{\ConBind}{{\rm ConBind}}
\newcommand{\datbind}{{\it datbind}}
\newcommand{\dec}{{\it dec}}
\newcommand{\dir}{{\it dir}}
\newcommand{\exnbind}{{\it exbind}}
\renewcommand{\exp}{{\it exp}}
\newcommand{\fvalbind}{{\it fvalbind}}
\newcommand{\handler}{{\it handler}}
\newcommand{\hanrule}{{\it hrule}}
\newcommand{\inexp}{{\it infexp}}
\newcommand{\labexps}{{\it exprow}}
\newcommand{\labpats}{{\it patrow}}
\newcommand{\labtys}{{\it tyrow}}
\newcommand{\match}{{\it match}}
\newcommand{\pat}{{\it pat}}
\newcommand{\mrule}{{\it mrule}}
\newcommand{\ty}{{\it ty}}
\newcommand{\tyseq}{{\it tyseq}}
\newcommand{\tyvarseq}{{\it tyvarseq}}
\newcommand{\typbind}{{\it typbind}}
\newcommand{\valbind}{{\it valbind}}

 % ranged over by the variables




       % Modules syntax classes

\newcommand{\datdesc}{{\it datdesc}}
\newcommand{\condesc}{{\it condesc}}
\newcommand{\exndesc}{{\it exdesc}}
\newcommand{\funbind}{{\it funbind}}
\newcommand{\fundec}{{\it fundec}}
\newcommand{\fundesc}{{\it fundesc}}
\newcommand{\funid}{{\it funid}}
\newcommand{\funsigexp}{{\it funsigexp}}
\newcommand{\funspec}{{\it funspec}}
\newcommand{\funtyp}{{\it funtyp}}
\newcommand{\longstridk}{\strid_1.\cdots.\strid_k}
\newcommand{\longtyconk}{\strid_1.\cdots.\strid_k.\tycon}
\newcommand{\topdec}{{\it topdec}}
\newcommand{\program}{{\it program}}
\newcommand{\sigid}{{\it sigid}}
\newcommand{\shareq}{{\it shareq}}
\newcommand{\sigbind}{{\it sigbind}}
\newcommand{\sigdec}{{\it sigdec}}
\newcommand{\sigexp}{{\it sigexp}}
\newcommand{\spec}{{\it spec}}
\newcommand{\strbind}{{\it strbind}}
\newcommand{\strdec}{{\it strdec}}
\newcommand{\strexp}{{\it strexp}}
\newcommand{\strdesc}{{\it strdesc}}
\newcommand{\typdesc}{{\it typdesc}}
\newcommand{\valdesc}{{\it valdesc}}
           % Core Language Phrases

           % Expressions

\newcommand{\recexp}{\mbox{$\langle\labexps\rangle$}}
\newcommand{\longlabexps}{\mbox{\lab\ \ml{=} \exp\
                          $\langle$ \ml{,} \labexps$\rangle$}}
\newcommand{\parexp}{\mbox{\ml{(} \exp\ \ml{)}}}
\newcommand{\appexp}
           {\mbox{\exp\ \atexp}}
\newcommand{\infexp}
           {\mbox{$\exp_1\ \id\ \exp_2$}}
\newcommand{\opp}
           {\mbox{$\langle\OP\rangle$}}
\newcommand{\typedexp}
           {\mbox{\exp\ \ml{:} \ty}}
\newcommand{\handlexp}
           {\mbox{\exp\ \HANDLE\ \match}}
\newcommand{\raisexp}
           {\mbox{\RAISE\ \exp}}
\newcommand{\letexp}
           {\mbox{\LET\ \dec\ \IN\ \exp\ \END}}
\newcommand{\fnexp}
           {\mbox{\FN\ \match}}

           % Matches and Handlers
\newcommand{\longmatch}
           {\mbox{\mrule\ $\langle$ \ml{|} \match$\rangle$}}
\newcommand{\longmatcha}
           {\mbox{\mrule\ \ml{|} \match}}
\newcommand{\longmrule}
           {\mbox{\pat\ \ml{=>} \exp}}
\newcommand{\longhandler}
           {\mbox{\hanrule\ $\langle$ \ml{||} \handler$\rangle$}}
\newcommand{\longhandlera}
           {\mbox{\hanrule\ \ml{||} \handler}}
\newcommand{\longhanrule}
           {\mbox{\longexn\ \WITH\ \match}}
\newcommand{\lasthanrule}
           {\mbox{? \ml{=>} \exp}}

          % Declarations
\newcommand{\valdec}
           {\mbox{\VAL\ \valbind}}
\newcommand{\valdecS}
           {\mbox{$\VAL_{\U}$\ \valbind}}
\newcommand{\typedec}
           {\mbox{\TYPE\ \typbind}}
\newcommand{\datatypedec}
           {\mbox{\DATATYPE\ \datbind}}
\newcommand{\datatypedeca}
           {\mbox{\DATATYPE\ \datbind\ $\langle\WITHTYPE\ \typbind\rangle$}}
\newcommand{\abstypedec}
           {\mbox{\ABSTYPE\ \datbind\ \WITH\ \dec\ \END}}
\newcommand{\abstypedeca}
           {\mbox{\ABSTYPE\ \datbind\ $\langle\WITHTYPE\ \typbind\rangle$}}
\newcommand{\exceptiondec}
           {\mbox{\EXCEPTION\ \exnbind}}
\newcommand{\localdec}
           {\mbox{\LOCAL\ $\dec_1\ \IN\ \dec_2$\ \END}}
\newcommand{\emptydec}
           {\mbox{\qquad}}
\newcommand{\seqdec}
           {\mbox{$\dec_1\ \langle\ml{;}\rangle\ \dec_2$}}
\newcommand{\longinfix}
           {\mbox{$\INFIX\ \langle d\rangle\ \id_1\ \cdots\ \id_n$}}
\newcommand{\longinfixr}
           {\mbox{$\INFIXR\ \langle d\rangle\ \id_1\ \cdots\ \id_n$}}
\newcommand{\longnonfix}
           {\mbox{$\NONFIX\ \id_1\ \cdots\ \id_n$}}

          % Bindings
\newcommand{\longvalbind}
           {\mbox{\pat\ \ml{=} \exp\ $\langle\AND\ \valbind\rangle$}}
\newcommand{\recvalbind}
           {\mbox{\REC\ \valbind}}
\newcommand{\longtypbind}
           {\mbox{\tyvarseq\ \tycon\ \ml{=} \ty
            \ $\langle\AND\ \typbind\rangle$}}
\newcommand{\longdatbind}
           {\mbox{\tyvarseq\ \tycon\ \ml{=} \constrs
                  \ $\langle\AND\ \datbind\rangle$}}
\newcommand{\longconstrs}
           {\mbox{$\con\ \langle\OF\ \ty\rangle\
                   \langle$ \ml{|} \constrs$\rangle$}}
\newcommand{\longerconstrs}
           {\mbox{$\con\ \langle\OF\ \ty\rangle\
                   \langle\langle$ \ml{|} \constrs$\rangle\rangle$}}
\newcommand{\generativeexnbind}
           {\mbox{$\langle\OP\rangle\exn\ \langle\OF\ \ty\rangle\
                   \langle\AND\ \exnbind\rangle$}}
\newcommand{\eqexnbind}
           {\mbox{$\langle\OP\rangle$\exn\ \ml{=} 
                  $\langle\OP\rangle\longexn\ 
                  \langle\AND\ \exnbind\rangle$}}
\newcommand{\longexnbinda}
           {\mbox{\exn\ $\langle\OF\ \ty\rangle$\ 
                  $\langle\langle\AND\ \exnbind\rangle\rangle$}}
\newcommand{\longexnbindaa}
           {\mbox{\exn\
                  $\langle\AND\ \exnbind\rangle$}}
\newcommand{\longexnbindb}
           {\mbox{\exn\ \ml{=}\ \longexn\ $\langle\AND\ \exnbind\rangle$}}
% from version 1:
%\newcommand{\longexnbind}
%           {\mbox{\exn\ $\langle$\ml{:} $\ty\rangle
%                  \langle$\ml{=} $\longexn\rangle
%                  \ \langle\AND\ \exnbind\rangle$}}
%\newcommand{\longexnbinda}
%           {\mbox{\exn\ $\langle$\ml{:} $\ty\rangle
%                  \ \langle\langle\AND\ \exnbind\rangle\rangle$}}
%\newcommand{\longexnbindaa}
%           {\mbox{\exn\
%                  $\langle\AND\ \exnbind\rangle$}}
%\newcommand{\longexnbindb}
%           {\mbox{\exn\ $\langle$\ml{:} $\ty\rangle
%                  \ $\ml{=} $\longexn
%                  \ \langle\langle\AND\ \exnbind\rangle\rangle$}}
%\newcommand{\longexnbindbb}
%           {\mbox{\exn\ \ml{=} \longexn\
%                  $\langle\AND\ \exnbind\rangle$}}

          % Patterns
\newcommand{\wildpat}{\mbox{\ml{\_}}}
\newcommand{\recpat}{\mbox{$\langle\labpats\rangle$}}
\newcommand{\wildrec}{\mbox{\ml{...}}}
\newcommand{\longlabpats}{\mbox{\lab\ \ml{=} \pat\
                          $\langle$ \ml{,} \labpats$\rangle$}}
\newcommand{\parpat}{\mbox{\ml{(} \pat\ \ml{)}}}
\newcommand{\conpat}
           {\mbox{\longcon\ \atpat}}
\newcommand{\exconpat}
           {\mbox{\longexn\ \atpat}}
\newcommand{\infpat}
           {\mbox{$\pat_1\ \con\ \pat_2$}}
\newcommand{\infexpat}
           {\mbox{$\pat_1\ \exn\ \pat_2$}}
\newcommand{\typedpat}
           {\mbox{\pat\ \ml{:} \ty}}
\newcommand{\layeredpat}
           {\mbox{\var$\langle$\ml{:} \ty$\rangle$ \AS\ \pat}}
\newcommand{\layeredpata}
           {\mbox{\var\ \AS\ \pat}}

          % Types
\newcommand{\rectype}{\mbox{$\langle\labtys\rangle$}}
\newcommand{\longlabtys}{\mbox{\lab\ \ml{:} \ty\
                          $\langle$ \ml{,} \labtys$\rangle$}}
\newcommand{\constype}
           {\mbox{\tyseq\ \longtycon}}
\newcommand{\funtype}
           {\mbox{\ty\ \ml{->} \ty$'$}}
\newcommand{\partype}{\mbox{\ml{(} \ty\ \ml{)}}}
\newcommand{\longtyseq}{\mbox{\ml{(} $\ty_1,\cdots\,\ty_k$ \ml{)}}}
\newcommand{\longtyvarseq}{\mbox{\ml{(} $\tyvar_1,\cdots,\tyvar_k$ \ml{)}}}

        % Modules Phrases

\newcommand{\emptyphrase}{\qquad}


    	% structure-level declarations

\newcommand{\singstrdec}{\mbox{$\STRUCTURE\ \strbind $}}
\newcommand{\localstrdec}{\mbox{$\LOCAL\ \strdec_1\ \IN\ \strdec_2\ \END $}}
\newcommand{\openstrdec}{\mbox{$\OPEN\ \longstrid_1\ \cdots\ \longstrid_n $}}
\newcommand{\emptystrdec}{\emptyphrase}
\newcommand{\seqstrdec}{\mbox{$\strdec_1\ \langle$\ml{;}$\rangle\ \strdec_2 $}}


        % structure bindings

\newcommand{\strbinder}
           {\mbox{$\strid\ \langle$\ml{:}$\ \sigexp\rangle$
            \ml{=} $\strexp\ \langle\langle\AND\ \strbind\rangle\rangle$}}
\newcommand{\strbindera}
           {\mbox{$\strid\ \langle$\ml{:}$\ \sigexp\rangle$
            \ml{=} $\strexp\ \langle\AND\ \strbind\rangle$}}


	% structure expressions

\newcommand{\encstrexp}{\mbox{\STRUCT\ \strdec\ \END}}
\newcommand{\funappdec}{\mbox{\funid\ \ml{(}\ \strdec\ \ml{)} }}
\newcommand{\funappstr}{\mbox{\funid\ \ml{(}\ \strexp\ \ml{)} }}
\newcommand{\letstrexp}{\mbox{\LET\ \strdec\ \IN\ \strexp\ \END}}

        % specifications

\newcommand{\valspec}{\mbox{\VAL\ \valdesc}}
\newcommand{\typespec}{\mbox{\TYPE\ \typdesc}}
\newcommand{\eqtypespec}{\mbox{\EQTYPE\ \typdesc}}
\newcommand{\datatypespec}{\mbox{\DATATYPE\ \datdesc}}
\newcommand{\exceptionspec}{\mbox{\EXCEPTION\ \exndesc}}
\newcommand{\structurespec}{\mbox{\STRUCTURE\ \strdesc}}
\newcommand{\sharingspec}{\mbox{\SHARING\ \shareq}}
\newcommand{\localspec}{\mbox{$\LOCAL\ \spec_1\ \IN\ \spec_2\ \END$}}
\newcommand{\openspec}{\mbox{$\OPEN\ \longstrid_1\ \cdots\ \longstrid_n $}}
\newcommand{\emptyspec}{\emptyphrase}
\newcommand{\seqspec}{\mbox{$\spec_1\ \langle$\ml{;}$\rangle\ \spec_2$}}
\newcommand{\inclspec}{\mbox{$\INCLUDE\ \sigid_1\ \cdots\ \sigid_n $}}


        % descriptions

\newcommand{\valdescription}
           {\mbox{\var\ \ml{:} $\ty\ \langle\AND\ \valdesc\rangle$}}

\newcommand{\typdescription}
           {\mbox{\tyvarseq\ \tycon\ $\langle\AND\ \typdesc\rangle$}}

\newcommand{\datdescription}
           {\mbox{\tyvarseq\ \tycon\ \ml{=} \condesc
             \ $\langle\AND\ \datdesc\rangle$}}

\newcommand{\condescription}
           {\mbox{$\con\ \langle\OF\ \ty\rangle\
                   \langle$ \ml{|} \condesc$\rangle$}}

\newcommand{\longcondescription}
           {\mbox{$\con\ \langle\OF\ \ty\rangle\
                   \langle\langle$ \ml{|} \condesc$\rangle\rangle$}}

\newcommand{\exndescription}
           {\mbox{\exn\ $\langle\OF\ \ty\rangle$
            \ $\langle\AND\ \exndesc\rangle$}}

\newcommand{\exndescriptiona}
           {\mbox{\exn\ $\langle\OF\ \ty\rangle$
            \ $\langle\langle\AND\ \exndesc\rangle\rangle$}}

\newcommand{\strdescription}
           {\mbox{\strid\ \ml{:} \sigexp
            \ $\langle\AND\ \strdesc\rangle$}}

        % sharing equations

\newcommand{\strshareq}{\mbox{$\longstrid_1$ \ml{=} $\cdots$
                                             \ml{=} $\longstrid_n$ }}
\newcommand{\typshareq}{\mbox{$\TYPE\ \longtycon_1$ \ml{=} $\cdots$
                                                    \ml{=} $\longtycon_n$ }}
\newcommand{\multshareq}{\mbox{$\shareq_1\ \AND\ \shareq_2$}}


	% signature expressions

\newcommand{\encsigexp}{\mbox{\SIG\ \spec\ \END}}

        % signature declarations

\newcommand{\singsigdec}{\mbox{$\SIGNATURE\ \sigbind $}}
\newcommand{\emptysigdec}{\emptyphrase}
\newcommand{\seqsigdec}{\mbox{$\sigdec_1\ \langle$\ml{;}$\rangle\ \sigdec_2 $}}


        % signature bindings

\newcommand{\sigbinder}
           {\mbox{\sigid\ \ml{=} \sigexp
            \ $\langle\AND\ \sigbind\rangle$}}


        % functor declarations

\newcommand{\singfundec}{\mbox{$\FUNCTOR\ \funbind $}}
\newcommand{\emptyfundec}{\emptyphrase}
\newcommand{\seqfundec}{\mbox{$\fundec_1\ \langle$\ml{;}$\rangle\ \fundec_2 $}}


        % functor bindings

\newcommand{\funbinder}
           {\mbox{\funid\ \ml{(} \spec\ \ml{)}
                          $\langle$\ml{:}$\ \sigexp\rangle$\ \ml{=} $\strexp$
                          $\langle\langle\AND\ \funbind\rangle\rangle$}}
\newcommand{\funbindera}
           {\mbox{\funid\ \ml{(} \spec\ \ml{)}
                          $\langle$\ml{:}$\ \sigexp\rangle$\ \ml{=} $\strexp$}}
\newcommand{\funstrbinder}
           {\mbox{\funid\ \ml{(}\ \strid\ \ml{:}\ \sigexp\ \ml{)}
                        $\langle$\ml{:}$\ \sigexp'\rangle$\ \ml{=} $\strexp$}}
\newcommand{\optfunbind}
           {\mbox{$\langle\langle\AND\ \funbind\rangle\rangle$}}
\newcommand{\optfunbinda}
           {\mbox{$\langle\AND\ \funbind\rangle$}}
\newcommand{\funstrbindera}
           {\mbox{\funid\ \ml{(}\ \strid\ \ml{)}\ \ml{=} \strexp
            \ $\langle\AND\ \funbind\rangle$}}


        %functor signature expressions

\newcommand{\longfunsigexp}
           {\mbox{\ml{(}\ \spec\ \ml{)}
                          \ml{:}\ \sigexp}}
\newcommand{\longfunsigexpa}
           {\mbox{\ml{(}\ \strid\ \ml{:}\ \sigexp\ \ml{)}
                          \ml{:}\ \sigexp$'$}}


        % functor specifications

\newcommand{\singfunspec}{\mbox{\FUNCTOR\ \fundesc}}
\newcommand{\emptyfunspec}{\emptyphrase}
\newcommand{\seqfunspec}
           {\mbox{$\funspec_1\ \langle$\ml{;}$\rangle\ \funspec_2$}}
 

        % functor descriptions

\newcommand{\longfundesc}
           {\mbox{\funid\ \funsigexp\ $\langle\AND\ \fundesc\rangle$}}


	% programs
\newcommand{\longprog}{\mbox{\topdec\ \ml{;}\ $\langle\program\rangle$}}
\newcommand{\seqprog}
 {\mbox{$\program_1\ \langle$\ml{;}$\rangle\ \program_2$}}
% **************** END SYNTAX *************************
\newcommand{\ML}{{\rm ML}}


% 	finite sets and maps (assume math mode)
%
\newcommand{\Fin}{\mathop{\rm Fin}\nolimits}
\newcommand{\Dom}{\mathop{\rm Dom}\nolimits}
\newcommand{\Ran}{\mathop{\rm Ran}\nolimits}
\newcommand{\finfun}[2]{#1\stackrel{{\rm fin}}{\to}#2}
\newcommand{\emptymap}{\{\}}
\newcommand{\kmap}[2]{\{#1_1\mapsto#2_1,\cdots,#1_k\mapsto#2_k\}}
\newcommand{\plusmap}[2]{#1 + #2}
%
\newcommand{\TyVarSet}{{\rm TyVarSet}}
%
%
%	Names     (assume math mode)
\newcommand{\TyNames}{{\rm TyName}}
\newcommand{\TyNameSets}{{\rm TyNameSet}}
\newcommand{\StrNames}{{\rm StrName}}
\newcommand{\StrNameSets}{{\rm StrNameSet}}

\newcommand{\NameSets}{{\rm NameSet}}
\newcommand{\TyNamesk}{\TyNames^{(k)}}
\newcommand{\Addr}{{\rm Addr}}
\newcommand{\Exc}{{\rm ExName}}
\newcommand{\BasVal}{{\rm BasVal}}
\newcommand{\SVal}{{\rm SVal}}
\newcommand{\BasExc}{{\rm BasExName}}
\newcommand{\BasTyp}{{\rm BasTyp}}
\newcommand{\CONT}{{\tt !}}
\newcommand{\ASS}{{\tt :=}}
\newcommand{\FAIL}{{\rm FAIL}}
\newcommand{\Fail}{{\rm FAIL}}
\newcommand{\fail}{{\rm FAIL}}
\newcommand{\APPLY}{{\rm APPLY}}

\newcommand{\A}{a}
\newcommand{\e}{\mbox{\it en}}
\newcommand{\sv}{\mbox{\it sv}}
\newcommand{\exval}{e}
\newcommand{\excs}{\mbox{\it ens}}
\newcommand{\exns}{\mbox{\it excons}} % used in the dynamic sem. of mod.
\newcommand{\f}{f}
\newcommand{\m}{m}
\newcommand{\mem}{\mbox{\it mem}}
\newcommand{\M}{M}
\newcommand{\n}{n}
\newcommand{\N}{N}
\newcommand{\p}{p}
\renewcommand{\r}{r}
\newcommand{\res}{\mbox{\it res}}
\newcommand{\s}{s}
\renewcommand{\t}{t}
\newcommand{\T}{T}
\newcommand{\U}{U}
\newcommand{\V}{v}
\newcommand{\vars}{\mbox{\it vars}}
\newcommand{\X}{X}

          % Compound Objects (Core Language)
\newcommand{\TyEnv}{{\rm TyEnv}}
\newcommand{\TE}{\mbox{$T\!E$}}
\newcommand{\TyStr}{{\rm TyStr}}

\newcommand{\ConEnv}{{\rm ConEnv}}
\newcommand{\CE}{\mbox{$C\!E$}}

\newcommand{\VarEnv}{{\rm VarEnv}}
\newcommand{\VE}{\mbox{$V\!E$}}

\newcommand{\ExnEnv}{{\rm ExConEnv}}
\newcommand{\EE}{\mbox{$E\!E$}}

\newcommand{\IntEnv}{{\rm IntEnv}}
\newcommand{\IE}{\mbox{$I\!E$}}

\newcommand{\Env}{{\rm Env}}
\newcommand{\E}{E}
\newcommand{\longE}[1]{(\SE_{#1},\TE_{#1},\VE_{#1},\EE_{#1})}

\newcommand{\StrEnv}{{\rm StrEnv}}
\newcommand{\SE}{\mbox{$S\!E$}}

\newcommand{\Str}{{\rm Str}}
\renewcommand{\S}{S}
\newcommand{\longS}[1]{(\m_{#1},(\SE_{#1},\TE_{#1},\VE_{#1},\EE_{#1}))}

\newcommand{\Int}{{\rm Int}}
\newcommand{\I}{I}

\newcommand{\Context}{\rm Context}
\newcommand{\C}{C}


\newcommand{\Record}{{\rm Record}}
\newcommand{\ExVal}{{\rm ExVal}}
\newcommand{\Fun}{{\rm Fun}}  % used?
\newcommand{\Pack}{{\rm Pack}}
\newcommand{\Closure}{{\rm Closure}}
\newcommand{\FunctorClosure}{{\rm FunctorClosure}}
\newcommand{\Match}{{\rm Match}}
\newcommand{\State}{{\rm State}}
\newcommand{\StrExp}{{\rm StrExp}}
\newcommand{\Mem}{{\rm Mem}}
\newcommand{\ExcSet}{{\rm ExNameSet}}
\newcommand{\Val}{{\rm Val}}

\newcommand{\arity}{\mathop{\rm arity}\nolimits}
\renewcommand{\k}{\mbox{$k$}}
\newcommand{\longtauk}{(\tau_1,\cdots,\tau_k)}
\newcommand{\tauk}{\mbox{$\tau^{(k)}$}}
\newcommand{\longalphak}{(\alpha_1,\cdots,\alpha_k)}
\newcommand{\alphak}{\mbox{$\alpha^{(k)}$}}
\newcommand{\alphakt}{\mbox{$\alpha^{(k)}t$}}
%
\newcommand{\thetak}{\mbox{$\theta^{(k)}$}}
\newcommand{\tk}{\mbox{$\t^{(k)}$}}
\newcommand{\typefcn}{\theta}
\newcommand{\typefcnk}{\Lambda\alphak.\tau}
\newcommand{\Type}{{\rm Type}}
\newcommand{\ConsType}{{\rm ConsType}}
\newcommand{\RecType}{{\rm RecType}}
\newcommand{\FunType}{{\rm FunType}}
\newcommand{\constypek}{\t(\tau_1,\cdots,\tau_k)}
\newcommand{\TypeScheme}{{\rm TypeScheme}}
\newcommand{\tych}{\sigma}
\newcommand{\longtych}{\forall\alphak.\tau}
\newcommand{\Abs}{\mathop{\rm Abs}\nolimits}
\newcommand{\Inter}{\mathop{\rm Inter}\nolimits}
\newcommand{\Rec}{\mathop{\rm Rec}\nolimits}
\newcommand{\TypeFcn}{{\rm TypeFcn}}
\newcommand{\TyConFcn}{\mathop{\rm TyCon}\nolimits} % used?
\newcommand{\TyVarFcn}{\mathop{\rm tyvars}\nolimits}
\newcommand{\imptyvars}{\mathop{\rm imptyvars}\nolimits}
\newcommand{\apptyvars}{\mathop{\rm apptyvars}\nolimits}
\newcommand{\scontype}{\mathop{\rm type}\nolimits}
\newcommand{\sconval}{\mathop{\rm val}\nolimits}
%
%         Compound Objects (Modules)
%
%
\newcommand{\Sig}{{\rm Sig}}
\newcommand{\sig}{\Sigma}
\newcommand{\longsig}[1]{(\N_{#1})\S_{#1}}
\newcommand{\FunSig}{{\rm FunSig}}
\newcommand{\funsig}{\Phi}
\newcommand{\longfunsig}[1]{(\N_{#1})(\S_{#1},(\N_{#1}')\S_{#1}
')}
\newcommand{\FunEnv}{{\rm FunEnv}}
\newcommand{\F}{F}

\newcommand{\SigEnv}{{\rm SigEnv}}
\newcommand{\G}{G}

\newcommand{\Basis}{{\rm Basis}}
\newcommand{\B}{B}
\newcommand{\Bstat}{B_{\rm STAT}}
\newcommand{\Bdyn}{B_{\rm DYN}}

\newcommand{\IntBasis}{{\rm IntBasis}}
\newcommand{\IB}{\mbox{$I\!B$}}


%         Selection of Components
%
\newcommand{\of}[2]{#1 \mathbin{\rm of} #2}
\newcommand{\In}{\mbox{\rm in}}
%
%         Names in Structures
\newcommand{\StrNamesFcn}{\mathop{\rm strnames}\nolimits}
\newcommand{\TyNamesFcn}{\mathop{\rm tynames}\nolimits}
\newcommand{\TyVarsFcn}{\mathop{\rm tyvars}\nolimits}
\newcommand{\NamesFcn}{\mathop{\rm names}\nolimits}
%
%         Type Schemes (assume math mode)
%
\newcommand{\cl}[2]{{\rm Clos}_{#1}#2}
%
%         Realisations (assume math mode)
%
\newcommand{\tyrea}{\varphi_{\rm Ty}}
\newcommand{\strrea}{\varphi_{\rm Str}}
\newcommand{\rea}{\varphi}
\newcommand{\longrea}{(\tyrea,\strrea)}
%
%             Support (assume math mode)
%
\newcommand{\Supp}{\mathop{\rm Supp}\nolimits}
\newcommand{\Yield}{\mathop{\rm Yield}\nolimits}
%
%           Instantiation (assume math mode)
%
\newcommand{\siginst}[3]{#1 {\geq_{#2}} #3}
\newcommand{\sigord}[3]{#1 {\geq_{#2}} #3}
\newcommand{\funsiginst}[3]{#1 {\geq_{#2}} #3}
%
%            Inference Rules
%
%
\newcommand{\ts}{\vdash}
\newcommand{\tsdyn}{\vdash_{\rm DYN}}
\newcommand{\tsstat}{\vdash_{\rm STAT}}

\newcommand{\ra}{\Rightarrow}

%            Initial Static Basis

%              Type names
\newcommand{\BOOL}{\mbox{\tt bool}}
\newcommand{\INT}{\mbox{\tt int}}
\newcommand{\REAL}{\mbox{\tt real}}
\newcommand{\NUM}{\mbox{\tt num}}
\newcommand{\EXCN}{\mbox{\tt exn}}
\newcommand{\STRING}{\mbox{\tt string}}
\newcommand{\LIST}{\mbox{\tt list}}
\newcommand{\INSTREAM}{\mbox{\tt instream}}
\newcommand{\OUTSTREAM}{\mbox{\tt outstream}}

%              Constructors
\newcommand{\FALSE}{\mbox{\tt false}}
\newcommand{\TRUE}{\mbox{\tt true}}
\newcommand{\NIL}{\mbox{\tt nil}}
\newcommand{\REF}{\mbox{\tt ref}}
\newcommand{\UNIT}{\mbox{\tt unit}}

%              Basic Values BasVal
%\newcommand{\MAP}{\mbox{\tt map}}
%\newcommand{\REV}{\mbox{\tt rev}}
%\newcommand{\NOT}{\mbox{\tt not}}
%\newcommand{\NEG}{\mbox{\verb-~-}}
%\newcommand{\ABS}{\mbox{\tt abs}}
%\newcommand{\FLOOR}{\mbox{\tt floor}}
%\newcommand{\REAL}{\mbox{\tt real}}
%\newcommand{\SQRT}{\mbox{\tt sqrt}}
%\newcommand{\SIN}{\mbox{\tt sin}}
%\newcommand{\COS}{\mbox{\tt cos}}
%\newcommand{\ARCTAN}{\mbox{\tt arctan}}
%\newcommand{\EXP}{\mbox{\tt exp}}
%\newcommand{\LN}{\mbox{\tt ln}}
%\newcommand{\SIZE}{\mbox{\tt size}}
%\newcommand{\CHR}{\mbox{\tt chr}}
%\newcommand{\ORD}{\mbox{\tt ord}}
%\newcommand{\EXPLODE}{\mbox{\tt explode}}
%\newcommand{\IMPLODE}{\mbox{\tt implode}}
%\newcommand{\REALDIV}{\mbox{\tt /}}
%\newcommand{\DIV}{\mbox{\tt div}}
%\newcommand{\MOD}{\mbox{\tt mod}}
%\newcommand{\TIMES}{\mbox{\tt  *}}
%\newcommand{\PLUS}{\mbox{\tt +}}
%\newcommand{\MINUS}{\mbox{\tt -}}
%\newcommand{\APPEND}{\mbox{\verb-@-}}
%\newcommand{\EQ}{\mbox{\verb-=-}}
%\newcommand{\NEQ}{\mbox{\verb-<>-}}
%\newcommand{\LESS}{\mbox{\verb-<-}}
%\newcommand{\GREATER}{\mbox{\verb->-}}
%\newcommand{\LEQ}{\mbox{\verb-<=-}}
%\newcommand{\GEQ}{\mbox{\verb->=-}}
%\newcommand{\COMP}{\mbox{\tt o}}

%\newcommand{\STDIN}{\mbox{\tt std\_in}}
%\newcommand{\OPENIN}{\mbox{\tt open\_in}}
%\newcommand{\INPUT}{\mbox{\tt input}}
%\newcommand{\LOOKAHEAD}{\mbox{\tt lookahead}}
%\newcommand{\CLOSEIN}{\mbox{\tt close\_in}}
%\newcommand{\ENDSTREAM}{\mbox{\tt end\_of\_stream}}
%\newcommand{\STDOUT}{\mbox{\tt std\_out}}
%\newcommand{\OPENOUT}{\mbox{\tt open\_out}}
%\newcommand{\OUTPUT}{\mbox{\tt output}}
%\newcommand{\CLOSEOUT}{\mbox{\tt close\_out}}
%\newcommand{\IOFAILURE}{\mbox{\tt io\_failure}}
\newcommand{\comment}{{\it Comment:\ }}
\newcommand{\comments}{{\it Comments:\ }}
\newcommand{\rulesec}[2]{\subsection*{{\bf#1}\hfill\fbox{$#2$}}}
%
\font\msxm=msxm10
\textfont10=\msxm
\mathchardef\restrict="0A16
%
% $$f\restrict_A$$                   % Example of use
%alignment
%\halign{\indent$#$&\quad$#$&\quad$#$\hfil&\quad\parbox[t]{6cm}{\strut#\strut}\cr
 % macros
\makeindex
%
%\includeonly{syncor}
\voffset -12mm
\begin{document}
\pagestyle{empty}
\maketitle
\cleardoublepage
\pagestyle{plain}
\setcounter{page}{3}
\renewcommand{\thepage}{\roman{page}}
\include{preface3}
\include{preface2}
\include{preface1}
\tableofcontents
\cleardoublepage
\pagestyle{headings}
\setcounter{page}{1}
\renewcommand{\thepage}{\arabic{page}}
\include{intro}
\section{Syntax of the Core}
\label{syn-core-sec}

\subsection{Reserved Words}
The\index{6.1}
following are the {\sl reserved words} used in the Core.   They
may not (except ~{\tt =}~) be used as identifiers.   
%In this document the alphabetic reserved words are 
%always shown in typewriter font.

\vspace*{-6pt}
\begin{verbatim}
     abstype   and   andalso   as   case   do   datatype   else
     end    exception    fn    fun    handle    if   in   infix
     infixr   let     local    nonfix   of   op   open   orelse
     raise   rec   then   type   val   with   withtype    while
     (  )   [  ]   {  }  ,  :  ;  ...    _   |  =   =>   ->   #   
\end{verbatim}
\vspace*{-6pt}

\subsection{Special constants}
\label{cr:speccon}
An\index{6.2} {\sl integer constant} 
is any non-empty sequence of digits, possibly preceded
by a negation symbol (\verb+~+).
A {\sl real constant} is an integer constant,
possibly followed by a point ({\tt .}) and one or
more digits, possibly followed by an exponent symbol ~{\tt E}~ and an integer
constant; at least one of the optional parts must occur, hence no integer
constant is a real constant.
Examples: ~~{\tt 0.7}~~~{\tt 3.32E5}~~~\verb(3E~7(~~.  Non-examples:
%Examples: ~~{\tt 0.7}~~~{\tt +3.32E5}~~~\verb(3E~7(~~.  Non-examples:
%Examples: ~~{\tt 0.7}~~~{\tt +3.32E5}~~~{\tt 3E~}~~.  Non-examples:
~~{\tt 23}~~~{\tt .3}~~~{\tt 4.E5}~~~{\tt 1E2.0}~~.

We assume an underlying alphabet of 256 characters 
(numbered 0 to 255) such that the characters with numbers 
0 to 127 coincide with the ASCII character set.
A {\sl string constant} is a sequence, between quotes ({\tt "}), of zero or
more printable characters (i.e., numbered 33--126), spaces or escape 
sequences.
Each escape sequence starts with the
escape character ~\verb+\+~, and stands for a character sequence. The
escape sequences are:
\smallskip

%\begin{quote}
%\begin{tabular}{ll}
\halign{\indent#\hfil&\quad\parbox[t]{11cm}{\strut#\strut}\cr
\verb+\n+   & A single character interpreted by the system as
end-of-line.\cr
\verb+\t+   & Tab.\cr
\verb+\^+$c$  & The control character $c$, where $c$ may
                be any character with number 64--95. The number
                of ~{\tt\char'134\char'136}$c$~ is 64 less than the 
                number of $c$.\cr
\verb+\+$ddd$ & The single character with number $ddd$ (3 decimal digits
denoting an integer in the interval $[0,255]$).\cr
\verb+\"+   & {\tt "}\cr
\verb+\\+   & {\tt\char'134}\cr
\verb+\+$f\cdot\cdot f$\verb+\+
            & This sequence is ignored,
              where $f\cdot\cdot f$ stands for a sequence 
             of one or more formatting characters.\cr
}
%\end{tabular}
%\end{quote}
\smallskip

The {\sl formatting characters}\index{6.3} are a subset of the non-printable
            characters including at least space, tab, newline, formfeed.
The last form allows long strings to be written on more than one line, by
            writing ~\verb+\+~ at the end of one line and at the start of the
            next.\nopagebreak

We denote by {\SCon} the class of {\sl special constants}, i.e., the integer,
real, and string constants; we shall use {\scon}
to range over \SCon.\index{6.4}

\subsection{Comments}
A\index{7.1} {\sl comment} 
is any character sequence within comment brackets ~{\tt (* *)}~
in which
comment brackets are properly nested. An unmatched comment bracket should be
detected by the compiler.

%
\subsection{Identifiers}
\label{cyn-core-identifiers-sec}
The classes of {\sl identifiers}\index{7.2} for the Core are shown in
Figure~\ref{identifiers}.
\begin{figure}
\vspace{4pt}
\makeatletter{}
\tabskip\@centering
\halign to\textwidth
{#\hfil\tabskip1em&(#)\hfil\tabskip1em&#\hfil\tabskip\@centering\cr
\Var	& value variables	& long\cr
\Con	& value constructors	& long\cr
\Exn    & exception constructors& long\cr
\TyVar	& type variables	& \cr
\TyCon  & type constructors     & long\cr
\Lab    & record labels         & \cr
\StrId  & structure identifiers & long\cr
}
\makeatother
\caption{Identifiers}
\label{identifiers}
\vspace*{-3mm}
\end{figure}
We use $\var$, $\tyvar$ to range over Var, TyVar etc.  For each class
X marked ``long'' there is a class longX of {\sl long identifiers}; if
$x$ ranges over X then {\it longx} ranges over longX.  The syntax of
these long identifiers is given by the following: 
\vspace*{-6pt}
\begin{quote}
\begin{tabular}{rcll} {\it longx} & $::=$ & $x$ & identifier\\
& &$\strid_1.\cdots.\strid_n.x$ & qualified identifier ($n\geq 1$)
\end{tabular} 
\end{quote}
\vspace*{-6pt}
The qualified identifiers constitute a link between the Core and the
Modules. Throughout this document, the term ``identifier'', occurring 
without an adjective, refers to non-qualified identifiers only.
%version 2: For each class X marked
%``long'' there is also a class
%\[ {\rm LongX} = \StrId^\ast \times {\rm X} \]
%If $x$ ranges over X, then {\it longx}, or
%$\strid_1.\cdots.\strid_k.x$, $k\geq 0$, ranges over LongX.
%These long identifiers constitute the only link between the Core
%and the language constructs for Modules; by omitting them, and the $\OPEN$
%declaration,
%we obtain the Core as a complete programming language in
%its own right. (The corresponding adjustment to the Core static and
%dynamic semantics is simply to omit structure environments $\SE$.).

An identifier is either {\sl alphanumeric}: any sequence of
letters, digits, primes ({\tt '}) and underbars (\wildpat) starting
with a letter or prime, or {\sl symbolic}: any non-empty sequence of the
following {\sl symbols}\index{7.3}
\vspace*{-6pt}
\begin{center}
%\verb(!  %  &  $  +  -  /  :  <  =  >  ?  (@\verb(  \  ~  `  ^  |  *(
\verb(!  %  &  $  #  +  -  /  :  <  =  >  ?  @  \  ~  `  ^  |  *(
\end{center}
\vspace*{-6pt}
In either case, however, reserved words are excluded.   This means that for
example ~\verb+#+~ and ~{\tt |}~ are not identifiers, but  ~\verb+##+~ and
~{\tt |=|}~ are identifiers.
The only exception to this rule is that the symbol ~{\tt =}~, which is
a reserved word, is also allowed as an identifier to stand for
the equality predicate.
The identifier ~{\tt =}~ may not be re-bound;
this precludes any syntactic ambiguity.

A type variable $\tyvar$\index{7.4}\label{etyvar-lab} may be any
alphanumeric identifier starting with a prime; the subclass EtyVar of
TyVar, the {\sl equality} type variables, consists of those which
start with two or more primes.  
%poly 
The subclass $\ImpTyVar$ of
$\TyVar$, the {\sl imperative} type variables, consists of those which
start with one or two primes followed by an underbar.  The complement
$\AppTyVar=\TyVar\setminus\ImpTyVar$\index{8.1} consists of the {\sl
applicative} type variables.  The other six classes ({\Var}, {\Con},
{\Exn}, {\TyCon}, {\Lab} and {\StrId}) are represented by identifiers
not starting with a prime. However,\index{7.5} {\tt *} is excluded from {\TyCon},
to avoid confusion with the derived form of tuple type (see
Figure~\ref{typ-gram}). The class Lab\index{8.2} is extended to
include the {\em numeric} labels ~{\tt 1}~~{\tt 2}~~{\tt 3}~ $\cdots$,
i.e. any numeral not starting with~{\tt 0}.

TyVar is therefore disjoint
from the other six classes.   Otherwise, the syntax class of an occurrence of
identifier $\id$ in a Core phrase (ignoring derived forms, 
Section~\ref{cor-der-form-sec}) is determined thus:
\begin{enumerate}
  \item Immediately before ``.'' -- i.e. in a long identifier -- or in an
        $\OPEN$ declaration, $\id$ is a structure
        identifier.  The following rules assume that all occurrences of
        structure identifiers have been removed.
  \item At the start of a component in a record type, record pattern or record
        expression,  $\id$ is a record label.
  \item Elsewhere in types $\id$ is a type constructor, and must be within the
        scope\index{8.3} of the type binding or datatype binding which introduced it.
%  \item Elsewhere $\id$ is an exception name if it occurs immediately after
%        $\RAISE$, at the start of a handler rule $\hanrule$, or within an
%        $\EXCEPTION$ declaration or specification.
  \item Elsewhere, $\id$ is an exception constructor if it occurs in
        the scope of an exception binding which introduces it as such, 
        or a value constructor if it occurs in the
        scope of a datatype binding which introduced it as such;
        otherwise it is a value variable.
\end{enumerate}
It follows from the last rule that no value declaration can make a
``hole'' in the scope of a value or exception constructor 
by introducing the same identifier as a variable; this
is because, in the scope of the declaration which introduces $\id$ as a value
or exception constructor, any occurrence of $\id$ in a pattern 
is interpreted as the
constructor and not as the binding occurrence of a new variable.

By means of the above rules a compiler can determine the class to which each
identifier occurrence belongs; for the remainder of this document we shall
therefore assume that the classes are all disjoint.

\subsection{Lexical analysis}
Each\index{8.4} item of lexical analysis is either a reserved word, a numeric label, a
special constant or a long identifier.
Comments and formatting characters
separate items (except within string constants; see Section~\ref{cr:speccon})
and are otherwise
ignored.   At each stage the longest next item is taken.

\subsection{Infixed operators}
An\index{8.5} identifier may be given {\sl infix status} by the
~$\INFIX$~ or ~$\INFIXR$~ directive , which may occur as a
declaration; this status only pertains to its use as a $\var$, a
$\con$ or an $\exn$ within the scope (see below) of the directive.
(Note that qualified identifiers never have infix status.)  If $\id$
has infix status, then ``$\exp_1\ \id\ \exp_2$'' (resp. ``$\pat_1\
\id\ \pat_2$'') may occur -- in parentheses if necessary -- wherever
the application ``$\id$\verb+{+{\tt 1=}$\exp_1$\verb+,+{\tt
2=}$\exp_2$\verb+}+'' or its derived form
``$\id$\verb+(+$\exp_1$\verb+,+$\exp_2$\verb+)+'' (resp
``$\id$\verb+(+$\pat_1$\verb+,+$\pat_2$\verb+)+'') would otherwise
occur.  On the other hand, an occurrence of any long identifier (qualified
or not) prefixed by {\OP} is treated as non-infixed. The only required
use of {\OP} is in prefixing a non-infixed occurrence of an
identifier $\id$ which has infix status; elsewhere {\OP}, where
permitted, has no effect.\index{9.1}
%version 2: On the other 
%hand, non-infixed occurrences of $\id$ must be prefixed by the reserved word
%~$\OP$~.\index{9.1} 
Infix status is cancelled by the ~$\NONFIX$~
directive.  We refer to the three directives collectively as {\sl
fixity directives}.

The form of the fixity directives is as follows ($n\geq 1$):
\[ \longinfix \]
\[ \longinfixr \]
\[ \longnonfix \]
where $\langle d\rangle$ is an optional decimal digit $d$ indicating
binding precedence. A higher value of $d$ indicates tighter binding;
the default is $0$.  ~$\INFIX$~ and ~$\INFIXR$~ dictate left and right
associativity respectively; association is always to the left for different
operators of the same precedence.  The precedence of infix operators relative
to other expression and pattern constructions is given in
Appendix~\ref{core-gram-app}.

The {\sl scope}\index{9.2} of a fixity directive $\dir$ is the ensuing program text,
except that if $\dir$ occurs in a declaration $\dec$ in either of the phrases
\[ \LET\ \dec\ \IN\ \cdots\ \END \]
\[ \LOCAL\ \dec\ \IN\ \cdots\ \END \]
then the scope of $\dir$ does not extend beyond the phrase. Further scope
limitations are imposed for Modules.

These directives and ~$\OP$~ are omitted from the semantic rules, since they
affect only parsing.

\subsection{Derived Forms}
\label{cor-der-form-sec}
There\index{9.3} are many standard syntactic forms in ML whose meaning can be expressed
in terms of a smaller number of syntactic forms, called the {\sl bare} language.
These derived forms, and their equivalent forms in the bare language, are
given in
Appendix~\ref{derived-forms-app}.

%With one exception, these derived forms use no new lexical items.  The
%exception is that the symbol \verb+#+ prefixed to an identifier of the
%class Lab constitutes a
%lexical item;  \verb+#+{\it lab} denotes a selector function on records, cf. page~\pageref{der-exp}.

\subsection{Grammar}

The phrase classes for the Core are shown in Figure~\ref{cor-phr}.
We use the variable $\atexp$ to range over AtExp, etc.

The grammatical rules for the Core are shown in Figures~\ref{exp-syn}
and \ref{pat-syn}.

\clearpage
\begin{figure}[t]
\vspace{4pt}
\makeatletter{}
\tabskip\@centering
\halign to\textwidth
{#\hfil\tabskip1em&#\hfil\tabskip\@centering\cr
AtExp	& atomic expressions \cr
ExpRow  & expression rows \cr
Exp     & expressions \cr
Match   & matches \cr
Mrule   & match rules \cr
\noalign{\vspace{2mm}}
%\cr
Dec     & declarations \cr
ValBind & value bindings \cr
TypBind & type bindings \cr
DatBind & datatype bindings \cr
ConBind & constructor bindings \cr
%version 1: Constrs & datatype constructions \cr
ExBind  & exception bindings \cr
\noalign{\vspace{2mm}}
%\cr
AtPat   & atomic patterns \cr
PatRow  & pattern rows \cr
Pat     & patterns \cr
\noalign{\vspace{2mm}}
%\cr
Ty      & type expressions \cr
TyRow   & type-expression rows \cr
}
\makeatother
\caption{Core Phrase Classes}
\label{cor-phr}
\end{figure}

The following\index{10.1} conventions are adopted in presenting the grammatical rules,
and in their interpretation:
\begin{itemize}
  \item The brackets\index{10.2} ~$\langle\ \rangle$~ enclose optional phrases.
  \item For any syntax class X (over which $x$ ranges)
we define the syntax class Xseq (over which {\it xseq} ranges) as follows:
    \begin{quote}
    \begin{tabular}{rcll}
       {\it xseq} & $::=$ & $x$ & (singleton sequence)\\
                  &       &     & (empty sequence)\\
                  &       & \ml{(}$x_1$\ml{,}$\cdots$\ml{,}$x_n$\ml{)}
                                & (sequence,~$n\geq 1$) \\
    \end{tabular}
    \end{quote}
(Note that the ``$\cdots$'' used here, meaning syntactic iteration, must not be
confused with ``$\wildrec$'' which is a reserved word of the language.)
  \item Alternative forms for each phrase class are in order of decreasing
        precedence; this resolves ambiguity in parsing, as explained
        in Appendix~\ref{core-gram-app}.\index{10.3}
  \item L (resp. R)\index{10.4} means left (resp. right) association.

\item The syntax of types binds more tightly than that of expressions.

\item Each\index{10.6} iterated construct (e.g. $\match$, $\cdots$)
extends as far
right as possible; thus, parentheses may be needed around an expression which
terminates with a match, e.g. ``$\FN\ \match$'', if this occurs within a
larger
match.
\end{itemize}


\begin{figure}[t]
\vspace{4pt}
\makeatletter{}
\tabskip\@centering
\halign to\textwidth
{#\hfil\tabskip1em&\hfil$#$\hfil&#\hfil&#\hfil\tabskip\@centering\cr
  \atexp& ::=	& \scon 	& special constant\cr
        & 	& \opp\longvar	& value variable\cr
	&	& \opp\longcon	& value constructor\cr
        &       & \opp\longexn  & exception constructor\cr
	&	& \verb+{ +\recexp\verb+ }+
	                	& record\cr
	&	& \letexp	& local declaration\cr
	&	& \parexp	& \cr
\noalign{\vspace{6pt}}
\labexps& ::=	& \longlabexps	& expression row\cr
\noalign{\vspace{6pt}}
  \exp  & ::=	& \atexp 	& atomic\cr
	&	& \appexp	& application (L)\cr
	&	& \infexp       & infixed application\cr
	&	& \typedexp	& typed (L)\cr
	&	& \handlexp	& handle exception\cr
	&	& \raisexp	& raise exception\cr
	&	& \fnexp        & function\cr
\noalign{\vspace{6pt}}
\match  & ::=	& \longmatch    & \cr
\noalign{\vspace{6pt}}
\mrule	& ::=	& \longmrule	& \cr
\noalign{\vspace{6pt}}
  \dec  & ::=	& \valdec	& value declaration\cr
	&	& \typedec	& type declaration\cr
	&	& \datatypedec  & datatype declaration\cr
	&	& \abstypedec   & abstype declaration\cr
	&	& \exceptiondec & exception declaration\cr
	&	& \localdec	& local declaration\cr
        &       & \openstrdec   & open declaration ($n\geq 1$) \cr
	&	& \emptydec	& empty declaration\cr
	&	& \seqdec	& sequential declaration\cr
        &       & \longinfix    & infix (L) directive\cr
        &       & \longinfixr   & infix (R) directive\cr
        &       & \longnonfix   & nonfix directive\cr
\noalign{\vspace{6pt}}
\valbind& ::=   & \longvalbind   & \cr
	&	& \recvalbind	& \cr
\noalign{\vspace{6pt}}
\typbind& ::=	& \longtypbind	& \cr
\noalign{\vspace{6pt}}
\datbind& ::=	& \longdatbind	& \cr
\noalign{\vspace{6pt}}
\constrs& ::=	& \opp\longconstrs & \cr
\noalign{\vspace{6pt}}
\exnbind& ::=	& \generativeexnbind	& \cr
        &       & \eqexnbind   & \cr
\noalign{\vspace{6pt}}
}
\makeatother
\vspace{-2mm}
\caption{Grammar: Expressions, Matches, Declarations and Bindings\index{11}\index{12.1}}
\label{exp-syn}
\end{figure}
\clearpage % 1 August
\begin{figure}[h]
%\vspace{4pt}
\makeatletter{}
\tabskip\@centering
\halign to\textwidth
{#\hfil\tabskip1em&\hfil$#$\hfil&#\hfil&#\hfil\tabskip\@centering\cr
  \atpat& ::=	& \wildpat	& wildcard\cr
  	&	& \scon  	& special constant\cr
  	&	& \opp\var  	& variable\cr
	&	& \opp\longcon  & constant\cr
        &       & \opp\longexn  & exception constant\cr
	&	& \verb+{ +\recpat\verb+ }+
	                        & record\cr
	&	& \parpat       & \cr
\noalign{\vspace{6pt}}
\labpats& ::=	& \wildrec	& wildcard\cr
  	&	& \longlabpats 	& pattern row\cr
\noalign{\vspace{6pt}}
  \pat	&	& \atpat	& atomic\cr
	&	& \opp\conpat	& value construction\cr
        &       & \opp\exconpat  & exception construction\cr
	&	& \infpat       & infixed value construction\cr
        &       & \infexpat     & infixed exception construction\cr
	&	& \typedpat	& typed\cr
	&	& \opp\layeredpat	& layered\cr
\noalign{\vspace{6pt}}
  \ty   & ::=	& \tyvar        & type variable\cr
	&	& \verb+{ +\rectype\verb+ }+
	                        & record type expression\cr
	&	& \constype 	& type construction\cr
	&	& \funtype      & function type expression (R)\cr
	&	& \partype      & \cr
\noalign{\vspace{6pt}}
\labtys & ::=	& \longlabtys   & type-expression row\cr
\noalign{\vspace{6pt}}
}
\makeatother
\vspace{-2mm}
\caption{Grammar: Patterns and Type expressions\index{12.2}\index{13.1}}
\label{pat-syn}
\end{figure}
\nopagebreak[4]
\subsection{Syntactic Restrictions}\index{13.2}
\begin{itemize}
\item No pattern may contain the same $\var$ twice. No expression row,
      pattern row or type row may bind the same $\lab$ twice.
\item No binding $\valbind$, $\typbind$, $\datbind$ or $\exnbind$ may bind
      the same identifier twice; this applies also to value constructors within
      a $\datbind$.
\item In the left side $\tyvarseq\ \tycon$ of any $\typbind$ or $\datbind$,
      $\tyvarseq$ must not contain the same $\tyvar$ twice. Any $\tyvar$
      occurring within the right side must occur in $\tyvarseq$. 
\item For each value binding \pat\ \ml{=} \exp\ within $\REC$,
      $\exp$ must be of the form \fnexp, possibly constrained by one
      or more type expressions. The derived form
      of function-value binding given in Appendix~\ref{derived-forms-app},
      page~\pageref{der-dec}, necessarily obeys this restriction.
%from version 1:
%\item Every non-local exception binding -- that is, not localised by $\LET$
%      or $\LOCAL$ -- must be explicitly constrained by a type containing
%      no type variables.
%not needed, covered by polymorphic references and exceptions:
%\item Any type expression $\ty$ occurring in a non-local 
%      exception binding -- that is,
%      one not localised by $\LET$ or $\LOCAL$ -- must contain no type
%      variables.
\end{itemize}

\include{synmod}
\section{Static Semantics for the Core}
Our\index{20.1} first task in presenting the semantics -- whether for Core or Modules,
static or dynamic -- is to define the objects concerned. In addition
to the class of {\em syntactic} objects, which we have already defined, 
there are classes of so-called {\em semantic} objects used to describe
the meaning of the syntactic objects. Some classes contain {\em simple}
semantic objects; such objects are usually identifiers or names of some
kind. Other classes contain {\em compound} semantic objects, such as
types or environments, which are constructed from component objects.

\subsection{Simple Objects}
All semantic objects in the static semantics of the entire 
language are built from identifiers and two further kinds of simple objects: 
type constructor names and structure names.
Type constructor names are the values taken by type constructors; we shall
usually refer to them briefly as type names, but they are to be clearly
distinguished from type variables and type constructors. 
Structure names play an active role only in
the Modules semantics; they enter the Core semantics only because
they appear in structure environments, which (in turn) are needed in the Core
semantics only to determine the values of long identifiers. The simple object
classes, and the variables ranging over them, are shown in
Figure~\ref{simple-objects}. We have included $\TyVar$ in the table to
make visible the use of $\alpha$ in the semantics to range over $\TyVar$.\index{20.2}

\vspace{-7mm}
\begin{figure}[h]
\vspace{2pt}
\begin{displaymath}
\begin{array}{rclr}
\alpha\ {\rm or}\ \tyvar & \in   & \TyVar	& \mbox{type variables}\\
\t               & \in   & \TyNames	& \mbox{type names}\\
\m		& \in	& \StrNames	& \mbox{structure names}
\end{array}
\end{displaymath}
\caption{Simple Semantic Objects}
\label{simple-objects}
%\vspace{3pt}
\end{figure}

Each\index{20.3} $\alpha \in\TyVar$ possesses a boolean {\sl equality} attribute,
which determines whether or not it {\sl admits equality}, i.e. whether
it is a member of EtyVar (defined on page~\pageref{etyvar-lab}).
%-- in which case we
%also say that it is an {\sl equality} type variable. 
%poly 
Independently hereof, each $\alpha$ possesses a boolean attribute,
the {\sl imperative} attribute, which determines whether it is imperative,
i.e. whether it is a member of $\ImpTyVar$ (defined on page~\pageref{etyvar-lab})
or not.

Each $\t\in\TyNames$ has
an arity $k\geq 0$, and also possesses an equality attribute.
We denote the class of type names with arity $k$ by $\TyNamesk$.

With\index{20.35} each special constant {\scon} we associate a type
name $\scontype(\scon)$ which is either {\INT}, {\REAL} or {\STRING}
as indicated by Section~\ref{cr:speccon}.

\subsection{Compound Objects}
When\index{20.4} $A$ and $B$ are sets $\Fin A$ denotes the set of finite subsets of $A$,
and $\finfun{A}{B}$ denotes the set of {\sl finite maps} (partial functions
with finite domain) from $A$ to $B$.
The domain\index{21.1} and range of a finite map, $f$, are denoted $\Dom f$ and
$\Ran f$.
A finite map will often be written explicitly in the form $\kmap{a}{b},
\ k\geq 0$;
in particular the empty map is $\emptymap$.
We shall use the form $\{x\mapsto e\  ;\  \phi\}$ -- a form of set
comprehension -- to stand for the finite map $f$ whose domain
is the set of values $x$ which satisfy the condition $\phi$, and
whose value on this domain is given by $f(x)=e$.

When $f$ and $g$ are finite maps the map $\plusmap{f}{g}$, called
$f$ {\sl modified} by $g$, is the finite map with domain
$\Dom f \cup \Dom g$ and values
\[(\plusmap{f}{g})(a) = \mbox{if $a\in\Dom g$ then $g(a)$ else $f(a)$.}
\]

The compound objects for the static semantics of the Core Language are
shown in Figure~\ref{compound-objects}.
We take $\cup$ to mean disjoint union over
semantic object classes. We also understand all the defined object
classes to be disjoint.

\begin{figure}[h]
%\vspace{2pt}
\begin{displaymath}
\begin{array}{rcl}
        \tau	&\in	&\Type = \TyVar\cup\RecType\cup\FunType
                                 \cup\ConsType\\
 \longtauk\ {\rm or}\ \tauk
                & \in   & \Type^k\\
 \longalphak\ {\rm or}\ \alphak
                & \in   & \TyVar^k\\
 \varrho        & \in   & \RecType = \finfun{\Lab}{\Type} \\
 \tau\rightarrow\tau'
                & \in   & \FunType = \Type\times\Type \\
                &       & \ConsType = \cup_{k\geq 0}\ConsType^{(k)}\\
        \tauk\t & \in 	& \ConsType^{(k)} = \Type^k\times\TyNamesk  \\
\typefcn\ {\rm or}\ \typefcnk
        	& \in	& \TypeFcn = \cup_{k\geq 0}\TyVar^k\times\Type\\
\tych\ {\rm or}\ \longtych
        	& \in	& \TypeScheme = \cup_{k\geq 0}\TyVar^k\times\Type\\
\S\ {\rm or}\ (\m,\E)
        	& \in	& \Str = \StrNames\times\Env \\
(\theta,\CE)    & \in   & \TyStr = \TypeFcn\times\ConEnv\\
\SE		& \in	& \StrEnv = \finfun{\StrId}{\Str}\\
\TE		& \in	& \TyEnv = \finfun{\TyCon}{\TyStr}\\
\CE             & \in   & \ConEnv = \finfun{\Con}{\TypeScheme}\\
\VE		& \in	& \VarEnv = \finfun{(\Var\cup\Con\cup\Exn)}{\TypeScheme}\\
\EE		& \in	& \ExnEnv = \finfun{\Exn}{\Type}\\
\E\ {\rm or}\ \longE{}
                & \in	& \Env = \StrEnv\times\TyEnv\times\VarEnv
                                         \times\ExnEnv\\
\T              & \in   & \TyNameSets = \Fin(\TyNames)\\
\U              & \in   & \TyVarSet = \Fin(\TyVar)\\
\C\ or\ \T,\U,\E   & \in   & \Context = \TyNameSets\times\TyVarSet\times\Env
\end{array}
\end{displaymath}
\caption{Compound Semantic Objects\index{21.2}}
\label{compound-objects}
%\vspace{3pt}
\end{figure}


Note that $\Lambda$\index{21.3} and $\forall$ bind type variables.  For any semantic object
$A$, $\TyNamesFcn A$ and $\TyVarsFcn A$ denote respectively the set of
type names and the set of type variables occurring free in $A$.
Moreover, $\imptyvars A$ and $\apptyvars A$ denote respectively the set
of imperative type variables and the set of applicative
type variables occurring free in $A$.\index{21.4}
\subsection{Projection, Injection and Modification}
\label{stat-proj}\index{22.1}
{\bf Projection}: We often need to select components of tuples -- for example,
the variable-environment component of a context. In such cases we
rely on  variable names to indicate which component
is selected. For instance ``$\of{\VE}{\E}$'' means ``the variable-environment
component
of $\E$'' and ``$\of{\m}{\S}$'' means ``the structure name of $\S$''.

Moreover, when a tuple contains a finite map we shall ``apply'' the
tuple to an argument, relying on the syntactic class of the argument to
determine the relevant function. For instance $\C(\tycon)$ means
$(\of{\TE}{\C})\tycon$.

A particular case needs mention:  $\C(\con)$ is taken to stand for
$(\of{\VE}{\C})\con$; similarly, $\C(\exn)$ is taken to stand for
$(\of{\VE}{\C})\exn$.
  The type scheme of a value constructor is
held in $\VE$ as well as in $\TE$ (where it will be recorded within
a $\CE$); similarly, the type of an exception constructor is held in
$\VE$ as well as in $\EE$.
Thus the re-binding of a constructor of either kind is given proper
effect by accessing it in $\VE$, rather than in $\TE$ or in $\EE$.

Finally, environments may be applied to long identifiers.
For instance if $\longcon = \strid_1.\cdots.\strid_k.\con$ then
$\E(\longcon)$ means
\[ (\of{\VE}
       {(\of{\SE}
            {\cdots(\of{\SE}
                       {(\of{\SE}{\E})\strid_1}
                   )\strid_2\cdots}
        )\strid_k}
    )\con.
\]

{\bf Injection}: Components may be injected into tuple classes; for example,\linebreak
``$\VE\ \In\ \Env$'' means the environment
$(\emptymap,\emptymap,\VE,\emptymap)$.

{\bf Modification}: The modification of one map $f$ by another map $g$,
written $f+g$, has already been mentioned.  It is commonly used for
environment modification, for example $\E+\E'$.  Often, empty components
will be left implicit in a modification; for example $\E+\VE$ means
$\E+(\emptymap,\emptymap,\VE,\emptymap)$.  For set components, modification
means union, so that $\C+(\T,\VE)$ means
\[ (\ (\of{\T}{\C})\cup\T,\ \of{\U}{\C},\ (\of{\E}{\C})+\VE\ ) \]
Finally, we frequently need to modify a context $\C$ by an environment $\E$
(or a type environment $\TE$ say),
at the same time extending $\of{\T}{\C}$ to include the type names of $\E$
(or of $\TE$ say).
We therefore define $\C\oplus\TE$,\index{22.2} for example, to mean
$\C+(\TyNamesFcn\TE,\TE)$.

\subsection{Types and Type functions}
\label{tyfun-sec}
A type $\tau$ is an {\sl equality type},\index{22.3} or {\sl admits equality}, if it is
of one of the forms
\begin{itemize}
\item $\alpha$, where $\alpha$ admits equality;
\item $\{\lab_1\mapsto\tau_1,\ \cdots,\ \lab_n\mapsto\tau_n\}$,
      where each $\tau_i$ admits equality;
\item $\tauk\t$, where $t$ and all members of $\tauk$ admit equality;
\item $(\tau')\REF$.\index{23.1}
\end{itemize}
\label{tyfcn-lab}
A type function $\theta=\Lambda\alphak.\tau$\index{23.2}
 has arity $k$; it must be
{\sl closed} -- i.e.
$\TyVarFcn(\tau)\subseteq\alphak$ -- and the bound variables must
be distinct. Two type functions are considered equal
if they only differ in their choice of bound variables (alpha-conversion).
In particular, the equality attribute has no significance in a 
bound variable of a type function; for example, $\Lambda\alpha.\alpha\to
\alpha$ and $\Lambda\beta.\beta\to\beta$ are equal type functions
even if $\alpha$ admits equality but $\beta$ does not.
%poly 
Similarly, the imperative attribute has no significance 
in the bound variable of a type function.
If $t$ has arity $k$, then we write $t$ to mean $\Lambda\alphak.\alphak\t$
(eta-conversion); thus $\TyNames\subseteq\TypeFcn$. $\theta=\Lambda\alphak.\tau$
is an {\sl equality} type function, or {\sl admits equality}, if when the
type variables $\alphak$ are chosen to admit equality then $\tau$ also admits
equality.

We write the application of a type function $\theta$ to a vector
$\tauk$ of types as $\tauk\theta$.
If $\theta=\Lambda\alphak.\tau$ we set $\tauk\theta=\tau\{\tauk/\alphak\}$
(beta-conversion). 

We write $\tau\{\thetak/\tk\}$ for the result of substituting type
functions $\thetak$ for type names $\tk$ in $\tau$.
We assume that all beta-conversions
are carried out after substitution, so that for example
\[(\tauk\t)\{\Lambda\alphak.\tau/\t\}=\tau\{\tauk/\alphak\}.\]
%poly 
\label{imp-ty-lab}
A type is {\sl imperative} if all type variables occurring in it are
imperative.
\subsection{Type Schemes}
\label{type-scheme-sec}
A type scheme $\tych=\forall\alphak.\tau$\index{23.3}
 {\sl generalises} a type $\tau'$,
written $\tych \succ\tau'$,
if $\tau'=\tau\{\tauk/\alphak\}$ for some $\tauk$, where each member $\tau_i$
of $\tauk$ admits equality if $\alpha_i$ does,  
%poly 
and $\tau_i$ is imperative if $\alpha_i$ is imperative.
If $\tych'=\forall\beta^{(l)}.\tau'$ then $\tych$ {\sl generalises} $\tych'$,
written $\tych\succ\tych'$, if $\tych\succ\tau'$ and $\beta^{(l)}$ contains
no free type variable of $\tych$.
It can be shown that $\tych\succ\tych'$ iff, for all $\tau''$, whenever
$\tych'\succ\tau''$ then also $\tych\succ\tau''$.

Two type schemes $\tych$ and $\tych'$ are considered equal
if they can be obtained from each other by
renaming and reordering of bound type variables, and deleting type
variables from the prefix which do not occur in the body.
Here, in contrast to the case for type functions, the equality attribute
must be preserved in renaming; for example $\forall\alpha.\alpha\to\alpha$
and $\forall\beta.\beta\to\beta$ are only equal if either both $\alpha$
and $\beta$ admit equality, or neither does.
%poly 
Similarly, the imperative attribute of a bound type variable of a
type scheme {\sl is} significant.
It can be shown that $\tych=\tych'$ iff $\tych\succ\tych'$ and
$\tych'\succ\tych$.

We consider a type $\tau$ to be a type scheme, identifying it with
$\forall().\tau$.

\subsection{Scope of Explicit Type Variables}
\label{scope-sec}

In\index{23.10} the Core language, a type or datatype binding can 
explicitly introduce type variables whose scope is that binding.
In the modules, a description of a value, type, or datatype
may contain explicit type variables whose scope is that
description. However, we\index{23.11} still have to account for the
scope of an explicit type variable occurring in the ``\ml{:} $\ty$'' 
of a typed expression or pattern 
or in the ``\ml{of} $\ty$'' of an exception binding. For the rest
of this section, we consider such occurrences of type variables only.

Every occurrence of a value declaration is said to
{\sl scope} a set of explicit type variables determined as follows.



%Every explicit type variable $\alpha$ is {\sl scoped at} a value binding
%which is determined as follows.

First, an occurrence of $\alpha$ in a value declaration $\valdec$ is said
to be {\sl unguarded} if the occurrence is not part of a smaller value
declaration within $\valbind$.
In this case we say that $\alpha$ {\sl occurs unguarded} in the 
value declaration.

Then we say that $\alpha$ is {\sl scoped at} a particular occurrence
$O$ of $\valdec$ in a program if 
(1) $\alpha$ occurs unguarded in this value declaration, and 
(2) $\alpha$ does not occur unguarded in any larger value declaration
containing the occurrence $O$.\label{scope-def-lab}

Hence, associated with every occurrence of a value declaration there is 
a set $\U$ of the explicit type variables that are scoped at that
occurrence. One may think of each occurrence of $\VAL$ as being implicitly
decorated with such a  set, for instance:

\vspace*{3mm}
\halign{\indent$#$&$#$&$#$\cr
\mbox{$\VAL_{\{\}}$ \ml{x = }}&\mbox{\ml{(}}&
\mbox{\ml{let $\VAL_{\{\mbox{\ml{'a}}\}}$ Id1:'a->'a = fn z=>z in Id1 Id1 end,}}\cr
& &\mbox{\ml{let $\VAL_{\{\mbox{\ml{'a}}\}}$ Id2:'a->'a = fn z=>z in Id2 Id2 end)}}\cr
\noalign{\vspace*{3mm}}
\mbox{$\VAL_{\{\mbox{\ml{'a}}\}}$ \ml{x = }}&\mbox{\ml{(}}&
\mbox{\ml{let $\VAL_{\{\}}$ Id:'a->'a = fn z=>z in Id Id end,}}\cr
& &\mbox{\ml{fn z=> z:'a)}}\cr}

According to the inference rules in Section~\ref{stat-cor-inf-rules}
the first example can be elaborated, but the second cannot since \ml{'a}
is bound at the outer value declaration leaving no possibility of two 
different instantiations of the type of \ml{Id} in the application
\ml{Id Id}.



\subsection{Non-expansive Expressions}
\label{expansive-sec}
In\index{23.4} order to treat polymorphic references and exceptions,
the set Exp of expressions is partitioned into two classes, the {\sl
expansive} and the {\sl non-expansive} expressions. Any variable,
constructor and $\FN$ expression, possibly constrained by one or more
type expressions, is non-expansive; all other expressions are said to
be expansive.  The idea is that the dynamic evaluation of a
non-expansive expression will neither generate an exception nor extend
the domain of the memory, while the evaluation of an expansive
expression might.

\subsection{Closure}
\label{closure-sec}
Let\index{24.2} $\tau$ be a type and $A$ a semantic object. Then $\cl{A}{(\tau)}$,
the {\sl closure} of $\tau$ with respect to $A$, is the type scheme
$\forall\alphak.\tau$, where $\alphak=\TyVarFcn(\tau)\setminus\TyVarFcn A$.
Commonly, $A$ will be a context $\C$.
We abbreviate the {\sl total} closure $\cl{\emptymap}{(\tau)}$ to
$\cl{}{(\tau)}$.
If the range of a variable environment $\VE$ contains only types (rather than
arbitrary type schemes) we set
\[\cl{A}{\VE}=\{\id\mapsto\cl{A}{(\tau)}\ ;\ \VE(\id)=\tau\}\]
with a similar definition for $\cl{A}{\CE}$.

\label{clos-def-lab}
Closing\index{24.3} a variable environment $\VE$ that stems from
the elaboration of a value binding $\valbind$ requires extra
care to ensure type security of references and exceptions and correct
scoping of explicit type variables.
Recall that $\valbind$ is not allowed to bind the
same variable twice. Thus, for each $\var\in\Dom\VE$ 
there is a unique \mbox{\pat\ \ml{=} \exp}
in $\valbind$ which binds $\var$. 
If $\VE(\var)=\tau$, let 
$\cl{\C,\valbind}{\VE(\var)}=\longtych$, where
\[\alphak=\cases{\TyVarFcn\tau\setminus\TyVarFcn\C,&if $\exp$ 
                                                    is non-expansive;\cr
                 \apptyvars\tau\setminus\TyVarFcn\C,&if $\exp$ is expansive.}
\]
%(Explicit type variables necessitate a slight strengthening of the requirements
%which a type variable must satisfy in order to be bound: a type variable 
%occurring unguarded in $\valbind$ can be admitted among $\alphak$ only
%if it is scoped at the particular occurrence of $\valbind$ at which
%the closure operation is performed.)

Notice that the form of $\valbind$ does not affect the binding of
applicative type variables, only the binding of imperative
type variables.

\subsection{Type Structures and Type Environments}
\label{typeenv-wf-sec}
A type\index{24.4} structure $(\theta,\CE)$ is {\sl well-formed} if either
$\CE=\emptymap$, or $\theta$ is a type name $t$.
(The latter case arises, with $\CE\neq\emptymap$, in $\DATATYPE$
declarations.)
All type structures occurring in elaborations are assumed to
be well-formed.

A type structure $(\t,\CE)$ is said to
{\sl respect equality} if, whenever $\t$ admits equality, then
either $\t=\REF$ (see Appendix~\ref{init-stat-bas-app}) or,
for each $\CE(\con)$ of the form 
$\forall\alphak.(\tau\rightarrow\alphak\t)$,
the type function $\Lambda\alphak.\tau$ also admits equality.
(This ensures that the equality
predicate ~{\tt =}~ will be applicable to a constructed value $(\con,v)$ of
type $\tauk\t$ only when it is applicable to the value $v$ itself,
whose type is $\tau\{\tauk/\alphak\}$.)
A type environment $\TE$ {\sl respects equality} if all its type
structures do so.

Let $\TE$ be a type environment, and let $T$ be the set of type names
$\t$ such that $(\t,\CE)$ occurs in $\TE$ for some
$\CE\neq\emptymap$.  Then $\TE$ is said to {\sl maximise equality}
if (a) $\TE$ respects equality, and also (b) if any larger subset of
$T$ were to admit equality (without any change in the equality
attribute of any type names not in $T$) then $\TE$ would cease to
respect equality.


For any $\TE$ of the form
\[\TE=\{\tycon_i\mapsto(t_i,\CE_i)\ ;\ 1\leq i\leq k\},\]
where no $\CE_i$ is the empty map, and for any $\E$ we define
$\Abs(\TE,\E)$ to\index{25.1} be the environment obtained from 
$\E$ and $\TE$ as
follows. First, let $\Abs(\TE)$ be the type environment
$\{\tycon_i\mapsto(t_i,\emptymap)\ ;\ 1\leq i\leq k\}$
in which all constructor
environments $\CE_i$ have been replaced by the empty map. 
Let $t_1',\cdots,t_k'$ be new distinct type names none of which
admit equality. Then $\Abs(\TE,\E)$ is the result of simultaneously
substituting
$t_i'$ for $t_i$, $1\leq i\leq k$,  throughout $\Abs(\TE)+\E$. 
(The effect of the latter substitution is to ensure that the use of 
equality on  an $\ABSTYPE$ is restricted to the $\WITH$ part.)
\label{abs-lab}
\clearpage

\subsection{Inference Rules}
\label{stat-cor-inf-rules}
Each rule\index{26.1} of the semantics allows inferences among sentences of the form
\[A\ts{\it phrase}\ra A'\]
where
$A$ is usually an environment or a context, {\it phrase} is a phrase of
the Core, and $A'$ is a semantic object -- usually a type or an
environment.  It may be pronounced ``{\it phrase} elaborates to $A'$ in
(context or environment) $A$''.  Some rules have extra hypotheses not of
this form; they are called {\sl side conditions}.  

In the presentation of the rules, phrases within single
angle brackets ~$\langle\ \rangle$~ are called {\sl
first options}, and those within double
angle brackets ~$\langle\langle\ \rangle\rangle$~ are called {\sl
second options}.  To reduce the number of rules, we have adopted the
following convention:
\begin{quote} In each instance of a rule, the
first options must be either all present or all absent;
similarly the second options must be either all present or all absent.
\end{quote}

Although not assumed in our definitions, it is intended that every
context $\C=\T,\U,\E$ has the property that $\TyNamesFcn\E\subseteq\T$.
Thus $\T$ may be thought of, loosely, as containing all type names
which ``have been generated''. It is necessary to include $\T$ as a
separate component in a context, since $\TyNamesFcn\E$ may not contain
all the type names which have been generated; one reason is that a
context $\T,\emptyset,\E$ is a projection of the basis
$\B=(\M,\T),\F,\G,\E$ whose other components $\F$ and $\G$
could contain other such names -- recorded in $\T$ but not present in
$\E$.  Of course, remarks about what ``has been generated'' are not
precise in terms of the semantic rules. But the following precise result
may easily be demonstrated:
\begin{quote}
Let S be a sentence
~$\T,\U,\E\ts{\it phrase}\ra A$~ such that
$\TyNamesFcn\E\subseteq\T$,
and let S$'$ be a sentence
~$\T',\U',\E'\ts{\it phrase}'\ra A'$~
occurring in a proof of S; then also
$\TyNamesFcn\E'\subseteq\T'$.
\end{quote}



%                       Atomic Expressions
%
\rulesec{Atomic Expressions\index{26.2}}{\C\vdash\atexp\ra\tau}
%\begin{figure}[h]

\begin{equation}	% special constant
\label{sconexp-rule}
\frac{}
     {\C\ts\scon\ra\scontype(\scon)}\index{26.3}
\end{equation}

\begin{equation}	% value variable
\label{varexp-rule}
\frac{\C(\longvar)\succ\tau}
     {\C\ts\longvar\ra\tau}
\end{equation}

\begin{equation}	% value constructor
\label{conexp-rule}
\frac{\C(\longcon)\succ\tau}
     {\C\ts\longcon\ra\tau}
\end{equation}

\begin{equation}      % exception constant
%\label{exconexp-rule}
\frac{\C(\longexn)=\tau}
     {\C\ts\longexn\ra\tau}
\end{equation}

\begin{equation}	% record expression
%\label{recexp-rule}
\frac{\langle\C\ts\labexps\ra\varrho\rangle}
     {\C\ts\ttlbrace\ \recexp\ \ttrbrace\ra\emptymap\langle +\ \varrho\rangle{\rm\ in\ \Type}}\index{27.0}
\end{equation}

\begin{equation}        % local declaration
\label{let-rule}
\frac{\C\ts\dec\ra\E\qquad\C\oplus\E\ts\exp\ra\tau}
     {\C\ts\letexp\ra\tau}\index{27.1}
\end{equation}

\begin{equation}	% paren expression
%\label{parexp-rule}
\frac{\C\ts\exp\ra\tau}
     {\C\ts\parexp\ra\tau}
\end{equation}
\comments
\begin{description}
\item{(\ref{varexp-rule}),(\ref{conexp-rule})} 
The instantiation of 
type schemes allows different occurrences of a single $\longvar$ 
or $\longcon$ to assume different types.
\item{(\ref{let-rule})} 
The use of $\oplus$, here and elsewhere, ensures that
type names generated by the first sub-phrase are different from 
type names generated by the second sub-phrase.
\end{description}

\rulesec{Expression Rows}{\C\ts\labexps\ra\varrho}
\begin{equation}	% expression rows
%\label{labexps-rule}
\frac{\C\ts\exp\ra\tau\qquad\langle\C\ts\labexps\ra\varrho\rangle}
     {\C\ts\longlabexps\ra\{\lab\mapsto\tau\}\langle +\ \varrho\rangle}\index{27.2}
\end{equation}
%                        Expressions
%
\rulesec{Expressions}{\C\ts\exp\ra\tau}
%\vspace{6pt}
%\fbox{$\C\ts\exp\ra\tau$}
\begin{equation}	% atomic
\label{atexp-rule}
\frac{\C\ts\atexp\ra\tau}
     {\C\ts\atexp\ra\tau}\index{27.3}
\end{equation}

\begin{equation}	% application
%\label{app-rule}
\frac{\C\ts\exp\ra\tau'\rightarrow\tau\qquad\C\ts\atexp\ra\tau'}
     {\C\ts\appexp\ra\tau}
\end{equation}

\begin{equation}	% typed
\label{typedexp-rule}
\frac{\C\ts\exp\ra\tau\qquad\C\ts\ty\ra\tau}
     {\C\ts\typedexp\ra\tau}
\end{equation}

\begin{equation}        % handle exception
%\label{handlexp-rule}
\frac{\C\ts\exp\ra\tau\qquad\C\ts\match\ra\EXCN\rightarrow\tau}
     {\C\ts\handlexp\ra\tau}
\end{equation}

\begin{equation}        % raise exception
\label{raiseexp-rule}
\frac{\C\ts\exp\ra\EXCN}
     {\C\ts\raisexp\ra\tau}
\end{equation}

\begin{equation}        % function
%\label{fnexp-rule}
\frac{\C\ts\match\ra\tau}
     {\C\ts\fnexp\ra\tau}
\end{equation}
\comments
\begin{description}
\item{(\ref{atexp-rule})}
The relational symbol $\ts$ is overloaded for all syntactic classes (here
atomic expressions and expressions).
\item{(\ref{typedexp-rule})}
Here $\tau$ is determined by $\C$ and $\ty$. Notice that type variables
in $\ty$ cannot be instantiated in obtaining $\tau$; thus the expression
\verb+1:'a+ will not elaborate successfully, nor will the expression
\verb+(fn x=>x):'a->'b+.
The effect of type variables in an explicitly typed expression is
to indicate exactly the degree of polymorphism present in the expression.\index{27.4}
\item{(\ref{raiseexp-rule})}
Note that $\tau$ does not occur in the premise; thus a $\RAISE$
expression has ``arbitrary'' type.
\end{description}
%                        Matches 
%
\rulesec{Matches}{\C\ts\match\ra\tau}
\begin{equation}	% match
%\label{match-rule}
\frac{\C\ts\mrule\ra\tau\qquad\langle\C\ts\match\ra\tau\rangle}
     {\C\ts\longmatch\ra\tau}\index{28.1}
\end{equation}
\rulesec{Match Rules}{\C\ts\mrule\ra\tau}
\begin{equation}	% mrule
%\label{mrule-rule}
\frac{\C\ts\pat\ra(\VE,\tau)\qquad\C+\VE\ts\exp\ra\tau'}
     {\C\ts\longmrule\ \ra\tau\rightarrow\tau'}
\end{equation}
\comment  This rule allows new free type variables to enter
the context. These new type variables will be chosen, in effect, during
the elaboration of $\pat$ (i.e., in the inference of the first
hypothesis). In particular, their choice may have to be made to
agree with type variables present in any explicit type expression
occurring within $\exp$ (see rule~\ref{typedexp-rule}).

%
%                        Declarations
%
\rulesec{Declarations}{\C\ts\dec\ra\E}
%poly
\begin{equation}	% value declaration
\label{valdec-rule}
\frac{\plusmap{\C}{\U}\ts\valbind\ra\VE\qquad 
      \VE'=\cl{\C,\valbind}{\VE}\qquad
      \U\cap\TyVarFcn\VE'=\emptyset}
     {\C\ts\valdecS\ra\VE'\ \In\ \Env}\index{28.2}
\end{equation}
%from version 1, without polymorphic refs:
%\begin{equation}	% value declaration
%\label{valdec-rule}
%\frac{\C\ts\valbind\ra\VE}
%     {\C\ts\valdec\ra\cl{\C}{\VE}\ \In\ \Env}
%\end{equation}

\begin{equation}	% type declaration
%\label{typedec-rule}
\frac{\C\ts\typbind\ra\TE}
     {\C\ts\typedec\ra\TE\ \In\ \Env}
\end{equation}

\begin{equation}	% datatype declaration
\label{datatypedec-rule}
\frac{\begin{array}{c}
\C\oplus\TE\ts\datbind\ra\VE,\TE\qquad
      \forall(\t,\CE)\in\Ran\TE,\ \t\notin(\of{\T}{\C}) \\
     \mbox{$\TE$ maximises equality}
     \end{array}
     }
     {\C\ts\datatypedec\ra(\VE,\TE)\ \In\ \Env}
\end{equation}

\begin{equation}	% abstype declaration
\label{abstypedec-rule}
\frac{\begin{array}{rl}
      \C\oplus\TE\ts\datbind\ra\VE,\TE\qquad &
      \forall(\t,\CE)\in\Ran\TE,\ \t\notin(\of{\T}{\C})\\
      \C\oplus(\VE,\TE)\ts\dec\ra\E\qquad & 
     \mbox{$\TE$ maximises equality}
      \end{array}
     }
     {\C\ts\abstypedec\ra\Abs(\TE,\E)}
\end{equation}

\begin{equation}	% exception declaration
\label{exceptiondec-rule}
\frac{\C\ts\exnbind\ra\EE\quad\VE=\EE }
     {\C\ts\exceptiondec\ra(\VE,\EE)\ \In\ \Env }
\end{equation}

\begin{equation}	% local declaration
%\label{localdec-rule}
\frac{\C\ts\dec_1\ra\E_1\qquad\C\oplus\E_1\ts\dec_2\ra\E_2}
     {\C\ts\localdec\ra\E_2}\index{28.3}
\end{equation}

\begin{equation}                % open declaration
%\label{open-dec-rule}
\frac{ \C(\longstrid_1)=(\m_1,\E_1)
            \quad\cdots\quad
       \C(\longstrid_n)=(\m_n,\E_n) }
     { \C\ts\openstrdec\ra \E_1 + \cdots + \E_n }
\end{equation}

\begin{equation}	% empty declaration
%\label{emptydec-rule}
\frac{}
     {\C\ts\emptydec\ra\emptymap\ \In\ \Env}
\end{equation}

\begin{equation}	% sequential declaration
%\label{seqdec-rule}
\frac{\C\ts\dec_1\ra\E_1\qquad\C\oplus\E_1\ts\dec_2\ra\E_2}
     {\C\ts\seqdec\ra\plusmap{E_1}{E_2}}
\end{equation}
\comments
\begin{description}
\item{(\ref{valdec-rule})}
Here $\VE$ will contain types rather than general
type schemes. The closure of $\VE$ is exactly what allows variables to
be used polymorphically, via rule~\ref{varexp-rule}.

Moreover, $\U$ is the set of explicit type variables scoped at this particular
occurrence of $\valdec$, cf. Section~\ref{scope-sec}, 
page~\pageref{scope-def-lab}. The side-condition on $\U$
ensures that these explicit type variables are bound by the closure operation.
On the other hand, no {\sl other} explicit type variable occurring
free in $\VE$ will become bound, since it must be in $\of{\U}{\C}$, and
is therefore excluded from closure by the definition of the closure operation
(Section~\ref{closure-sec}, page~\pageref{clos-def-lab})
since $\of{\U}{\C}\subseteq\TyVarFcn\C$.
\item{(\ref{datatypedec-rule}),(\ref{abstypedec-rule})}
The side conditions
express that the elaboration of each datatype binding
generates new type names and that as many of these new names
as possible admit equality.  Adding $\TE$ to the context on the left
of the $\ts$ captures the recursive nature of the binding.
%The side condition is
%the formal way of expressing that the elaboration of each datatype binding
%generates new type names.  Adding $\TE$ to the context on the left
%of the $\ts$ captures the recursive nature of the binding. Recall that $\TE$
%is assumed well-formed (as defined in Section~\ref{typeenv-wf-sec}). If
%$\TyNamesFcn(\of{\E}{\C})\subseteq\of{\T}{\C}$ and the side condition is
%satisfied then $\C\oplus\TE$ is well-formed.
\item{(\ref{abstypedec-rule})}
The $\Abs$ operation was defined in Section~\ref{typeenv-wf-sec}, page~\pageref{abs-lab}.
\item{(\ref{exceptiondec-rule})}
No closure operation is used here, since $\EE$ maps exception 
names to types rather than  to general type schemes.
Note that $\EE$ is also recorded in the {\VarEnv} component of
the resulting environment (see Section~\ref{stat-proj}, page~\pageref{stat-proj}).\index{29.0}
\end{description}

%                        Bindings
%
\rulesec{Value Bindings}{\C\ts\valbind\ra\VE}
%\vspace{6pt}
\begin{equation}	% value binding
\label{valbind-rule}
\frac{\C\ts\pat\ra(\VE,\tau)\qquad\C\ts\exp\ra\tau\qquad
      \langle\C\ts\valbind\ra\VE'\rangle }
     {\C\ts\longvalbind\ra\VE\ \langle +\ \VE'\rangle}\index{29.1}
\end{equation}

\begin{equation}	% recursive value binding
\label{recvalbind-rule}
\frac{\C+\VE\ts\valbind\ra\VE}
     {\C\ts\recvalbind\ra\VE}
\end{equation}
\comments
\begin{description}
\item{(\ref{valbind-rule})}
When the option is present we have $\Dom\VE\cap
\Dom\VE' = \emptyset$ by the syntactic restrictions.\index{29.2}
\item{(\ref{recvalbind-rule})}
Modifying $\C$ by $\VE$ on the left captures the 
recursive nature of the binding. From rule~\ref{valbind-rule} we see that any
type scheme occurring in $\VE$ will have to be a type. Thus each use of a
recursive function in its own body must be ascribed the same type.
\end{description}

\rulesec{Type Bindings}{\C\ts\typbind\ra\TE}
%\fbox{$\C\ts\typbind\ra\TE$}
\begin{equation}	% type binding
%\label{typbind-rule}
\frac{\tyvarseq=\alphak\qquad\C\ts\ty\ra\tau\qquad
      \langle\C\ts\typbind\ra\TE\rangle}
     {\begin{array}{c}
      \C\ts\longtypbind\ra\\
      \qquad\qquad\qquad
      \{\tycon\mapsto(\typefcnk,\emptymap)\}\ \langle +\ \TE\rangle
      \end{array}
     }\index{29.3}
\end{equation}
\comment The syntactic restrictions ensure that the type function
$\typefcnk$ satisfies the well-formedness constraints of 
Section~\ref{tyfun-sec} and they ensure $tycon\notin\Dom\TE$.

\rulesec{Data Type Bindings}{\C\ts\datbind\ra\VE,\TE}
%\fbox{$\C\ts\datbind\ra\VE,\TE$}
\begin{equation}	% datatype binding
%\label{datbind-rule}
\frac{\begin{array}{c}
        \tyvarseq=\alphak\qquad\C,\alphakt\ts\constrs\ra\CE\\
        \langle\C\ts\datbind\ra\VE,\TE\qquad
        \forall(\t',\CE)\in\Ran\TE, \t\neq\t'\rangle
      \end{array}
     }
     {\begin{array}{c}
        \C\ts\longdatbind\ra\\
        \qquad\qquad\qquad\cl{}{\CE}\langle +\ \VE\rangle,\
        \{\tycon\mapsto(\t,\cl{}{\CE})\}\ \langle +\ \TE\rangle
      \end{array}
     }\index{30.1}
\end{equation}
\comment The syntactic restrictions ensure $\Dom\VE\cap\Dom\CE = \emptyset$
and $\tycon\notin\Dom\TE$.

\rulesec{Constructor Bindings}{\C,\tau\ts\constrs\ra\CE}
%\fbox{$\C,\tau\ts\constrs\ra\CE$}
\begin{equation}	% data constructors
%\label{constrs-rule}
\frac{\langle\C\ts\ty\ra\tau'\rangle\qquad
      \langle\langle\C,\tau\ts\constrs\ra\CE\rangle\rangle }
     {\begin{array}{c}
      \C,\tau\ts\longerconstrs\ra\\
      \qquad\qquad\qquad\{\con\mapsto\tau\}\
     \langle +\ \{\con\mapsto\tau'\to\tau\}\ \rangle\
      \langle\langle +\ \CE\rangle\rangle
      \end{array}
     }\index{30.2}
\end{equation}
\comment By the syntactic restrictions $\con\notin\Dom\CE$.

\rulesec{Exception Bindings}{\C\ts\exnbind\ra\EE}
%poly with polymorphic exceptions:
\begin{equation}	% exception binding
\label{exnbind1-rule}
\frac{\langle\C\ts\ty\ra\tau\quad\mbox{$\tau$ is imperative}\rangle\qquad
      \langle\langle\C\ts\exnbind\ra\EE\rangle\rangle }
     {\begin{array}{c}
      \C\ts\longexnbinda\ra\\
      \qquad\qquad\qquad\{\exn\mapsto\EXCN\}\
      \langle +\ \{\exn\mapsto\tau\rightarrow\EXCN\}\ \rangle\
      \langle\langle +\ \EE\rangle\rangle
      \end{array}
     }\index{30.3}
\end{equation}

%with mono typed exceptions:
%\begin{equation}	% exception binding
%\label{exnbind1-rule}
%\frac{\langle\C\ts\ty\ra\tau\quad\TyVarFcn(\tau)=\emptyset\rangle\qquad
%      \langle\langle\C\ts\exnbind\ra\EE\rangle\rangle }
%     {\begin{array}{c}
%      \C\ts\longexnbinda\ra\\
%      \qquad\qquad\qquad\{\exn\mapsto\EXCN\}\
%      \langle +\ \{\exn\mapsto\tau\rightarrow\EXCN\}\ \rangle\
%      \langle\langle +\ \EE\rangle\rangle
%      \end{array}
%     }
%\end{equation}

\vspace*{4mm}
\begin{equation}	% exception binding
\label{exnbind2-rule}
\frac{\C(\longexn)=\tau\qquad
      \langle\C\ts\exnbind\ra\EE\rangle }
      {\C\ts\longexnbindb\ra\{\exn\mapsto\tau\}\ \langle +\ \EE\rangle}
%     {\begin{array}{c}
%      \C\ts\longexnbindb\ra\\
%      \qquad\qquad\qquad\{\exn\mapsto\tau\}\
%      \langle +\ \EE\rangle
%      \end{array}
%     }
\end{equation}
\comments
\begin{description}
\item{(\ref{exnbind1-rule})} Notice that $\tau$ must not contain
any applicative type variables.\index{30.35}
%with monotyped exceptions:
%\item{(\ref{exnbind1-rule})} Notice that $\tau$ must be a monotype
%(see also restriction~\ref{monotypes-res} in 
%Section~\ref{further-restrictions-sec}).
\item{(\ref{exnbind1-rule}),(\ref{exnbind2-rule})}
There is a unique $\EE$, for each $\C$ and $\exnbind$,
%No matter which of the options are present, given $\C$ and $\exnbind$ there
%is at most one $\EE$ 
such that $\C\ts\exnbind\ra\EE$.
\end{description}

%\caption{Rules for Bindings}
%\end{figure}

%                        Atomic Patterns
%
\rulesec{Atomic Patterns}{\C\ts\atpat\ra(\VE,\tau)}
%\vspace{6pt}
%\fbox{$\C\ts\atpat\ra(\VE,\tau)$}
\begin{equation}	% wildcard pattern
%\label{wildcard-rule}
\frac{}
     {\C\ts\wildpat\ra (\emptymap,\tau)}\index{30.4}
\end{equation}

\begin{equation}	% special constant in pattern
\frac{}
     {\C\ts\scon\ra (\emptymap,\scontype(\scon))}\index{30.5}
\end{equation}

\begin{equation}	% variable pattern
\label{varpat-rule}
\frac{}
     {\C\ts\var\ra (\{\var\mapsto\tau\},\tau) }
\end{equation}

\begin{equation}	% constant pattern
%\label{constpat-rule}
\frac{\C(\longcon)\succ\tauk\t }
     {\C\ts\longcon\ra (\emptymap,\tauk\t)}
\end{equation}

\begin{equation}       % exception constant
%\label{exconapat-rule}
\frac{\C(\longexn)=\EXCN}
     {\C\ts\longexn\ra (\emptymap,\EXCN)}
\end{equation}



\begin{equation}	% record pattern
%\label{recpat-rule}
\frac{\langle\C\ts\labpats\ra(\VE,\varrho)\rangle}
     {\C\ts\lttbrace\ \recpat\ \rttbrace\ra(\ \emptymap\langle +\ \VE\rangle,\ \emptymap
      \langle +\ \varrho\rangle\ \In\ \Type\ ) }\index{31.1}
\end{equation}

\begin{equation}	% parenthesised pattern
%\label{parpat-rule}
\frac{\C\ts\pat\ra(\VE,\tau)}
     {\C\ts\parpat\ra(\VE,\tau)}
\end{equation}
\comments
\begin{description}
\item{(\ref{varpat-rule})} 
Note that $\var$ can assume a type, not a general type scheme.
\end{description}

\rulesec{Pattern Rows}{\C\ts\labpats\ra(\VE,\varrho)}
%\fbox{$\C\ts\labpats\ra(\VE,\varrho)$}
\begin{equation}	% wildcard record
%\label{wildrec-rule}
\frac{}
     {\C\ts\wildrec\ra(\emptymap,\varrho)}\index{31.2}
\end{equation}

\begin{equation}	% record component
\label{longlab-rule}
\frac{\C\ts\pat\ra(\VE,\tau)\qquad
      \langle\C\ts\labpats\ra(\VE',\varrho)\qquad\lab\notin\Dom\varrho\rangle
}
     {\C\ts\longlabpats\ra
     (\VE\langle +\ \VE'\rangle,\
      \{\lab\mapsto\tau\}\langle +\ \varrho\rangle) }
\end{equation}
\comment 
\begin{description}
\item{(\ref{longlab-rule})} 
 By the syntactic restrictions, $\Dom\VE\cap\Dom\VE' = \emptyset$.
\end{description}
%                        Patterns
%
%\begin{figure}[h]
%\label{pat-rules}
\rulesec{Patterns}{\C\ts\pat\ra(\VE,\tau)}
\begin{equation}	% atomic pattern
%\label{atpat-rule}
\frac{\C\ts\atpat\ra (\VE,\tau)}
     {\C\ts\atpat\ra (\VE,\tau)}\index{31.3}
\end{equation}

\begin{equation}	% construction pattern
%\label{conpat-rule}
\frac{\C(\longcon)\succ\tau'\to\tau\qquad\C\ts\atpat\ra(\VE,\tau')}
     {\C\ts\conpat\ra (\VE,\tau)}
\end{equation}

\begin{equation}       %  exception construction pattern
%\label{exconpat-rule}
\frac{\C(\longexn)=\tau\rightarrow\EXCN\qquad
      \C\ts\atpat\ra(\VE,\tau)}
     {\C\ts\exconpat\ra(\VE,\EXCN)}
\end{equation}

\begin{equation}	% typed pattern
%\label{typedpat-rule}
\frac{\C\ts\pat\ra(\VE,\tau)\qquad\C\ts\ty\ra\tau}
     {\C\ts\typedpat\ra (\VE,\tau)}
\end{equation}

\begin{equation}	% layered pattern
\label{layeredpat-rule}
\frac{\begin{array}{c}
      \C\ts\var\ra(\VE,\tau)\qquad\langle\C\ts\ty\ra\tau\rangle\\
      \C\ts\pat\ra(\VE',\tau)
%\qquad\VE\sim\VE'
      \end{array}
     }
     {\C\ts\layeredpat\ra(\plusmap{\VE}{\VE'},\tau)}
\end{equation}
\comments
\begin{description}
\item{(\ref{layeredpat-rule})}
By the syntactic restrictions, $\Dom\VE\cap\Dom\VE' = \emptyset$.
\end{description}


%                        Type Expressions
\rulesec{Type Expressions}{\C\ts\ty\ra\tau}
\begin{equation}	% atype variable
%\label{tyvar-rule}
\frac{\tyvar=\alpha}
     {\C\ts\tyvar\ra\alpha}\index{32.1}
\end{equation}

\begin{equation}	% record type
%\label{rectype-rule}
\frac{\langle\C\ts\labtys\ra\varrho\rangle}
     {\C\ts\lttbrace\ \rectype\ \rttbrace\ra\emptymap\langle +\ \varrho\rangle\ \In\ \Type}
\end{equation}

\begin{equation}	% constructed type
\label{constype-rule}
\frac{\begin{array}{c}
      \tyseq=\ty_1\cdots\ty_k\qquad\C\ts\ty_i\ra\tau_i\ (1\leq i\leq k)\\
      \C(\longtycon)=(\theta,\CE)
      \end{array}
     }
     {\C\ts\constype\ra\tauk\theta}
\end{equation}

\begin{equation}	% function type
%\label{funtype-rule}
\frac{\C\ts\ty\ra\tau\qquad\C\ts\ty'\ra\tau'}
     {\C\ts\funtype\ra\tau\to\tau'}
\end{equation}

\begin{equation}	% parenthesised type
%\label{partype-rule}
\frac{\C\ts\ty\ra\tau}
     {\C\ts\partype\ra\tau}
\end{equation}
\comments
\begin{tabbing}
(\ref{constype-rule}) \= Recall that for $\tauk\theta$ to be defined, $\theta$
must have arity $k$.
\end{tabbing}

\rulesec{Type-expression Rows}{\C\ts\labtys\ra\varrho}
%\fbox{$\C\ts\labtys\ra\varrho$}
\begin{equation}	% record type components
%\label{longlabtys-rule}
\frac{\C\ts\ty\ra\tau\qquad\langle\C\ts\labtys\ra\varrho\rangle}
     {\C\ts\longlabtys\ra\{\lab\mapsto\tau\}\langle +\ \varrho\rangle}\index{32.15}
\end{equation}
\comment The syntactic constraints ensure $\lab\notin\Dom\varrho$.
%\caption{Rules for Types}
%\end{figure}

\subsection{Further Restrictions}
\label{further-restrictions-sec}
There\index{32.2} are a few restrictions on programs which should be enforced by a
compiler, but are better expressed apart from the preceding
Inference Rules.  They are:
\begin{enumerate}
%poly
%The following restriction has been removed from Version 1 because
%of polymorphic references and exceptions
%\item \label{monotypes-res}
%The reference value constructor ~{\tt ref}~ may occur in patterns
%with polymorphic type, but in an expression it must always elaborate to
%a monotype, i.e. a type containing no type variables.  
%Moreover, exception constructors can have monotypes only.
%These restrictions
%will be relaxed in future Versions, but some restrictions will always be
%necessary to ensure soundness of the type discipline.
\item For each occurrence of a record pattern containing a record wildcard,
i.e. of the form
%\begin{quote}
\verb+{+${\it lab}_1$\ml{=}$\pat_1$\ml{,}$\cdots$\ml{,}${\it lab}_m$\ml{=}$\pat_m$\ml{,}\verb+...}+
%\end{quote}
the program context must determine uniquely the domain
$\{{\it lab}_1,\cdots,{\it lab}_n\}$
of its record type, where $m\leq n$; thus, the context must
determine the labels $\{{\it lab}_{m+1},\cdots,{\it lab}_n\}$ of the fields
to be matched by the wildcard. For this purpose, an explicit type
constraint may be needed.  This restriction is necessary to ensure the
existence of principal type schemes.
\item In a match of the form ${\it pat}_1$ \ml{=>} $\exp_1$ \ml{|}$\;\cdots\;$
\ml{|} ${\tt pat}_n$ \ml{=>} $\exp_n$ 
the pattern sequence $\pat_1,\ldots,\pat_n$ should be {\sl irredundant};
that is, each $\pat_j$ must match some value
(of the right type) which is not matched by $\pat_i$ for any $i<j$.
In the context {\fnexp}, the $\match$ must also be {\sl exhaustive}; that is,
every value (of the right type) must be matched by some $\pat_i$.
The compiler must give warning on violation of these restrictions, 
but should still compile the match. 
The restrictions are inherited by derived forms; in particular,
this means that in the function binding\index{33.1}
 $\var\ \atpat_1\ \cdots\ \atpat_n\langle : \ty\rangle$\ \ml{=}\ $\exp$
(consisting of one clause only), each separate $\atpat_i$ should be
exhaustive by itself.
%must be {\sl irredundant} and {\sl exhaustive}.  That is, each ${\it pat}_j$
%must match {\sl some} value (of the right type) which is not matched by
%${\it pat}_i$ for any $i <j$, and {\sl every} value (of the right type) must be
%matched by some ${\it pat}_i$. The compiler must give a warning on violation
%of this restriction, but should still compile the match.
\end{enumerate}


\subsection{Principal Environments}
\label{principal-env-sec}
The\index{33.15} notion of {\sl enrichment}, $\E\succ\E'$, between environments
$\E=(\SE,\TE,\VE,\EE)$ and $\E'=(\SE',\TE',\VE',\EE')$ is defined
in Section~\ref{enrichment-sec}. For the present section,  $\E\succ\E'$
may be taken to mean $\SE=\SE'=\emptymap$, $\TE=\TE'$,
$\EE=\EE'$, $\Dom\VE=\Dom\VE'$ and, for each $\id\in\Dom\VE$,
$\VE(\id)\succ\VE'(\id)$.

Let\index{33.2} $\C$ be a context, and suppose that $\C\ts\dec\ra\E$
according to the preceding Inference Rules. Then $E$ is {\em principal}
(for $\dec$ in the context $\C$) if, for all $\E'$ for
which $\C\ts\dec\ra\E'$, we have $\E\succ\E'$. We claim that if
$\dec$ elaborates to any environment in $\C$ then it elaborates to
a principal environment in $\C$. Strictly, we must allow for the
possibility that type names and imperative type variables
which do not occur in $\C$ are chosen
differently for $\E$ and $\E'$. 
The stated claim is therefore made up to such variation.



























\section{Static Semantics for Modules}
\label{statmod-sec}
\subsection{Semantic Objects}
\label{statmod-sem-obj-sec}
The\index{34.1} simple 
objects for Modules static semantics are exactly as for the Core.
The compound objects are those for the Core,
augmented by those in Figure~\ref{module-objects}.


\begin{figure}[h]
%\vspace{2pt}
\begin{displaymath}
\begin{array}{rcl}
\M		& \in	& \StrNameSets = \Fin(\StrNames)\\
\N\ {\rm or}\ (\M,\T)
                & \in	& \NameSets = \StrNameSets\times\TyNameSets\\
\sig\ {\rm or}\ \longsig{}
        	& \in	& \Sig =  \NameSets\times\Str \\
\funsig\ {\rm or}\ \longfunsig{}
         	& \in	& \FunSig = \NameSets\times
                                         (\Str\times\Sig)\\
\G		& \in	& \SigEnv	 =	 \finfun{\SigId}{\Sig} \\
\F		& \in	& \FunEnv	 =	 \finfun{\FunId}{\FunSig} \\
\B\ {\rm or}\ \N,\F,\G,\E
        	& \in	& \Basis = \NameSets\times
                                              \FunEnv\times\SigEnv\times\Env\\
\end{array}
\end{displaymath}
\caption{Further Compound Semantic Objects}
\label{module-objects}
%\vspace{3pt}
\end{figure}
%
The prefix $(\N)$, in signatures and functor signatures, binds both type names
and structure names. We shall always consider a set $\N$ of names as
partitioned into a pair $(\M,\T)$ of sets of the two kinds of name.

It is sometimes convenient to work with an arbitrary semantic object $A$, or
assembly $A$ of such objects.
As with the function $\TyNamesFcn$,
$\StrNamesFcn(A)$ and $\NamesFcn(A)$ denote respectively the set of structure names
and the set of names occurring free in $A$.

%We shall often need to change bound names in semantic objects.
%For example, we sometimes require that
%$\N\cap\N'=\emptyset$ in a functor signature.  More generally,
%for arbitrary $A$
%it is sometimes convenient to assume that {\sl all}
%nameset prefixes $\N$ occurring in $A$ are disjoint.  In that
%case we say that we are {\sl disjoining bound names} in $A$.

Certain operations require a change of bound names in semantic objects;
see for example Section~\ref{realisation-sec}. When bound type names are
changed, we demand that all of their attributes (i.e. imperative, equality
and arity) are preserved.\index{34.2}

For any structure $\S=\longS{}$ we call $m$ the {\sl structure name} or
{\sl name} of $\S$; also, the {\sl proper substructures} of $\S$ are
the members of $\Ran\SE$ and their proper substructures.  The 
{\sl substructures} of
$\S$ are $\S$ itself and its proper substructures.  The structures
{\sl occurring in}
an object or assembly $A$ are the structures and
substructures from which it is built.

The operations of projection, injection and modification are as for the
Core. Moreover, we define $\of{\C}{\B}$ to be the context
$(\of{\T}{\B},\emptyset,\of{\E}{\B})$, i.e.~with an empty set of
explicit type variables.
Also,
we frequently need to modify a basis $\B$ by an environment $\E$
(or a structure environment $\SE$ say),
at the same time extending $\of{\N}{\B}$ to include the type names and
structure names of $\E$ (or of $\SE$ say).
We therefore define $\B\oplus\SE$, for example, to mean
$\B+(\NamesFcn\SE,\SE)$.\index{34.3}

\subsection{Consistency}
\label{consistency-sec}
A\index{35.1} set of type structures is said to be {\sl consistent} if, for all
$(\theta_1,\CE_1)$ and $(\theta_2,\CE_2)$ in the set, if $\theta_1 = \theta_2$
then
\[\CE_1=\emptymap\ {\rm or}\ \CE_2=\emptymap\ {\rm or}\ \Dom\CE_1=\Dom\CE_2\]
A semantic object $A$ or assembly $A$ of objects is said to be
{\sl consistent} if (after changing bound names to make all nameset prefixes
in $A$ disjoint) for all $\S_1$ and
$\S_2$ occurring in $A$ and for every $\longstrid$ and every $\longtycon$
\begin{enumerate}
\item If $\of{\m}{\S_1}=\of{\m}{\S_2}$, and both
      $\S_1(\longstrid)$ and $\S_2(\longstrid)$ exist, then
      \[ \of{\m}{\S_1(\longstrid)}\ =\ \of{\m}{\S_2(\longstrid)}\]
\item If $\of{\m}{\S_1}=\of{\m}{\S_2}$, and both
      $\S_1(\longtycon)$ and $\S_2(\longtycon)$ exist, then
      \[ \of{\theta}{\S_1(\longtycon)}\ =\ \of{\theta}{\S_2(\longtycon)}\]
\item The set of all type structures in $A$ is consistent
\end{enumerate}

As an example, a functor signature $\longfunsig{}$ is
consistent if, assuming first that $\N\cap\N'=\emptyset$,
the assembly $A=\{\S,\S'\}$ is consistent.

We may loosely say that two structures $\S_1$ and $\S_2$ are consistent if
$\{\S_1,\S_2\}$ is consistent, but must remember that this is stronger than
the assertion that $\S_1$ is consistent and $\S_2$ is consistent.

Note that if $A$ is a consistent assembly and $A'\subset A$ then $A'$ is
also a consistent assembly.

\subsection{Well-formedness}
A signature\index{35.2} $\longsig{}$ is {\sl well-formed} 
if $\N\subseteq\NamesFcn\S$,
and also, whenever $(\m,\E)$ is a
substructure of $\S$ and $\m\notin\N$, then $\N\cap(\NamesFcn\E)=\emptyset$.
A functor signature $\longfunsig{}$ is {\sl well-formed} if
$\longsig{}$ and  $(\N')\S'$ are well-formed, and also, whenever
$(\m',\E')$ is a substructure of $\S'$ and $\m'\notin\N\cup\N'$,
then $(\N\cup\N')\cap(\NamesFcn\E')=\emptyset$.

An object or assembly $A$ is {\sl well-formed} if every type environment,
signature and functor signature occurring in $A$ is well-formed.

\subsection{Cycle-freedom}
An\index{35.3} object or assembly $A$ is {\sl cycle-free} if it contains no
cycle of structure names; that is, there is no sequence
\[\m_0,\cdots,\m_{k-1},\m_k=m_0\ \ (k>0)\]
of structure names such that, for each $i\ (0\leq i<k)$ some structure
with name $m_i$ occurring in $A$ has a proper substructure with name
$m_{i+1}$.

\subsection{Admissibility}
\label{admis-sec}
An\index{36.1} object or assembly $A$ is {\sl admissible} if it is
consistent, well-formed and cycle-free.  Henceforth it is assumed that
all objects mentioned are admissible.  We also require that
\begin{enumerate}
\item In every sentence $A\ts\phrase\ra A'$  inferred by the rules
given in Section~\ref{statmod-rules-sec}, the assembly $\{A,A'\}$ is
admissible.  
\item In the special case of a sentence $\B\ts\sigexp\ra\S$,
we further require that the assembly consisting of all semantic
objects occurring in the entire inference of this sentence be
admissible. This  is important for the definition of principal
signatures in Section~\ref{prinsig-sec}.
\end{enumerate}
In our semantic definition we have not undertaken to
indicate how admissibility should be checked in an implementation.
 
\subsection{Type Realisation}
%
A {\sl type realisation}\index{36.2} is a map
$\tyrea:\TyNames\to\TypeFcn$
such that
$\t$ and $\tyrea(\t)$ have the same arity, and
if $t$ admits equality then so does $\tyrea(\t)$.

The {\sl support} $\Supp\tyrea$ of a type realisation $\tyrea$ is the set of
type names $\t$ for which $\tyrea(\t)\ne\t$.
%
\subsection{Realisation}
\label{realisation-sec}
A {\sl realisation}\index{36.3} is a function $\rea$ of names,
partitioned into a type realisation $\tyrea:\TyNames\to\TypeFcn$
and a function $\strrea : \StrNames\to\StrNames$.
The {\sl support} $\Supp\rea$
of a realisation $\rea$ is the set of
names $\n$ for which $\rea(\n)\ne\n$.
The {\sl yield} $\Yield\rea$ of a realisation $\rea$ is the set of
names which occur in some $\rea(\n)$ for which $\n\in\Supp\rea$.

Realisations $\rea$ are extended to apply to all semantic objects; their
effect is to
replace each name $n$ by $\rea(\n)$.  In applying $\rea$ to an object with
bound names, such as a signature $\longsig{}$, first bound names must be
changed so that, for each binding prefix $(\N)$,
\[\N\cap(\Supp\rea\cup\Yield\rea)=\emptyset\ .\]
%
\subsection{Type Explication}
\label{type-explication-sec}
A\index{36.35} signature $(\N)\S$ is {\sl type-explicit\/} if,
whenever $\t\in\N$ and occurs free in $\S$, then some substructure of
$\S$ contains a type environment $\TE$ such that
$\TE(\tycon)=(\t,\CE)$ for some $\tycon$ and some $\CE$.  
%
\subsection{Signature Instantiation}
A\index{36.4} structure $\S_2$ {\sl is an instance of} a signature
$\sig_1=\longsig{1}$,
written $\siginst{\sig_1}{}{\S_2}$, if there exists a realisation
$\rea$
such that $\rea(\S_1)=\S_2$ and $\Supp\rea\subseteq\N_1$.
(Note that if $\sig_1$ is type-explicit then there is at most one
such $\rea$.)\ 
A signature
$\sig_2=\longsig{2}$ {\sl is an instance of}
$\sig_1 =\longsig{1}$,
written $\siginst{\sig_1}{}{\sig_2}$, if
$\siginst{\sig_1}{}{\S_2}$ and $\N_2\cap(\NamesFcn\sig_1)=\emptyset$.
It can be shown that $\siginst{\sig_1}{}{\sig_2}$ iff, for all $\S$,
whenever $\siginst{\sig_2}{}{\S}$ then $\siginst{\sig_1}{}{\S}$.

\subsection{Functor Signature Instantiation}
A\index{36.5} pair $(\S,(\N')\S')$ is called a {\sl functor instance}.
Given $\funsig=\longfunsig{1}$,
a functor instance $(\S_2,(\N_2')\S_2')$ is an {\sl instance} of
$\funsig$,
written $\funsiginst{\funsig}{}{(\S_2,(\N_2')\S_2')}$,
if there exists a realisation $\rea$
such that
$\rea(\S_1,(\N_1')\S_1')=(\S_2,(\N_2')\S_2')$ and
$\Supp\rea\subseteq\N_1$.
%
\subsection{Enrichment}
\label{enrichment-sec}
In\index{37.1} matching a structure to a signature, the structure will be allowed both to
have more components, and to be more polymorphic, than (an instance of) the
signature.  Precisely, we  define enrichment of structures, environments and
type structures by mutual recursion as follows.

A structure $\S_1=(\m_1,\E_1)$
{\sl enriches} another structure
$\S_2=(\m_2,\E_2)$, written $\S_1\succ\S_2$, if
\begin{enumerate}
\item $\m_1=\m_2$
\item $\E_1\succ\E_2$
\end{enumerate}
An environment $\E_1=\longE{1}$
{\sl enriches} another environment $\E_2=$ $\longE{2}$,
written $\E_1\succ\E_2$,
if
\begin{enumerate}
\item $\Dom\SE_1\supseteq\Dom\SE_2$, and $\SE_1(\strid)\succ\SE_2(\strid)$
                                               for all $\strid\in\Dom\SE_2$
\item $\Dom\TE_1\supseteq\Dom\TE_2$, and $\TE_1(\tycon)\succ\TE_2(\tycon)$
                                               for all $\tycon\in\Dom\TE_2$
\item $\Dom\VE_1\supseteq\Dom\VE_2$, and $\VE_1(\id)\succ\VE_2(\id)$
                                               for all $\id\in\Dom\VE_2$
\item $\Dom\EE_1\supseteq\Dom\EE_2$, and $\EE_1(\exn)=\EE_2(\exn)$
                                               for all $\exn\in\Dom\EE_2$
\end{enumerate}
Finally, a type structure $(\theta_1,\CE_1)$
{\sl enriches} another type structure $(\theta_2,\CE_2)$,
written $(\theta_1,\CE_1)\succ(\theta_2,\CE_2)$,
if
\begin{enumerate}
\item $\theta_1=\theta_2$
\item Either $\CE_1=\CE_2$ or $\CE_2=\emptymap$
\end{enumerate}

\subsection{Signature Matching}
\label{sigmatch-sec}
A\index{37.2} structure $\S$ {\sl matches} a signature $\sig_1$ if there exists
a structure $\S^-$ such that $\sig_1\geq\S^-\prec\S$. Thus matching
is a combination of instantiation and enrichment. There is at most
one such $\S^-$, given $\sig_1$ and $\S$. Moreover, writing $\sig_1=
\longsig{1}$, if $\sig_1\geq\S^-$ then there exists a realisation $\rea$
with $\Supp\rea\subseteq\N_1$ and $\rea(\S_1)=\S^-$.
We shall then say that $\S$ matches $\sig_1$ {\em via} $\rea$.
(Note that if $\sig_1$ is type-explicit 
then $\rea$ is uniquely determined by $\sig_1$ and $\S$.)

A\index{37.2.5} signature $\sig_2$ {\em matches} a signature $\sig_1$
if for all structures $\S$, if $\S$ matches $\sig_2$ then $\S$
matches $\sig_1$. It can be shown that $\sig_2=\longsig{2}$ matches
$\sig_1=\longsig{1}$ if and only if there exists a realisation
$\rea$ with $\Supp\rea\subseteq\N_1$ and $\rea(\S_1)\prec\S_2$
and $\N_2\cap\NamesFcn\sig_1=\emptyset$.

\subsection{Principal Signatures}
\label{prinsig-sec}
The definitions in this section concern the elaboration of signature
expressions; more precisely they concern inferences of sentences of the
form $\B\ts\sigexp\ra\S$, where $\S$ is a structure and $\B$ is a basis.
Recall, from Section~\ref{admis-sec}, that the assembly of all semantic
objects in such an inference must be admissible.

For any basis $\B$ and any structure $\S$, 
we say that $\B$ {\sl covers} $\S$
if for every substructure $(m,E)$ of $\S$ such that
$m\in\of{\N}{\B}$:
\begin{enumerate}
\item
For every structure identifier $\strid\in\Dom\E$,
$\B$ contains a substructure $(m,\E')$ with $m$
free and $\strid\in\Dom\E'$
\item
For every type constructor $\tycon\in\Dom\E$,
$\B$ contains a substructure $(m,\E')$ with $m$ free
and $\tycon\in\Dom\E'$
\end{enumerate}
(This condition is not a consequence of consistency of $\{\B,\S\}$; 
informally, it states that if $\S$ shares a substructure with $\B$,
then $\S$ mentions no more components of the substructure than
$\B$ does.)



We\index{38.1} say that a signature
$\longsig{}$ is {\sl principal for $\sigexp$ in $\B$} if, choosing $\N$
so that $(\of{\N}{\B})\cap\N=\emptyset$,
\begin{enumerate}
\item $\B$ covers $\S$ 
\item $\B\vdash\sigexp\ra\S$
\item Whenever $\B\vdash\sigexp\ra\S'$, then $\sigord{\longsig{}}{}{\S'}$
\end{enumerate}
We claim that if $\sigexp$ elaborates in $\B$ to some structure covered
by $\B$, then it possesses a principal signature in $\B$.

Analogous to the definition given for type environments in
Section~\ref{typeenv-wf-sec}, we say that a semantic object $A$
{\sl respects equality} if every type environment occurring in 
$A$ respects equality. 
%
%
%Further, let $T$ be the set of type names
%$\t$ such that $(\t,\CE)$ occurs in $A$ for some
%$\CE\neq\emptymap$.  Then $A$ is said to {\sl maximise equality}
%if (a) $A$ respects equality, and also (b) if any larger subset of
%$T$ were to admit equality (without any change in the equality
%attribute of any type names not in $T$) then $A$ would cease to
%respect equality.
%

Now\index{38.5} let us assume that $\sigexp$ possesses a principal signature
$\sig_0=\longsig{0}$ in $B$. We wish to
define, in terms of $\sig_0$, another signature $\sig$ which provides more
information about the equality attributes of structures which will
match $\sig_0$. To this end, let $\T_0$ be the set of type names $\t\in\N_0$
which do not admit equality, and such that $(\t,\CE)$ occurs in $\S_0$
for some $\CE\neq\emptymap$.  Then we say $\sig$ is 
{\sl equality-principal for $\sigexp$ in $\B$} if
\begin{enumerate}
\item
$\sig$ respects equality
\item
$\sig$ is obtained from $\sig_0$ just by making as many
members of $\T_0$ admit equality as possible, subject to 1.~above
\end{enumerate}
It is easy to show that, if any such $\sig$ exists, it is determined
uniquely by $\sig_0$; moreover, $\sig$ exists if $\sig_0$ itself
respects equality.

%
\clearpage

%                   Inference Rules
%
\subsection{Inference Rules}
\label{statmod-rules-sec}
As\index{39.1} for the Core, the rules of the Modules static semantics allow
sentences of the form
\[ A\ts\phrase\ra A'\]
to be inferred, where in this case $A$ is either a basis, a context or
an environment and $A'$ is a semantic object.  The convention for options
is as in the Core semantics. 

Although not assumed in our definitions, it is intended that every basis
$\B=\N,\F,\G,\E$ in which a $\topdec$ is elaborated has the property
that $\NamesFcn\F\ \cup$\linebreak$\NamesFcn\G\cup\NamesFcn\E\subseteq\N$. This is not
the case for bases in which signature expressions and specifications are
elaborated, but the following Theorem can be proved:
\begin{quote}
Let S be an inferred sentence $\B\ts\topdec\ra\B'$ in which $\B$ satisfies
the above condition. Then $\B'$ also satisfies the condition.

Moreover, if S$'$ is a sentence of the form
$\B''\ts\phrase\ra A$ occurring in a proof of S, where $\phrase$ is
either a structure expression or a structure-level declaration, then $\B''$
also satisfies the condition.

Finally, if $\T,\U,\E\ts\phrase\ra A$ occurs
in a proof of S, where $\phrase$ is a phrase of the Core, then
$\TyNamesFcn\E\subseteq\T$.
\end{quote}
%		SEMANTICS
%
%                       Structure Expressions
%
\rulesec{Structure Expressions}{\B\ts\strexp\ra \S}
\begin{equation}	% generative strexp
\label{generative-strexp-rule}
\frac{\B\ts\strdec\ra\E\qquad\m\notin(\of{\N}{\B})\cup\NamesFcn\E}
     {\B\ts\encstrexp\ra(\m,\E)}\index{39.2}
\end{equation}
\begin{equation}	% longstrid
%\label{longstrid-strexp-rule}
\frac{\B(\longstrid)=\S}
     {\B\ts\longstrid\ra\S}
\end{equation}

\vspace{6pt}
\begin{equation}		% functor application
\label{functor-application-rule}
\frac{ \begin{array}{c}
        \B\ts\strexp\ra\S\\
        \funsiginst{\B(\funid)}{}{(\S'',(\N')\S')}\ ,
                                                    \ \S\succ\S''\\
        (\of{\N}{\B})\cap\N'=\emptyset
       \end{array}
     }
     {\B\ts\funappstr\ra\S'}
\end{equation}

\vspace{6pt}
\begin{equation}	% let strexp
\label{letstrexp-rule}
\frac{\B\ts\strdec\ra\E\qquad\B\oplus\E\ts\strexp\ra\S}
     {\B\ts\letstrexp\ra\S}
\end{equation}

\comments
\begin{description}
\item{(\ref{generative-strexp-rule})}
   The side condition ensures that each generative structure
expression receives a new name. If the expression occurs in
a functor body the structure name will be bound by $(\N')$ in
rule~\ref{funbind-rule}; this will ensure that for each application of the 
functor, by rule~\ref{functor-application-rule}, a new distinct name
will be chosen for the structure generated.
%
\item{(\ref{functor-application-rule})}
   The side condition $ (\of{\N}{\B})\cap\N'=\emptyset$  can always
be satisfied by renaming bound names in $(\N')S'$ thus ensuring that the
generated structures receive new names.\index{40.1}

Let $\B(\funid)=(N)(\S_f,(N')\S_f')$. Assuming that $(\N)\S_f$ is
type-explicit, the realisation $\rea$ for which
$\rea(\S_f,(N')\S_f')=(\S'',(\N')\S')$ is uniquely determined by $\S$,
since $\S\succ\S''$ can only hold if the type names and structure
names in $\S$ and $\S''$ agree.  Recall that enrichment $\succ$ allows
more components and more polymorphism, while instantiation $\geq$ does
not.

Sharing between argument and result specified in the declaration of
the functor $\funid$ is represented by the occurrence of the same name
in both $\S_f$ and $\S_f'$, and this repeated occurrence is preserved
by $\rea$, yielding sharing between the argument structure $\S$ and
the result structure $\S'$ of this functor application.
%
\item{(\ref{letstrexp-rule})}
   The use of $\oplus$, here and elsewhere, ensures that structure
and type names generated by
the first sub-phrase
are distinct from names generated by the second
sub-phrase.
\end{description}

%                              declarations
\rulesec{Structure-level Declarations}{\B\ts\strdec\ra\E}   		
\begin{equation}                % core declaration
\label{dec-rule}
\frac{ \of{\C}{\B}\ts\dec\ra\E
       \quad\E\ {\rm principal\ for\ \dec\ in\ } (\of{\C}{\B})
%version 2:        \quad\imptyvars\E=\emptyset
}
     { \B\ts\dec\ra\E }\index{40.2}
\end{equation}

\vspace{6pt}
\begin{equation}        	% structure declaration
%\label{structure-decl-rule}
\frac{ \B\ts\strbind\ra\SE }
     { \B\ts\singstrdec\ra\SE\ \In\ \Env }
\end{equation}

\vspace{6pt}
\begin{equation}                % local structure-level declaration
%\label{local structure-level declaration}
\frac{ \B\ts\strdec_1\ra\E_1\qquad
       \B\oplus\E_1\ts\strdec_2\ra\E_2 }
     { \B\ts\localstrdec\ra\E_2 }
\end{equation}

\vspace{6pt}
\begin{equation}                % empty declaration
%\label{empty-strdec-rule}
\frac{}
     {\B\ts\emptystrdec\ra \emptymap{\rm\ in}\ \Env}
\end{equation}

\vspace{6pt}
\begin{equation}		% sequential declaration
%\label{sequential-strdec-rule}
\frac{ \B\ts\strdec_1\ra\E_1\qquad
       \B\oplus\E_1\ts\strdec_2\ra\E_2 }
     { \B\ts\seqstrdec\ra\plusmap{\E_1}{\E_2} }
\end{equation}
\comments
\begin{description}
\item{(\ref{dec-rule})}
The side condition ensures that all type schemes in $\E$ are as
general as possible.
% and that no imperative type variables occur
%free in $\E$.
%from version 1:
%   The side condition ensures that all type schemes in $\E$ are as
%general as possible and that all new type names in $\E$ admit
%equality, if possible.
\end{description}
\rulesec{Structure Bindings}{\B\ts\strbind\ra\SE}
\begin{equation}                % structure binding
\label{structure-binding-rule}
\frac{ \begin{array}{cl}
       \B\ts\strexp\ra\S\qquad\langle\B\ts\sigexp\ra\sig\ ,
                                      \ \sig\geq\S'\prec\S\rangle\\
       \langle\langle\plusmap{\B}{\NamesFcn\S}\ts
                                      \strbind\ra\SE\rangle\rangle
       \end{array}
     }
     { \B\ts\strbinder\ra\{\strid\mapsto\S\langle'\rangle\}
       \ \langle\langle +\ \SE\rangle\rangle }\index{41.1}
%version 2:\frac{ \begin{array}{cl}
%       \B\ts\strexp\ra\S\qquad\langle\B\ts\sigexp\ra\S'\ ,
%                                      \ \S\succ\S'\rangle\\
%       \langle\langle\plusmap{\B}{\NamesFcn\S}\ts
%                                      \strbind\ra\SE\rangle\rangle
%       \end{array}
%     }
%     { \B\ts\strbinder\ra\{\strid\mapsto\S\langle'\rangle\}
%       \ \langle\langle +\ \SE\rangle\rangle }\index{41.1}
\end{equation}
\comment If present, $\sigexp$ has the effect of restricting the
view which $\strid$ provides of $\S$ while retaining sharing of names. 
The notation $\S\langle'\rangle$ means $\S'$, if the first option is present,
and $\S$ if not.
%
%                   Signature Rules
%
\rulesec{Signature Expressions}{\B\ts\sigexp\ra\S}
\begin{equation}		% encapsulation sigexp
\label{encapsulating-sigexp-rule}
\frac{\B\ts\spec\ra\E }
     {\B\ts\encsigexp\ra (\m,\E)}\index{41.2}
\end{equation}

\begin{equation}		% signature identifier
\label{signature-identifier-rule}
\frac{ \sigord{\B(\sigid)}{}{\S} }
     { \B\ts\sigid\ra\S }
\end{equation}
\comments
\begin{description}
\item{(\ref{encapsulating-sigexp-rule})}
   In contrast to rule~\ref{generative-strexp-rule}, $m$ is not here 
required to be new. 
The name $m$ may be chosen to achieve the sharing required
in rule~\ref{strshareq-rule}, or to achieve the enrichment side conditions
of rule~\ref{structure-binding-rule} or \ref{funbind-rule}. 
The choice of $m$ must result in an admissible object.
\item{(\ref{signature-identifier-rule})}
   The instance $\S$ of $\B(\sigid)$ is not determined by this rule,
but -- as in rule~\ref{encapsulating-sigexp-rule} -- the instance
may  be chosen to achieve sharing properties or enrichment
conditions.
\end{description}

\rulesec{}{\B\ts\sigexp\ra\sig}
\begin{equation}		% any sigexp
\label{topmost-sigexp-rule}
\frac{\begin{array}{c}
\B\ts\sigexp\ra\S\quad\mbox{$(\N)\S$ equality-principal for $\sigexp$ in $\B$}\\
\mbox{$(\N)\S$ type-explicit}
      \end{array}}
     {\B\ts\sigexp\ra (\N)\S}\index{41.25}
\end{equation}

\comment
A signature expression $\sigexp$ which is an immediate constituent of
a structure binding, a signature binding, a functor binding or a
functor signature is elaborated to an equality-principal and type-explicit
signature, see rules \ref{structure-binding-rule}, \ref{sigbind-rule}, 
\ref{funsigexp-rule} and \ref{funbind-rule}.  By contrast, signature 
expressions occurring in structure descriptions are elaborated to
structures using the liberal rules
\ref{encapsulating-sigexp-rule} and \ref{signature-identifier-rule}, 
see rule~\ref{strdesc-rule}, so that names can be chosen to achieve
sharing, when necessary.

\rulesec{Signature Declarations}{\B\ts\sigdec\ra\G}
\begin{equation}	% single signature declaration
\label{single-sigdec-rule}
\frac{ \B\ts\sigbind\ra\G }
     { \B\ts\singsigdec\ra\G }\index{41.3}
\end{equation}

\begin{equation}	% empty signature declaration
%\label{empty-sigdec-rule}
\frac{}
     { \B\ts\emptysigdec\ra\emptymap }
\end{equation}

\begin{equation}	% sequential signature declaration
\label{sequence-sigdec-rule}
\frac{ \B\ts\sigdec_1\ra\G_1 \qquad \plusmap{\B}{\G_1}\ts\sigdec_2\ra\G_2 }
     { \B\ts\seqsigdec\ra\plusmap{\G_1}{\G_2} }
\end{equation}
\comments
\begin{description}
%
\item{(\ref{single-sigdec-rule})}
The first closure restriction of Section~\ref{closure-restr-sec}
can be  enforced by replacing the $\B$ in the premise by $\B_0+\of{\G}{\B}$.

\item{(\ref{sequence-sigdec-rule})}
   A signature declaration does not create any new structures
or types; hence the use of $+$ instead of $\oplus$.
\end{description}

\rulesec{Signature Bindings}{\B\ts\sigbind\ra\G}
\begin{equation}	% signature binding
\label{sigbind-rule}
\frac{ \B\ts\sigexp\ra\sig
        \qquad\langle\B\ts\sigbind\ra\G\rangle }
     { \B\ts\sigbinder\ra\{\sigid\mapsto\sig\}
       \ \langle +\ \G\rangle }\index{42.1}
\end{equation}
\comment The  condition that $\sig$ be equality-principal,
implicit in the first premise, ensures that the
signature found is as general as possible given the sharing
constraints present in $\sigexp$. 
%version 2: The set $\N$ is determined by
%the definition of principality in Section~\ref{prinsig-sec}.
%
                     % Specifications
\rulesec{Specifications}{\B\ts\spec\ra\E}
\begin{equation}        % value specification
\label{valspec-rule}
\frac{ \of{\C}{\B}\ts\valdesc\ra\VE }
     { \B\ts\valspec\ra\cl{}{\VE}\ \In\ \Env }\index{42.2}
\end{equation}

\begin{equation}        % type specification
\label{typespec-rule}
\frac{ \of{\C}{\B}\ts\typdesc\ra\TE }
     { \B\ts\typespec\ra\TE\ \In\ \Env }
\end{equation}

\begin{equation}        % eqtype specification
\label{eqtypspec-rule}
\frac{ \of{\C}{\B}\ts\typdesc\ra\TE \qquad
       \forall(\theta,\CE)\in \Ran\TE,\ \theta {\rm\ admits\ equality} }
     { \B\ts\eqtypespec\ra\TE\ \In\ \Env }
\end{equation}

\begin{equation}        % data specification
\label{datatypespec-rule}
\frac{ \plusmap{\of{\C}{\B}}{\TE}\ts\datdesc\ra\VE,\TE }
     { \B\ts\datatypespec\ra(\VE,\TE)\ \In\ \Env }
\end{equation}

\begin{equation}        % exception specification
\label{exceptionspec-rule}
\frac{ \of{\C}{\B}\ts\exndesc\ra\EE\quad\VE=\EE }
     { \B\ts\exceptionspec\ra(\VE,\EE)\ \In\ \Env }
\end{equation}

\begin{equation}        % structure specification
%\label{structurespec-rule}
\frac{ \B\ts\strdesc\ra\SE }
     { \B\ts\structurespec\ra\SE\ \In\ \Env }
\end{equation}

\begin{equation}        % sharing specification
%\label{sharingspec-rule}
\frac{ \B\ts\shareq\ra\emptymap }
     { \B\ts\sharingspec\ra\emptymap\ \In\ \Env }\index{42.3}
\end{equation}

\begin{equation}        % local specification
%\label{localspec-rule}
\frac{ \B\ts\spec_1\ra\E_1 \qquad \plusmap{\B}{\E_1}\ts\spec_2\ra\E_2 }
     { \B\ts\localspec\ra\E_2 }
\end{equation}

\begin{equation}        % open specification
%\label{openspec-rule}
\frac{ \B(\longstrid_1)=(\m_1,\E_1)\quad\cdots\quad
       \B(\longstrid_n)=(\m_n,\E_n) }
     { \B\ts\openspec\ra\E_1 + \cdots +\E_n }
\end{equation}

\begin{equation}        % include signature specification
\label{inclspec-rule}
\frac{ \sigord{\B(\sigid_1)}{}{(\m_1,\E_1)} \quad\cdots\quad
       \sigord{\B(\sigid_n)}{}{(\m_n,\E_n)} }
     { \B\ts\inclspec\ra\E_1 + \cdots +\E_n }
\end{equation}

\begin{equation}        % empty specification
%\label{emptyspec-rule}
\frac{}
     { \B\ts\emptyspec\ra\emptymap{\rm\ in}\ \Env }
\end{equation}

\begin{equation}        % sequential specification
%\label{seqspec-rule}
\frac{ \B\ts\spec_1\ra\E_1 \qquad \plusmap{\B}{\E_1}\ts\spec_2\ra\E_2 }
     { \B\ts\seqspec\ra\plusmap{\E_1}{\E_2} }
\end{equation}
\comments
\begin{description}
\item{(\ref{valspec-rule})}
   $\VE$ is determined by $\B$ and $\valdesc$.
\item{(\ref{typespec-rule})--(\ref{datatypespec-rule})}
   The type functions in $\TE$ may be chosen to achieve the sharing hypothesis
of rule~\ref{typshareq-rule} or the enrichment conditions of 
rules~\ref{structure-binding-rule} and~\ref{funbind-rule}. In particular, the type
names in $\TE$ in rule~\ref{datatypespec-rule} need not be new.
Also, in rule~\ref{typespec-rule} the type functions in $\TE$ may admit
equality.
%
\item{(\ref{exceptionspec-rule})}
   $\EE$ is determined by $\B$ and $\exndesc$ and contains monotypes only.
\item{(\ref{inclspec-rule})}
   The names $\m_i$ in the instances may be chosen to achieve sharing or
enrichment conditions.\index{43.0}
\end{description} 

                  % Descriptions
\rulesec{Value Descriptions}{\C\ts\valdesc\ra\VE}
\begin{equation}         % value description
%\label{valdesc-rule}
\frac{ \C\ts\ty\ra\tau\qquad
       \langle\C\ts\valdesc\ra\VE\rangle }
     { \C\ts\valdescription\ra\{\var\mapsto\tau\}
       \ \langle +\ \VE\rangle }\index{43.1}
\end{equation}

\rulesec{Type Descriptions}{\C\ts\typdesc\ra\TE}
\begin{equation}         % type description
\label{typdesc-rule}
\frac{ \tyvarseq = \alphak
       \qquad\langle \C\ts\typdesc\ra\TE\rangle\qquad\arity\theta=k }
     { \C\ts\typdescription\ra\{\tycon\mapsto(\theta,\emptymap)\}
       \ \langle +\ \TE\rangle }\index{43.2}
\end{equation}
\comment Note that any $\theta$ of arity $k$ may be chosen but that
the constructor environment in the resulting type structure must be
empty. For example, \mbox{\ml{datatype s=c type t sharing s=t}}\  
is a legal specification, but the type structure bound to \ml{t}
does not bind any value constructors.

\rulesec{Datatype Descriptions}{\C\ts\datdesc\ra\VE,\TE}
\begin{equation}         % datatype description
\label{datdesc-rule}
\frac{ \tyvarseq = \alphak\qquad\C,\alphakt\ts\condesc\ra\CE
       \qquad\langle\C\ts\datdesc\ra\VE,\TE\rangle }
     { \begin{array}{cl}
       \C\ts\datdescription\ra\\
       \qquad\qquad\cl{}{\CE}\langle +\ \VE\rangle,\
       \{\tycon\mapsto(t,\cl{}{\CE})\}\ \langle +\ \TE\rangle
       \end{array}
     }\index{43.3}
\end{equation}

\rulesec{Constructor Descriptions}{\C,\tau\ts\condesc\ra\CE}
\begin{equation}         % constructor description
%\label{condesc-rule}
\frac{\langle\C\ts\ty\ra\tau'\rangle\qquad
      \langle\langle\C,\tau\ts\condesc\ra\CE\rangle\rangle }
     {\begin{array}{c}
      \C,\tau\ts\longcondescription\ra\\
      \qquad\qquad\qquad\{\con\mapsto\tau\}\
     \langle +\ \{\con\mapsto\tau'\to\tau\}\ \rangle\
      \langle\langle +\ \CE\rangle\rangle
      \end{array}
     }\index{43.35}
\end{equation}

\rulesec{Exception Descriptions}{\C\ts\exndesc\ra\EE}
\begin{equation}         % exception description
\label{exndesc-rule}
\frac{ \langle\C\ts\ty\ra\tau\qquad\TyVarsFcn(\tau)=\emptyset\rangle\qquad
       \langle\langle\C\ts\exndesc\ra\EE\rangle\rangle }
     { \begin{array}{l}
        \C\ts\exndescriptiona\ra\\
        \quad\quad\{\exn\mapsto\EXCN\}\ \langle +\ \{\exn\mapsto\tau\rightarrow\EXCN\}\rangle\ \langle\langle +\ \EE\rangle\rangle 
       \end{array}
     }\index{43.4}
\end{equation}

\rulesec{Structure Descriptions}{\B\ts\strdesc\ra\SE}
\begin{equation}
\label{strdesc-rule}
\frac{ \B\ts\sigexp\ra\S\qquad\langle\B\ts\strdesc\ra\SE\rangle }
     { \B\ts\strdescription\ra\{\strid\mapsto\S\}\ \langle +\ \SE\rangle }\index{43.5}
\end{equation}


\rulesec{Sharing Equations}{\B\ts\shareq\ra\emptymap}
\begin{equation}          % structure sharing equation
\label{strshareq-rule}
\frac{ \of{\m}{\B(\longstrid_1)}=\cdots =\of{\m}{\B(\longstrid_n)} }
     { \B\ts\strshareq\ra\emptymap }\index{44.1}
\end{equation}

\vspace{6pt}
\begin{equation}          % type sharing equation
\label{typshareq-rule}
\frac{ \of{\typefcn}{\B(\longtycon_1)}=\cdots=\of{\typefcn}{\B(\longtycon_n)} }
     { \B\ts\typshareq\ra\emptymap }
\end{equation}

\vspace{6pt}
\begin{equation}          % multiple sharing equation
%\label{multshareq-rule}
\frac{ \B\ts\shareq_1\ra\emptymap\qquad\B\ts\shareq_2\ra\emptymap }
     { \B\ts\multshareq\ra\emptymap }
\end{equation}
\comments
\begin{description}
\item{(\ref{strshareq-rule})}
   By the definition of consistency the premise is weaker than\linebreak
$\B(\longstrid_1) = \cdots = \B(\longstrid_n)$.
Two different structures with the same name may be thought of
as representing different views. The requirement that $\B$ is 
consistent forces different views to be consistent.
\end{description}
%
\begin{description}
\item{(\ref{typshareq-rule})}
   By\index{44.1.5} 
the definition of consistency the premise is weaker than\linebreak
$\B(\longtycon_1) = \cdots = \B(\longtycon_n)$.
A type structure with empty constructor environment may have the
same type name as one with a non-empty constructor environment;
the former could arise from a type description, and the latter
from a datatype description. 
%However, the requirement that $\B$ is
%consistent will prevent two type structures with different 
%non-empty constructor environments from sharing the same type name.
However, the requirement that $\B$ is
consistent will prevent two type structures with constructor
environments which have different 
non-empty domains from sharing the same type name.

\end{description}
%    			Functor Specification rules
%
\rulesec{Functor Specifications}{\B\ts\funspec\ra\F}
\begin{equation}        % single functor specification
\label{singfunspec-rule}
\frac{ \B\ts\fundesc\ra\F }
     { \B\ts\singfunspec\ra\F }\index{44.2}
\end{equation}

\vspace{6pt}
\begin{equation}        % empty functor specification
%\label{emptyfunspec-rule}
\frac{}
     { \B\ts\emptyfunspec\ra\emptymap }
\end{equation}

\vspace{6pt}
\begin{equation}        % sequential functor specification
%\label{seqfunspec-rule}
\frac{ \B\ts\funspec_1\ra\F_1\qquad
       \B+\F_1\ts\funspec_2\ra\F_2 }
     { \B\ts\seqfunspec\ra\plusmap{\F_1}{\F_2} }
\end{equation}
\comments
\begin{description}
\item{(\ref{singfunspec-rule})}
The second closure restriction of Section~\ref{closure-restr-sec}
can be enforced by replacing the $\B$ in the premise by $\B_0+\of{\G}{\B}$.
\end{description}
\rulesec{Functor Descriptions}{\B\ts\fundesc\ra\F}
\begin{equation}        % functor description
%\label{fundesc-rule}
\frac{ \B\ts\funsigexp\ra\funsig\qquad
       \langle\B\ts\fundesc\ra\F\rangle}
     { \B\ts\longfundesc\ra\{\funid\mapsto\funsig\}
       \langle +\ \F\rangle}\index{44.3}
\end{equation}

\rulesec{Functor Signature Expressions}{\B\ts\funsigexp\ra\funsig}
\begin{equation}	% functor signature
\label{funsigexp-rule}
%version 1:
%\frac{
%      \begin{array}{c}
%      \B\ts\sigexp\ra\S\qquad\longsig{}{\rm\ principal\ in\ }\B\\
%      \B\oplus\{\strid\mapsto\S\} \ts\sigexp'\ra\S'\\
%      \N' = \NamesFcn\S'\setminus((\of{\N}{\B})\cup\N) 
%      \end{array}
%     }
%     {\B\ts\longfunsigexpa\ra(\N)(\S,(\N')\S')}\index{44.4}
%version2: \frac{\begin{array}{rl}
%      \B\ts\sigexp\ra\S&\mbox{$(N)S$ principal in $\B$}\\
%      \B\oplus\{\strid\mapsto\S\}\ts\sigexp'\ra\S'&
%      \mbox{$(N')S'$ principal in $\B\oplus\{\strid\mapsto\S\}$}
%      \end{array}}
%     {\B\ts\longfunsigexpa\ra(N)(S,(N')S')}\index{44.4}
%\end{equation}
\frac{\B\ts\sigexp\ra(\N)\S\qquad
      \B\oplus\{\strid\mapsto\S\}\ts\sigexp'\ra(\N')\S'}
     {\B\ts\longfunsigexpa\ra(N)(S,(N')S')}\index{44.4}
\end{equation}
\comment
The signatures $(\N)\S$ and $(\N')\S'$ are equality-principal 
and type-explicit, see rule~\ref{topmost-sigexp-rule}.
%    			Functor and Program rules

\rulesec{Functor Declarations}{\B\ts\fundec\ra\F}
\begin{equation}        % single functor declaration
\label{singfundec-rule}
\frac{ \B\ts\funbind\ra\F }
     { \B\ts\singfundec\ra\F }\index{45.1}
\end{equation}

\vspace{6pt}
\begin{equation}        % empty functor declaration
%\label{emptyfundec-rule}
\frac{}
     { \B\ts\emptyfundec\ra\emptymap }
\end{equation}

\vspace{6pt}
\begin{equation}        % sequential functor declaration
%\label{seqfundec-rule}
\frac{ \B\ts\fundec_1\ra\F_1\qquad
       \B+\F_1\ts\fundec_2\ra\F_2 }
     { \B\ts\seqfundec\ra\plusmap{\F_1}{\F_2} }\index{45.1.5}
\end{equation}
\comments
\begin{description}
\item{(\ref{singfundec-rule})}
The third closure restriction of Section~\ref{closure-restr-sec}
can be enforced by replacing the $\B$ in the premise 
by $\B_0+(\of{\G}{\B})+(\of{\F}{\B})$.
\end{description}
\rulesec{Functor Bindings}{\B\ts\funbind\ra\F}
\begin{equation}	% functor binding
\label{funbind-rule}
\frac{
      \begin{array}{c}
      \B\ts\sigexp\ra(\N)\S\qquad
      \B\oplus\{\strid\mapsto\S\} \ts\strexp\ra\S' \\
       \langle
      \B\oplus\{\strid\mapsto\S\} \ts\sigexp'\ra\sig',\ \sig'\geq\S''\prec\S'
       \rangle\\
      \N' = \NamesFcn\S'\setminus((\of{\N}{\B})\cup\N) \\
       \langle\langle\B\ts\funbind\ra\F\rangle\rangle
      \end{array}
     }
     {
      \begin{array}{c}
       \B\ts\funstrbinder\ \optfunbind\ra\\
       \qquad\qquad \qquad
              \{\funid\mapsto(\N)(\S,(\N')\S'\langle'\rangle)\}
              \ \langle\langle +\ \F\rangle\rangle
      \end{array}
     }\index{45.2}
\end{equation}
\comment The  requirement that $(\N)\S$ be equality-principal,
implicit in the first premise, forces $(\N)\S$ to be
as general as possible given the sharing constraints in $\sigexp$.
The requirement that $(\N)\S$ be type-explicit ensures that there is
at most one realisation via which an actual argument can match
$(\N)\S$.
Since $\oplus$ is used, any structure name $\m$ and type name $\t$ in
$\S$ acts like a constant in the functor body; in particular,
it ensures that further names generated during elaboration of the
body are distinct from $\m$ and $\t$. The set $\N'$ is
chosen such that every  name free
in $(\N)\S$ or $(\N)(\S,(\N')\S')$ is free in $\B$.

\rulesec{Top-level Declarations}{\B\ts\topdec\ra\B'}
%\rulesec{Programs}{\B\ts\program\ra\B'}
\begin{equation}	% structure-level declaration
\label{strdectopdec-rule}
\frac{\B\ts\strdec\ra\E \quad\imptyvars\E=\emptyset}
     {\B\ts\strdec\ra
      (\NamesFcn\E,\E)\ \In\ \Basis
     }\index{45.3}
\end{equation}

\vspace{6pt}
\begin{equation}	% signature declaration
%\label{sigdectopdec-rule}
\frac{\B\ts\sigdec\ra\G \quad\imptyvars\G=\emptyset}
     {\B\ts\sigdec\ra
      (\NamesFcn\G,\G)\ \In\ \Basis
     }\index{46.0}
\end{equation}

\vspace{6pt}
\begin{equation}	% functor declaration
\label{fundectopdec-rule}
\frac{\B\ts\fundec\ra\F \quad\imptyvars\F=\emptyset}
     {\B\ts\fundec\ra
      (\NamesFcn\F,\F)\ \In\ \Basis
     }
\end{equation}
\comments
\begin{description}
\item{(\ref{strdectopdec-rule})--(\ref{fundectopdec-rule})} The side
conditions ensure that no free imperative type variables enter the 
basis.\index{46.01}
\end{description}
%from version 1:
%\vspace{6pt}
%\begin{equation}	% sequential program
%\label{seqprog-rule}
%\frac{\B\ts\program_1\ra\B_1\qquad\
%      \plusmap{\B}{\B_1}\ts\program_2\ra\B_2
%     }
%     {\B\ts\seqprog\ra\plusmap{\B_1}{\B_2}}
%\end{equation}

\subsection{Functor Signature Matching}
\label{fun-sig-match-sec}
As\index{46} pointed out in Section~\ref{mod-gram-sec} on the 
grammar for Modules, there is no phrase class whose elaboration 
requires matching one functor signature to another functor signature.
But a precise definition of this matching is needed, since a 
functor $g$ may only be separately compiled in the presence of 
specification of any functor $f$ to which $g$ refers, and then a 
real functor $f$ must match this specification.
In the case, then, that $f$ has been specified by a functor signature
\[\funsig_1\ =\ \longfunsig{1}\]
and that later $f$ is declared with functor signature
\[\funsig_2\ =\ \longfunsig{2}\]
the following matching rule will be employed:

A functor signature
$\funsig_2\ =\ \longfunsig{2}$ {\sl matches} another functor signature,
$\funsig_1\ =\ \longfunsig{1}$, if there exists a realisation $\rea$ 
such that
\begin{enumerate}
\item $\longsig{1}$ matches $\longsig{2}$ via $\rea$, and
\item $\rea((\N_2')\S_2')$ matches $(\N_1')\S_1'$.
\end{enumerate}
The first condition ensures that the real functor signature $\funsig_2$
for $f$ requires the argument $\strexp$ of any application $\f(\strexp)$
to have no more sharing, and no more richness, than was predicted by
the specified signature $\funsig_1$.
The second condition ensures that the real functor signature $\funsig_2$,
instantiated to $(\rea\S_2,\rea((\N_2')\S_2'))$, provides in the result of
the application $\f(\strexp)$
no less sharing, and no less richness, than was predicted by
the specified signature $\funsig_1$.

%We claim that any phrase -- e.g. the declaration of the functor $g$ above --
%which elaborates successfully in a basis $\B$ with $\B(f)=\funsig_1$ will
%also elaborate successfully in the basis $\B+\{f\mapsto\funsig_2\}$.  This
%claim justifies our definition of functor matching.
% -- this claim is false because of open.



\section{Dynamic Semantics for the Core}
\subsection{Reduced Syntax}
Since\index{47.1} types are fully dealt with in the static semantics,
the dynamic semantics ignores them.  The Core syntax is therefore
reduced by the following transformations, for the purpose of the dynamic
semantics:
\begin{itemize}
\item All explicit type ascriptions ``\ml{:} $\ty$'' are omitted, and
      qualifications ``$\OF\  \ty$'' are omitted from exception
      bindings.
\item Any declaration of the form ``$\typedec$'' or ``$\datatypedec$''
      is replaced by the empty declaration.
\item A declaration of the form ``$\abstypedec$'' is replaced by ``$\dec$''.
\item The Core phrase classes TypBind, DatBind, $\ConBind$, Ty and
      TyRow are omitted.
%version 2:\item The Core phrase classes $\typbind$, $\datbind$, $\constrs$, $\ty$ and
%      $\labtys$ are omitted.
\end{itemize}

\subsection{Simple Objects}
All\index{47.2} objects in the dynamic semantics are built from
identifier classes together with the simple object classes shown (with the
variables which range over them) in Figure~\ref{simp-dyn-obj}.

\begin{figure}[h]
\vspace{2pt}
\begin{displaymath}
\begin{array}{rclr}
\A               & \in   & \Addr	& \mbox{addresses}\\
\e               & \in   & \Exc 	& \mbox{exception names}\\
b      		& \in	& \BasVal	& \mbox{basic values}\\
\sv             & \in   & \SVal         & \mbox{special values}\\
                &       & \{\FAIL\}     & \mbox{failure}\\   
\end{array}
\end{displaymath}
\caption{Simple Semantic Objects}
\label{simp-dyn-obj}
\vspace{3pt}
\end{figure}

$\Addr$ and $\Exc$ are infinite sets. BasVal is described below.
{\SVal} is the class of values denoted by the special constants
\SCon. Each integer or real constant denotes a value according to normal 
mathematical conventions; each string constant denotes a sequence of ASCII
characters as explained in Section~\ref{cr:speccon}. The value denoted
by {\scon} is written $\sconval(\scon)$.
$\FAIL$ is the result of a failing attempt to match a value and a
pattern. Thus $\FAIL$ is neither a value nor an exception, but simply
a semantic object used in the rules to express operationally
how matching proceeds.

Exception constructors evaluate to exception names, unlike value constructors
which simply evaluate to themselves. This is to accommodate the generative
nature of exception bindings;\index{48.0} each evaluation of a declaration of a
exception constructor binds it to a new unique name.

\subsection{Compound Objects}

The\index{48.2} compound objects for the dynamic semantics are
shown in Figure~\ref{comp-dyn-obj}.
Many conventions and notations are adopted as in the static semantics; in
particular projection, injection and modification all retain
their meaning.  We generally omit the injection functions taking $\Con$,
$\Con\times\Val$ etc into $\Val$. For records $\r\in\Record$ however,
we write this injection explicitly as ``$\In\ \Val$''; this accords with
the fact that there is a separate phrase class ExpRow, whose members
evaluate to records. 

We take $\cup$ to mean disjoint union over
semantic object classes. We also understand all the defined object
classes to be disjoint. A particular case deserves mention; $\ExVal$
and $\Pack$ (exception values and packets) are isomorphic classes,
but the latter class corresponds to exceptions which have been
raised, and therefore has different semantic significance from the
former, which is just a subclass of values.


\begin{figure}[t]
\vspace{2pt}
\begin{displaymath}
\begin{array}{rcl}
        \V	&\in	&\Val =\{\mbox{\tt :=}\}\cup\SVal\cup\BasVal\cup\Con\\
                &       &\qquad\cup(\Con\times\Val)\cup\ExVal\\
                &       &\qquad\cup\Record\cup\Addr\cup\Closure\\
        \r      & \in   & \Record =  \finfun{\Lab}{\Val}\\
{\exval}      & \in   & \ExVal = \Exc \cup (\Exc\times\Val)\\
{[\exval]}\ {\rm or}\ \p
                & \in   & \Pack = \ExVal\\
(\match,\E,\VE) & \in   & \Closure = \Match\times\Env\times\VarEnv\\
        \mem    & \in   & \Mem = \finfun{\Addr}{\Val}\\
        \excs   & \in   & \ExcSet = \Fin(\Exc)\\
(\mem,\excs)\ {\rm or}\ \s
                & \in   & \State = \Mem\times\ExcSet\\
(\SE,\VE,\EE)\ {\rm or}\ \E
                & \in   & \Env = \StrEnv\times\VarEnv\times\ExnEnv\\
        \SE     & \in   & \StrEnv = \finfun{\StrId}{\Env}\\
        \VE	& \in	& \VarEnv = \finfun{\Var}{\Val}\\
        \EE	& \in	& \ExnEnv = \finfun{\Exn}{\Exc}\\
\end{array}
\end{displaymath}
\caption{Compound Semantic Objects\index{48.1}}
\label{comp-dyn-obj}
\vspace{3pt}
\end{figure}
%
%
Although the same names, e.g. $\E$ for an environment, are used
as in the static semantics, the objects denoted are different.  This need cause
no confusion since the static and dynamic semantics are presented %completely
separately. An important point is that structure names $\m$ have
no significance at all in the dynamic semantics; this explains why the
object class $\Str = \StrNames\times\Env$ is absent here -- for the dynamic
semantics the concepts {\sl structure} and {\sl environment} coincide.

\subsection{Basic Values}
The\index{49.1} basic values in $\BasVal$ are the values bound to predefined variables.
These values are denoted by the identifiers to which they are bound in the
initial dynamic basis (see Appendix~\ref{init-dyn-bas-app}), 
and are as follows:
\begin{verbatim}
           abs  floor  real  sqrt  sin  cos  arctan  exp  ln
              size  chr  ord  explode  implode  div  mod
                  ~  /  *  +  -  =  <>  <  >  <=  >=
        std_in  std_out  open_in  open_out  close_in  close_out
                input  output  lookahead  end_of_stream
\end{verbatim}
The meaning of basic values (almost all of which are functions) is
represented by the function
\[ \APPLY\ :\ \BasVal\times\Val\to\Val\cup\Pack \]
 which is detailed in Appendix~\ref{init-dyn-bas-app}.

\subsection{Basic Exceptions}
\label{bas-exc}
A\index{49.2} subset $\BasExc\subset\Exc$ of the exception names are bound to predefined
exception constructors.
These names are denoted by the identifiers to which they are bound in the
initial dynamic basis (see Appendix~\ref{init-dyn-bas-app}), 
and are as follows:
\begin{verbatim}
        Abs  Ord  Chr   Div  Mod  Quot  Prod  
        Neg  Sum  Diff  Floor  Sqrt  Exp  Ln
        Io   Match  Bind  Interrupt
\end{verbatim}
The exceptions on the first two  lines are raised by 
corresponding basic functions, where \verb+~+ {\tt /} {\tt *}
{\tt +} {\tt -} correspond respectively to {\tt Neg} {\tt Quot}
{\tt Prod} {\tt Sum} {\tt Diff}. The details are given
in Appendix~\ref{init-dyn-bas-app}. The exception $(\mbox{{\tt Io}},s)$,
where $s$ is a string, is raised
by certain of the basic input/output functions,
as detailed in Appendix~\ref{init-dyn-bas-app}.  
The exceptions ~\ml{Match}~ and
~\ml{Bind}~
are raised upon failure of pattern-matching in evaluating a 
function {\fnexp} or a
$\valbind$, as detailed in the rules to follow.   Finally, ~\ml{Interrupt}~
is raised by external intervention.

Recall from Section~\ref{further-restrictions-sec} 
that in the context {\fnexp}, the $\match$ 
must be irredundant and exhaustive and that the compiler should flag
the {\match} if it violates these restrictions. The exception~\ml{Match}
can only be raised for a match which is not exhaustive, and has therefore 
been flagged by the compiler.

%In a match of the form $\pat_1\ \ml{=>}\ \exp_1\ \ml{|}\ \ldots\ \ml{|}\ 
%            \pat_n\ \ml{=>}\ \exp_n$
%the pattern sequence $\pat_1,\ldots,\pat_n$ should be {\sl irredundant};
%that is, each $\pat_j$ must match some value
%(of the right type) which is not matched by $\pat_i$ for any $i<j$.
%In the context {\fnexp}, the $\match$ must also be {\sl exhaustive}; that is,
%every value (of the right type) must be matched by some $\pat_i$.
%The compiler must give warning on violation of these restrictions, 
%but should still compile the match. 
%The \ml{match} exception
%can only be raised for a match which is not exhaustive, and has therefore
%been flagged by the compiler.
%The restrictions are inherited by derived forms; in particular,
%this means that in the function binding 
% $\var\ \atpat_1\ \cdots\ \atpat_n\langle : \ty\rangle\ \ml{=}\ \exp$
%(consisting of one clause only), each separate $\atpat_i$ should be
%exhaustive by itself.

For each value binding $\pat\ \mbox{\ml{=}}\ \exp$ the compiler must issue a 
report (but still compile) if {\it either} pat is not exhaustive {\it or}
pat contains no variable. This will (on both counts) detect a mistaken
declaration like $\VAL\ \ml{nil}\ \mbox{\ml{=}}\ \exp$ in which the user
expects to declare a new variable \ml{nil} (whereas the language dictates
that \ml{nil} is here a constant pattern, so no variable gets declared).
However, these warnings should not be given when the binding is a component
of a top-level declaration $\valdec$; e.g. 
 $\VAL\ \mbox{\ml{x::l = }}\exp_1\ \mbox{\ml{\AND\ y = }}\exp_2$ 
is not faulted by the compiler at top level, but may of course generate
a \ml{Bind} exception.

\subsection{Closures}
The\index{50.1} informal understanding of a {\sl closure} $(\match,\E,\VE)$ is as follows:
when the closure is applied to a value $\V$,
$\match$ will be evaluated against $\V$, in the environment $\E$ modified in
a special sense by
$\VE$.  The domain $\Dom\VE$ of this third component contains those function
identifiers to be treated recursively in the evaluation.  To achieve this
effect, the evaluation of $\match$ will take place not in $\plusmap{\E}{\VE}$
but in $\plusmap{\E}{\Rec\VE}$, where
\[ \Rec\ :\ \VarEnv\to\VarEnv \]
is defined as follows:
\begin{itemize}
\item $\Dom(\Rec\VE)=\Dom\VE $
\item If $\VE(\var)\notin\Closure$, then $(\Rec\VE)(\var)=\VE(\var)$
\item If $\VE(\var)=(\match',\E',\VE')$
      then $(\Rec\VE)(\var)=(\match',\E',\VE)$
\end{itemize}
The effect is that, before application of $(\match,\E,\VE)$ to $\V$, the
closure values in $\Ran\VE$ are ``unrolled'' once, to prepare for their possible
recursive application during the evaluation of $\match$ upon $\V$.

This device is adopted to ensure that all semantic objects are finite (by
controlling the unrolling of recursion).  The operator $\Rec$ is invoked in
just two places in the semantic rules: in the rule for recursive value
bindings of the form ``$\REC\ \valbind$'', and in the rule for evaluating
an application expression ``$\exp\ atexp$'' in the case that $\exp$
evaluates to a closure.

\subsection{Inference Rules}
\label{dyncor-inf-rules-sec}
The\index{50.2} semantic rules allow sentences  of the form
\[ \s,A\ts\phrase\ra A',\s' \]
to be inferred, where $A$ is usually an environment, $A'$ is some semantic
object and $\s$,$\s'$ are the states before and after the evaluation
represented by the sentence.  Some hypotheses in rules are not of this form;
they are called {\sl side-conditions}.  The convention for options is
the same as for the Core static semantics.  

In most rules the states $\s$ and $\s'$ are omitted from sentences; they
are only included for those rules which are directly concerned with the state
-- either referring to its contents or changing it.  When omitted, the
convention for restoring them is as follows.  If the rule is presented in the
form
\[ \frac{ \begin{array}{r}
          A_1\ts\phrase_1\ra A_1'\qquad
          A_2\ts\phrase_2\ra A_2'\quad\cdots\\
          \cdots\quad A_n\ts\phrase_n\ra A_n'
          \end{array}
        }
        { A\ts\phrase\ra A'} \]
then the full form is intended to be
\[ \frac{ \begin{array}{r}
          \s_0,A_1\ts\phrase_1\ra A_1',\s_1\qquad
          \s_1,A_2\ts\phrase_2\ra A_2',\s_2\quad\cdots\\
          \cdots\quad\s_{n-1},A_n\ts\phrase_n\ra A_n',\s_n
          \end{array}
        }
        { \s_0,A\ts\phrase\ra A',\s_n} \]
(Any side-conditions are left unaltered).
Thus the left-to-right order of the hypotheses indicates the order of
evaluation.  Note that in the case $\n=0$, when there are no hypotheses
(except possibly side-conditions), we have $\s_n=\s_0$; this implies that the
rule causes no side effect.
The convention is called the {\sl state convention},\index{51.1} and
must be applied to each version of a rule obtained by inclusion or
omission of its options.

A second convention, the {\sl exception convention}, is adopted to deal
with the propagation of exception packets $\p$.
For each rule whose full form (ignoring side-conditions) is
\[ \frac{ \s_1,A_1\ts\phrase_1\ra A_1',\s_1'\qquad\cdots\qquad
          \s_n,A_n\ts\phrase_n\ra A_n',\s_n' }
        { \s,A\ts\phrase\ra A',\s'} \]
and for each $k$, $1\leq k\leq n$, for which the result $A_k'$ is not a
packet $\p$, an extra rule is added of the form
\[ \frac{ \s_1,A_1\ts\phrase_1\ra A_1',\s_1'\qquad\cdots\qquad
          \s_k,A_k\ts\phrase_k\ra \p',\s' }
        { \s,A\ts\phrase\ra \p',\s'} \]
where $\p'$ does not occur in the original rule.\footnote{There is one
exception to the exception convention; no extra rule is added for
rule~\ref{handlexp-dyn-rule1} which deals with handlers, 
since a handler is the only
means by which propagation of an exception can be arrested.}
This indicates that evaluation of phrases in the hypothesis terminates with the
first whose result is a packet (other than one already treated in the rule),
and this packet is the result of the phrase in the conclusion.

A third convention is that we allow compound variables (variables built
from the variables in Figure~\ref{comp-dyn-obj} and the symbol ``/'')
to range over unions of semantic objects. For instance 
the compound variable $\V/\p$ ranges
over $\Val\cup\Pack$. 
We also allow $x/\FAIL$ to range over $X\cup\{\FAIL\}$ where $x$ 
ranges over $X$;
furthermore, we extend environment modification to allow for failure
as follows:
\[\VE+\FAIL=\FAIL.\]
%
%                       Atomic Expressions
%
\rulesec{Atomic Expressions}{\E\ts\atexp\ra\V/\p}
\begin{equation}	% special constant
%\label{sconexp-dyn-rule}
\frac{}
     {\E\ts\scon\ra\sconval(\scon)}\index{51.15}
\end{equation}

\begin{equation}	% value variable
%\label{varexp-dyn-rule}
\frac{\E(\longvar)=\V}
     {\E\ts\longvar\ra\V}\index{51.2}
\end{equation}

\begin{equation}	% value constructor
\label{conexp-dyn-rule}
\frac{\longcon=\strid_1.\cdots.\strid_k.\con}
     {\E\ts\longcon\ra\con}\index{52.1}
\end{equation}

\begin{equation}       %  exception constant
\label{exconexp-dyn-rule}
\frac{\E(\longexn)=\e}
     {\E\ts\longexn\ra\e}
\end{equation}


\begin{equation}	% record expression
%\label{recexp-dyn-rule}
\frac{\langle\E\ts\labexps\ra\r\rangle}
     {\E\ts\lttbrace\ \recexp\ \rttbrace\ra\emptymap
                                  \langle +\ \r\rangle\ \In\ \Val}
\end{equation}

\begin{equation}        % local declaration
%\label{let-dyn-rule}
\frac{\E\ts\dec\ra\E'\qquad\E+\E'\ts\exp\ra\V}
     {\E\ts\letexp\ra\V}
\end{equation}

\begin{equation}	% paren expression
%\label{parexp-dyn-rule}
\frac{\E\ts\exp\ra\V}
     {\E\ts\parexp\ra\V}
\end{equation}
\comments
\begin{description}
\item{(\ref{conexp-dyn-rule})}
   Value constructors denote themselves.
\item{(\ref{exconexp-dyn-rule})}
   Exception constructors are looked up in the exception environment
   component of $\E$.
\end{description}

\rulesec{Expression Rows}{\E\ts\labexps\ra\r/\p}
\begin{equation}	% labelled expressions
%\label{labexps-dyn-rule}
\frac{\E\ts\exp\ra\V\qquad\langle\E\ts\labexps\ra\r\rangle}
     {\E\ts\longlabexps\ra\{\lab\mapsto\V\}\langle +\ \r\rangle}\index{52.2}
\end{equation}
\comment We may think of components as being evaluated from left to right,
because of the state and exception conventions.
%
%                        Expressions
%
\rulesec{Expressions}{\E\ts\exp\ra\V/\p}
\begin{equation}	% atomic
%\label{atexp-dyn-rule}
\frac{\E\ts\atexp\ra\V}
     {\E\ts\atexp\ra\V}\index{52.3}
\end{equation}

\begin{equation}	% constructor application
\label{conapp-dyn-rule}
\frac{\E\ts\exp\ra\con\qquad\con\neq\REF\qquad\E\ts\atexp\ra\V}
     {\E\ts\appexp\ra(\con,\V)}
\end{equation}

\begin{equation}        % exception constructor application
\frac{\E\ts\exp\ra\e\qquad\E\ts\atexp\ra\V}
     {\E\ts\appexp\ra(\e,\V)}
\end{equation}

\begin{equation}	% reference application
\label{refapp-dyn-rule}
\frac{\s,\E\ts\exp\ra~\ml{ref}~,\s'\qquad
      \s',\E\ts\atexp\ra\V,\s''\qquad
      \A\notin\Dom(\of{\mem}{\s''})}
     {\s,\E\ts\appexp\ra\A,\ \s''+\{\A\mapsto\V\} }
\end{equation}

%%%%%%%%%%%%%%%%%%%%%%%%%%%%%%%%%%%%%%%%%%%%%%%%%%%%%%%%%%%%%%%%%%%%%%
%\begin{equation}	% contents application
%\label{contapp-dyn-rule}
%\frac{\s,\E\ts\exp\ra~\mbox{\tt !}~,\s'\qquad\s',\E\ts\atexp\ra\A,\s''
%      \qquad\s''(\A)=\V}
%     {\s,\E\ts\appexp\ra\V,\s''}
%\end{equation}
%%%%%%%%%%%%%%%%%%%%%%%%%%%%%%%%%%%%%%%%%%%%%%%%%%%%%%%%%%%%%%%%%%%%%%

\begin{equation}	% assignment application
\label{assapp-dyn-rule}
\frac{\s,\E\ts\exp\ra~\mbox{\tt :=}~,\s'\qquad
      \s',\E\ts\atexp\ra\{{\tt 1}\mapsto\A,\ {\tt 2}\mapsto\V\},\s''}
     {\s,\E\ts\appexp\ra\emptymap\ \In\ \Val,\ \s''+\{\A\mapsto\V\} }
\end{equation}

\begin{equation}	% basic function application
%\label{basapp-dyn-rule}
\frac{\E\ts\exp\ra b
      \qquad\E\ts\atexp\ra\V\qquad\APPLY(b,\V)=\V'}
     {\E\ts\appexp\ra\V'}\index{53.1}
\end{equation}

\begin{equation}	% closure application
\label{closapp-dyn-rule}
\frac{\begin{array}{c}
      \E\ts\exp\ra(\match,\E',\VE)\qquad\E\ts\atexp\ra\V\\
      \E'+\Rec\VE,\ \V\ts\match\ra\V'
      \end{array}
     }
     {\E\ts\appexp\ra\V'}
\end{equation}

\begin{equation}        % failing closure application
\label{closapp-dyn-rule1}
\frac{\begin{array}{c}
      \E\ts\exp\ra(\match,\E',\VE)\qquad\E\ts\atexp\ra\V\\
      \E'+\Rec\VE,\ \V\ts\match\ra\FAIL
      \end{array}
     }
     {\E\ts\appexp\ra[{\tt Match}]}
\end{equation}



\begin{equation}        % handle exception 1
\label{handlexp-dyn-rule1}
\frac{\E\ts\exp\ra\V}
     {\E\ts\handlexp\ra\V}
\end{equation}

\begin{equation}        % handle exception 2
\label{handlexp-dyn-rule2}
\frac{\E\ts\exp\ra[\exval]\qquad\E,\exval\ts\match\ra\V}
     {\E\ts\handlexp\ra\V}
\end{equation}

\begin{equation}        % handle exception 3
\label{handlexp-dyn-rule3}
\frac{\E\ts\exp\ra[\exval]\qquad\E,\exval\ts\match\ra\FAIL}
     {\E\ts\handlexp\ra[\exval]}
\end{equation}

\begin{equation}        % raise exception
%\label{raisexp-dyn-rule}
\frac{\E\ts\exp\ra\exval}
     {\E\ts\raisexp\ra[\exval]}
\end{equation}

\begin{equation}        % function
\label{fnexp-dyn-rule}
\frac{}
     {\E\ts\fnexp\ra(\match,\E,\emptymap)}
\end{equation}
\comments
\begin{description}
\item{(\ref{refapp-dyn-rule})}
  The side condition ensures that a new address is chosen. There are
no rules concerning disposal of inaccessible addresses (``garbage
collection'').
%
\item{(\ref{conapp-dyn-rule})--(\ref{closapp-dyn-rule1})}
  Note that none of the rules for function application has a
premise in which the operator evaluates to a constructed
value, a record or an address. This is because we are interested
in the evaluation of well-typed programs only, and in such programs $\exp$
will always have a functional type.
% so $\V$ will be either a closure,
%a constructor, a basic value or \ml{:=}.
%
\item{(\ref{handlexp-dyn-rule1})}
  This is the only rule to which the exception convention does not apply.
If the operator evaluates to a packet then rule~\ref{handlexp-dyn-rule2}
or rule~\ref{handlexp-dyn-rule3} must be used.
%
\item{(\ref{handlexp-dyn-rule3})}
 Packets that are not handled by the $\match$ propagate.
%
\item{(\ref{fnexp-dyn-rule})}
  The third component of the closure is empty because the match does not
introduce new recursively defined values.
\end{description}
%
%                        Matches
%
\rulesec{Matches}{\E,\V\ts\match\ra\V'/\p/\FAIL}
\begin{equation}	% match 1
%\label{match-dyn-rule1}
\frac{\E,\V\ts\mrule\ra\V'}
     {\E,\V\ts\longmatch\ra\V'}\index{54.1}
\end{equation}

\begin{equation}	% match 2
%\label{match-dyn-rule2}
\frac{\E,\V\ts\mrule\ra\FAIL}
     {\E,\V\ts\mrule\ra\FAIL}
\end{equation}

\begin{equation}	% match 3
%\label{match-dyn-rule}
\frac{\E,\V\ts\mrule\ra\FAIL\qquad\E,\V\ts\match\ra\V'/\FAIL}
     {\E\ts\longmatcha\ra\V'/\FAIL}
\end{equation}
\comment A value $\V$ occurs on the left of the turnstile, in evaluating
a $\match$. We may think of a $\match$ as being evaluated {\sl against}
a value; similarly, we may think of a pattern as being evaluated {\sl
against} a value.
Alternative match rules are tried from left to right.

\rulesec{Match Rules}{\E,\V\ts\mrule\ra\V'/\p/\fail}
\begin{equation}	% mrule 1
%\label{mrule-dyn-rule1}
\frac{\E,\V\ts\pat\ra\VE\qquad\E+\VE\ts\exp\ra\V'}
     {\E,\V\ts\longmrule\ \ra\V'}\index{54.2}
\end{equation}

\begin{equation}	% mrule 2
%\label{mrule-dyn-rule2}
\frac{\E,\V\ts\pat\ra\FAIL}
     {\E,\V\ts\longmrule\ \ra\FAIL}
\end{equation}

%
%                        Declarations
%
\rulesec{Declarations}{\E\ts\dec\ra\E'/\p}
\begin{equation}	% value declaration
%\label{valdec-dyn-rule}
\frac{\E\ts\valbind\ra\VE}
     {\E\ts\valdec\ra\VE\ \In\ \Env}\index{54.3}
\end{equation}

\begin{equation}	% exception declaration
%\label{exceptiondec-dyn-rule}
\frac{\E\ts\exnbind\ra\EE }
     {\E\ts\exceptiondec\ra\EE\ \In\ \Env }
\end{equation}

\begin{equation}	% local declaration
%\label{localdec-dyn-rule}
\frac{\E\ts\dec_1\ra\E_1\qquad\E+\E_1\ts\dec_2\ra\E_2}
     {\E\ts\localdec\ra\E_2}
\end{equation}

\begin{equation}                % open declaration
%\label{open-strdec-dyn-rule}
\frac{ \E(\longstrid_1)=\E_1
            \quad\cdots\quad
       \E(\longstrid_k)=\E_k }
     { \E\ts\openstrdec\ra \E_1 + \cdots + \E_k }
\end{equation}

\vspace{6pt}
\begin{equation}	% empty declaration
%\label{emptydec-dyn-rule}
\frac{}
     {\E\ts\emptydec\ra\emptymap\ \In\ \Env}
\end{equation}

\begin{equation}	% sequential declaration
%\label{seqdec-dyn-rule}
\frac{\E\ts\dec_1\ra\E_1\qquad\E+\E_1\ts\dec_2\ra\E_2}
     {\E\ts\seqdec\ra\plusmap{E_1}{E_2}}
\end{equation}
%
%                        Bindings
%
\rulesec{Value Bindings}{\E\ts\valbind\ra\VE/\p}
\begin{equation}	% value binding 1
%\label{valbind-dyn-rule1}
\frac{\E\ts\exp\ra\V\qquad\E,\V\ts\pat\ra\VE\qquad
      \langle\E\ts\valbind\ra\VE'\rangle }
     {\E\ts\longvalbind\ra\VE\ \langle +\ \VE'\rangle}\index{55.1}
\end{equation}

\begin{equation}	% value binding 2
%\label{valbind-dyn-rule2}
\frac{\E\ts\exp\ra\V\qquad\E,\V\ts\pat\ra\FAIL}
     {\E\ts\longvalbind\ra[\ml{Bind}]}
\end{equation}

\begin{equation}	% recursive value binding
%\label{recvalbind-dyn-rule}
\frac{\E\ts\valbind\ra\VE}
     {\E\ts\recvalbind\ra\Rec\VE}
\end{equation}

\rulesec{Exception Bindings}{\E\ts\exnbind\ra\EE/\p}
\begin{equation}	% exception binding 1
\label{exnbind-dyn-rule1}
\frac{\e\notin\of{\excs}{\s}\qquad\s'=\s+\{\e\}\qquad
      \langle\s',\E\ts\exnbind\ra\EE,\s''\rangle }
     {\s,\E\ts\longexnbindaa\ra\{\exn\mapsto\e\}\langle +\ \EE\rangle,\
                               \s'\langle'\rangle}\index{55.2}
\end{equation}

\begin{equation}	% exception binding 2
%\label{exnbind-dyn-rule2}
\frac{\E(\longexn)=\e\qquad
      \langle\E\ts\exnbind\ra\EE\rangle }
     {\E\ts\longexnbindb\ra\{\exn\mapsto\e\}\langle +\ \EE\rangle}
\end{equation}
\comments
\begin{description}
\item{(\ref{exnbind-dyn-rule1})}
  The two side conditions ensure that a new exception name is generated and 
recorded as ``used'' in subsequent states.
\end{description}
%
%                        Atomic Patterns
%
\rulesec{Atomic Patterns}{\E,\V\ts\atpat\ra\VE/\fail}
\begin{equation}	% wildcard pattern
%\label{wildcard-dyn-rule}
\frac{}
     {\E,\V\ts\wildpat\ra \emptymap}\index{55.3}
\end{equation}

\begin{equation}	% special constant in pattern (1)
%\label{sconpat-dyn-rule1}
\frac{\V=\sconval(\scon)}
     {\E,\V\ts\scon\ra \emptymap}\index{55.35}
\end{equation}

\begin{equation}	% special constant in pattern (2)
\label{sconpat-dyn-rule2}
\frac{\V\neq\sconval(\scon)}
     {\E,\V\ts\scon\ra \FAIL}\index{55.36}
\end{equation}

\begin{equation}	% variable pattern
%\label{varpat-dyn-rule}
\frac{}
     {\E,\V\ts\var\ra \{\var\mapsto\V\} }
\end{equation}

\begin{equation}	% constant pattern
%\label{conapat-dyn-rule1}
\frac{\longcon=\strid_1.\cdots.\strid_k.\con\qquad\V=\con }
     {\E,\V\ts\longcon\ra \emptymap}
\end{equation}

\begin{equation}
\label{conapat-dyn-rule2}
\frac{\longcon=\strid_1.\cdots.\strid_k.\con\qquad\V\neq\con}
     {\E,\V\ts\longcon\ra\FAIL}
\end{equation}
%\begin{equation}	% constant pattern
%\label{conpat-dyn-rule}
%\frac{\longcon=\strid_1.\cdots.\strid_k.\con }
%     {\con\ts\longcon\ra \emptymap}
%\end{equation}

\begin{equation}        % exception constant
%\label{exconapat-dyn-rule1}
\frac{\E(\longexn)=\V}
     {\E,\V\ts\longexn\ra\emptymap}
\end{equation}

\begin{equation}	
\label{exconapat-dyn-rule2}
\frac{\E(\longexn)\neq\V}
     {\E,\V\ts\longexn\ra\FAIL}\index{56.0}
\end{equation}


\begin{equation}	% record pattern
%\label{recpat-dyn-rule}
\frac{\V=\emptymap\langle+\r\rangle\ \In\ \Val\qquad
      \langle\E,\r\ts\labpats\ra\VE/\fail\rangle}
     {\E,\V\ts\lttbrace\ \langle\labpats\rangle\ \rttbrace\ra\emptymap\langle+\VE/\fail\rangle}
\end{equation}
%\begin{equation}	% record pattern
%\label{recpat-dyn-rule}
%\frac{\langle\r\ts\labpats\ra\VE\rangle}
%     {\emptymap\langle +\ \r\rangle\ \In\ \Val
%      \ts\{\ \recpat\ \}\ra\emptymap\langle +\ \VE\rangle}
%\end{equation}

\begin{equation}	% parenthesised pattern
%\label{parpat-dyn-rule}
\frac{\E,\V\ts\pat\ra\VE/\fail}
     {\E,\V\ts\parpat\ra\VE/\fail}\index{56.1}
\end{equation}

%\begin{equation}	% failure of atomic pattern
%\label{failatpat-dyn-rule}
%\frac{\forall\VE\ (\V\ts\atpat\not\Rightarrow\VE)}
%     {\V\ts\atpat\ra\FAIL}
%\end{equation}
\comments
\begin{description}
\item{(\ref{sconpat-dyn-rule2}),(\ref{conapat-dyn-rule2}),(\ref{exconapat-dyn-rule2})}
  Any evaluation resulting in $\FAIL$ must do so because 
rule~\ref{sconpat-dyn-rule2},
rule~\ref{conapat-dyn-rule2},
rule~\ref{exconapat-dyn-rule2},
rule~\ref{conpat-dyn-rule2},
or rule~\ref{exconpat-dyn-rule2} has been
applied.
\end{description}

\rulesec{Pattern Rows}{\E,\r\ts\labpats\ra\VE/\fail}
\begin{equation}	% wildcard record
%\label{wildrec-dyn-rule}
\frac{}
     {\E,\r\ts\wildrec\ra\emptymap}\index{56.2}
\end{equation}
 

\begin{equation}	% record component with inherited FAIL
\label{longlab-dyn-rule1}
\frac{\E,\r(\lab)\ts\pat\ra\FAIL}
     {\E,\r\ts\longlabpats\ra\FAIL}
\end{equation}

\begin{equation}	% record component
\label{longlab-dyn-rule2}
\frac{\E,\r(\lab)\ts\pat\ra\VE\qquad
      \langle\E,\r\ts\labpats\ra\VE'/\fail\rangle }
     {\E,\r\ts\longlabpats\ra
      \VE\langle +\ \VE'/\fail\rangle}
\end{equation}
\comments
\begin{description}
\item{(\ref{longlab-dyn-rule1}),(\ref{longlab-dyn-rule2})}
For well-typed programs $\lab$ will be in the domain of $\r$.
\end{description}
%\begin{equation}	% record component
%\label{longlab-dyn-rule}
%\frac{\V\ts\pat\ra\VE\qquad
%      \langle\r\ts\labpats\ra\VE'\qquad\VE\sim\VE'\rangle }
%     {\{\lab\mapsto\V\}\langle +\r\rangle\ts\longlabpats\ra
%      \VE\langle +\ \VE'\rangle}
%\end{equation}

%\begin{equation}	% failure of labelled patterns
%\label{faillabpats-dyn-rule}
%\frac{\forall\VE\ (\r\ts\labpats\not\ra\VE)}
%     {\r\ts\labpats\ra\FAIL}
%\end{equation}
%
%                        Patterns
%

\rulesec{Patterns}{\E,\V\ts\pat\ra\VE/\fail}
\begin{equation}	% atomic pattern
%\label{atpat-dyn-rule}
\frac{\E,\V\ts\atpat\ra \VE/\fail}
     {\E,\V\ts\atpat\ra \VE/\fail}\index{56.3}
\end{equation}

%\begin{equation}	% atomic pattern
%%\label{atpat-dyn-rule}
%\frac{\V\ts\atpat\ra \VE}
%     {\V\ts\atpat\ra \VE}
%\end{equation}

\begin{equation}	% construction pattern
%\label{conpat-dyn-rule1}
\frac{\begin{array}{c}
       \longcon=\strid_1.\cdots.\strid_k.\con\neq\REF\qquad
      \V=(\con,\V')\\
      \E,\V'\ts\atpat\ra\VE/\fail
      \end{array}}
     {\E,\V\ts\conpat\ra \VE/\fail}
\end{equation}

\begin{equation}	% construction pattern
\label{conpat-dyn-rule2}
\frac{\longcon=\strid_1.\cdots.\strid_k.\con\neq\REF\qquad
      \V\notin\{\con\}\times\Val}
     {\E,\V\ts\conpat\ra \FAIL}
\end{equation}

%\begin{equation}	% construction pattern
%\label{conpat-dyn-rule}
%\frac{\longcon=\strid_1.\cdots.\strid_k.\con\neq\REF\qquad\V\ts\atpat\ra\VE}
%     {(\con,\V)\ts\conpat\ra \VE}
%\end{equation}

\begin{equation}        % exception construction
%\label{exconpat-dyn-rule1}
\frac{\begin{array}{c}
      \E(\longexn)=\e\qquad\V=(\e,\V')\\
      \E,\V'\ts\atpat\ra\VE/\FAIL
      \end{array}
     }
     {\E,\V\ts\exconpat\ra\VE/\FAIL}
\end{equation}

\begin{equation} 
\label{exconpat-dyn-rule2}
\frac{\E(\longexn)=\e\qquad\V\notin\{\e\}\times\Val}
     {\E,\V\ts\exconpat\ra\FAIL}
\end{equation}

\begin{equation}	% reference pattern
%\label{refpat-dyn-rule}
\frac{\s(\A)=\V\qquad\s,\E,\V\ts\atpat\ra\VE/\fail,\s}
     {\s,\E,\A\ts\REF\ \atpat\ra \VE/\fail,\s}\index{57.0}
\end{equation}

%\begin{equation}	% reference pattern
%\label{refpat-dyn-rule}
%\frac{\s(\A)=\V\qquad\s,\V\ts\atpat\ra\VE,\s}
%     {\s,\A\ts\REF\ \atpat\ra \VE,\s}
%\end{equation}

\begin{equation}	% layered pattern
%\label{layeredpat-dyn-rule}
\frac{\E,\V\ts\pat\ra\VE/\fail}
     {\E,\V\ts\layeredpata\ra\{\var\mapsto\V\}+\VE/\fail}
\end{equation}
%
%\begin{equation}	% layered pattern
%\label{layeredpat-dyn-rule}
%\frac{\V\ts\pat\ra\VE\qquad\{\var\mapsto\V\}\sim\VE}
%     {\V\ts\layeredpat\ra\VE}
%\end{equation}
%
%\begin{equation}	% failure of pattern
%\label{failpat-dyn-rule}
%\frac{\forall\VE\ (\V\ts\pat\not\ra\VE)}
%     {\V\ts\pat\ra\FAIL}
%\end{equation}
\comments
\begin{description}
\item{(\ref{conpat-dyn-rule2}),(\ref{exconpat-dyn-rule2})}
  Any evaluation resulting in $\FAIL$ must do so because 
rule~\ref{sconpat-dyn-rule2},
rule~\ref{conapat-dyn-rule2},
rule~\ref{exconapat-dyn-rule2},
rule~\ref{conpat-dyn-rule2},
or rule~\ref{exconpat-dyn-rule2} has been
applied.\index{57.1}
\end{description}

\section{Dynamic Semantics for Modules}
\label{dynmod-sec}
\subsection{Reduced Syntax}
Since\index{58.1} signature expressions
are mostly dealt with in the static semantics,
the dynamic semantics need only take limited account of them.  Unlike types,
it cannot ignore them completely; the reason is that an explicit signature
ascription plays the role of restricting the ``view'' of a structure - that is,
restricting the domains of its component environments.  However, the types
and the sharing properties of structures and signatures are irrelevant to
dynamic evaluation; the syntax is therefore
reduced by the following transformations (in addition to those for the Core),
for the purpose of the dynamic semantics of Modules:
\begin{itemize}
\item Qualifications ``$\OF\ \ty$'' are omitted from exception descriptions.
\item Any specification of the form ``$\typespec$'', ``$\eqtypespec$'',
``$\DATATYPE$\ $\datdesc$'' or
``$\sharingspec$'' is replaced by the empty specification.
\item The Modules phrase classes TypDesc, DatDesc, ConDesc and SharEq
      are omitted.
%version 2:\item The Modules phrase classes $\typdesc$, $\datdesc$, $condesc$ and $\shareq$
%      are omitted.
\end{itemize}

\subsection{Compound Objects}
\label{dynmod-comp-obj-sec}
The\index{58.2} compound objects for the Modules dynamic semantics, extra to those for the
Core dynamic semantics, are shown in Figure~\ref{comp-dynmod-obj}.
\begin{figure}[h]
\vspace{2pt}
\begin{displaymath}
\begin{array}{rcl}
(\strid:\I,\strexp\langle:\I'\rangle,\B)
                & \in   & \FunctorClosure\\
                &       & \qquad  = (\StrId\times\Int)\times
                          (\StrExp\langle\times\Int\rangle)\times\Basis\\
(\IE,\vars,\exns)\ {\rm or}\ \I
                & \in   & \Int = \IntEnv\times\Fin(\Var)\times\Fin(\Exn)\\
        \IE     & \in   & \IntEnv = \finfun{\StrId}{\Int}\\
        \G 	& \in	& \SigEnv = \finfun{\SigId}{\Int}\\
        \F	& \in	& \FunEnv = \finfun{\FunId}{\FunctorClosure}\\
(\F,\G,\E)\ {\rm or}\ \B
                & \in   & \Basis = \FunEnv\times\SigEnv\times\Env\\
(\G,\IE)\ {\rm or}\ \IB
                & \in   & \IntBasis = \SigEnv\times\IntEnv
\end{array}
\end{displaymath}
\caption{Compound Semantic Objects}
\label{comp-dynmod-obj}
\vspace{3pt}
\end{figure}
%
%
An {\sl interface} $\I\in\Int$ represents a ``view'' of a structure.
Specifications and signature expressions will evaluate to interfaces; 
moreover, during the evaluation of a specification or signature expression, 
structures (to which a specification or signature expression may
refer via ``$\OPEN$'') are represented only by their interfaces.  To extract an
interface from a dynamic environment we define the operation
\[ \Inter\ :\ \Env\to\Int \]
as follows:
\[ \Inter(\SE,\VE,\EE)\ =\ (\IE,\Dom\VE,\Dom\EE)\]
where
\[ \IE\ =\ \{\strid\mapsto\Inter\E\ ;\ \SE(\strid)=\E\}\ .\]
An {\sl interface basis}\index{59.1} $\IB=(\G,\IE)$ is that part of a basis needed to
evaluate signature expressions and specifications.
The function $\Inter$ is extended to create an interface basis
from a basis $\B$ as follows:
\[ \Inter(\F,\G,\E)\ =\ (\G, \of{\IE}{(\Inter\E)}) \]

A further operation
\[ \downarrow\ :\ \Env\times\Int\to\Env\]
is required, to cut down an environment $\E$ to a given interface $\I$,
representing the effect of an explicit signature ascription.  It is defined
as follows:
\[ (\SE,\VE,\EE)\downarrow(\IE,\vars,\exns)\ =\ (\SE',\VE',\EE') \]
where
\[ \SE'\ =\ \{\strid\mapsto\E\downarrow\I\ ;\
          \SE(\strid)=\E\ {\rm and}\ \IE(\strid)=\I\} \]
and (taking $\downarrow$ now to mean restriction of a function domain)
\[\VE'=\VE\downarrow\vars,\ \EE'=\EE\downarrow\exns.\]

It is important to note that an interface is also a projection of the
{\sl static} value $\Sigma$ of a signature expression; 
it is obtained by omitting structure names $\m$ and type environments
$\TE$, and replacing each variable environment $\VE$ and each 
exception environment $\EE$ by its domain.
%it is obtained by omitting the
%structure names $\m$, type functions $\theta$ and type environments $\TE$.
Thus in an implementation interfaces would naturally be obtained from the
static elaboration; we choose to give separate rules here for obtaining them
in the dynamic semantics since we wish to maintain our separation of the
static and dynamic semantics, for reasons of presentation.

\subsection{Inference Rules}
The\index{59.2} semantic rules allow sentences  of the form
\[ \s,A\ts\phrase\ra A',\s' \]
to be inferred, where $A$ is either a basis or an interface basis or empty,
$A'$ is some semantic
object and $\s$,$\s'$ are the states before and after the evaluation
represented by the sentence.  Some hypotheses in rules are not of this form;
they are called {\sl side-conditions}.  The convention for options is
the same as for the Core static semantics.  

The state and exception conventions are adopted as in the Core dynamic
semantics.  However, it may be shown that the only Modules phrases whose 
evaluation
may cause a side-effect or generate an exception packet are of the form
$\strexp$, $\strdec$, $\strbind$ or $\topdec$.
%Also, as will be seen in Section~\ref{prog-sec}, a phrase of the
%form $\program$ can have side-effects, but not generate an
%exception packet.

%		SEMANTICS
%
%                       Structure Expressions
%
\rulesec{Structure Expressions}{\B\ts\strexp\ra \E/\p}
\begin{equation}	% generative strexp
%\label{generative-strexp-dyn-rule}
\frac{\B\ts\strdec\ra\E}
     {\B\ts\encstrexp\ra\E}\index{60.1}
\end{equation}

\begin{equation}	% longstrid
%\label{longstrid-strexp-dyn-rule}
\frac{\B(\longstrid)=\E}
     {\B\ts\longstrid\ra\E}
\end{equation}

\vspace{6pt}
\begin{equation}		% functor application
\label{functor-application-dyn-rule}
\frac{ \begin{array}{c}
        \B(\funid)=(\strid:\I,\strexp'\langle:\I'\rangle,\B')\\
        \B\ts\strexp\ra\E\qquad
       \B'+\{\strid\mapsto\E\downarrow\I\}\ts\strexp'\ra\E'\\
       \end{array}
     }
     {\B\ts\funappstr\ra\E'\langle\downarrow\I'\rangle}
\end{equation}

\vspace{6pt}
\begin{equation}	% let strexp
%\label{letstrexp-dyn-rule}
\frac{\B\ts\strdec\ra\E\qquad\B+\E\ts\strexp\ra\E'}
     {\B\ts\letstrexp\ra\E'}
\end{equation}
\comments
\begin{description}
\item{(\ref{functor-application-dyn-rule})}
Before the evaluation of the functor body $\strexp'$, the
actual argument $\E$ is cut down by the formal parameter
interface $\I$, so that any opening of $\strid$ resulting
from the evaluation of $\strexp'$ will produce no more components
than anticipated during the static elaboration.
\end{description}

\rulesec{Structure-level Declarations}{\B\ts\strdec\ra\E/\p}
   		% declarations
\begin{equation}                % core declaration
%\label{dec-dyn-rule}
\frac{ \of{\E}{\B}\ts\dec\ra\E' }
     { \B\ts\dec\ra\E' }\index{60.2}
\end{equation}

\vspace{6pt}
\begin{equation}        	% structure declaration
%\label{structure-decl-dyn-rule}
\frac{ \B\ts\strbind\ra\SE }
     { \B\ts\singstrdec\ra\SE\ \In\ \Env }
\end{equation}

\vspace{6pt}
\begin{equation}                % local structure-level declaration
%\label{local structure-level declaration-dyn-rule}
\frac{ \B\ts\strdec_1\ra\E_1\qquad
       \B+\E_1\ts\strdec_2\ra\E_2 }
     { \B\ts\localstrdec\ra\E_2 }
\end{equation}

\vspace{6pt}
\begin{equation}                % empty declaration
%\label{empty-strdec-dyn-rule}
\frac{}
     {\B\ts\emptystrdec\ra \emptymap{\rm\ in}\ \Env}
\end{equation}

\vspace{6pt}
\begin{equation}		% sequential declaration
%\label{sequential-strdec-dyn-rule}
\frac{ \B\ts\strdec_1\ra\E_1\qquad
       \B+\E_1\ts\strdec_2\ra\E_2 }
     { \B\ts\seqstrdec\ra\plusmap{\E_1}{\E_2} }
\end{equation}

\rulesec{Structure Bindings}{\B\ts\strbind\ra\SE/\p}
\begin{equation}                % structure binding
\frac{ \begin{array}{cl}
       \B\ts\strexp\ra\E\qquad\langle\Inter\B\ts\sigexp\ra\I\rangle\\
       \langle\langle\B\ts\strbind\ra\SE\rangle\rangle
       \end{array}
     }
     {\begin{array}{c}
      \B\ts\strbinder\ra\\
      \qquad\qquad\qquad\{\strid\mapsto\E\langle\downarrow\I\rangle\}
      \ \langle\langle +\ \SE\rangle\rangle
      \end{array}
     }\index{61.1}
\end{equation}
\comment As in the static semantics, when present, $\sigexp$ constrains the
``view'' of the structure. The restriction must be done in the
dynamic semantics to ensure that any dynamic opening of the structure
produces no more components than anticipated during the static
elaboration.
%
%                   Signature Rules
%

\rulesec{Signature Expressions}{\IB\ts\sigexp\ra\I}
\begin{equation}		% encapsulation sigexp
%\label{encapsulating-sigexp-dyn-rule}
\frac{\IB\ts\spec\ra\I }
     {\IB\ts\encsigexp\ra\I}\index{61.2}
\end{equation}

\begin{equation}		% signature identifier
%\label{signature-identifier-dyn-rule}
\frac{ \IB(\sigid)=\I}
     { \IB\ts\sigid\ra\I }
\end{equation}

\rulesec{Signature Declarations}{\IB\ts\sigdec\ra\G}
\begin{equation}	% single signature declaration
%\label{single-sigdec-dyn-rule}
\frac{ \IB\ts\sigbind\ra\G }
     { \IB\ts\singsigdec\ra\G }\index{61.3}
\end{equation}

\begin{equation}	% empty signature declaration
%\label{empty-sigdec-dyn-rule}
\frac{}
     { \IB\ts\emptysigdec\ra\emptymap }
\end{equation}

\begin{equation}	% sequential signature declaration
%\label{sequence-sigdec-dyn-rule}
\frac{ \IB\ts\sigdec_1\ra\G_1 \qquad \plusmap{\IB}{\G_1}\ts\sigdec_2\ra\G_2 }
     { \IB\ts\seqsigdec\ra\plusmap{\G_1}{\G_2} }
\end{equation}

\rulesec{Signature Bindings}{\IB\ts\sigbind\ra\G}

\begin{equation}	% signature binding
%\label{sigbind-dyn-rule}
\frac{ \IB\ts\sigexp\ra\I
        \qquad\langle\IB\ts\sigbind\ra\G\rangle }
     { \IB\ts\sigbinder\ra\{\sigid\mapsto\I\}
       \ \langle +\ \G\rangle }\index{61.4}
\end{equation}
%
                     % Specifications
%
\rulesec{Specifications}{\IB\ts\spec\ra\I}

\begin{equation}        % value specification
%\label{valspec-dyn-rule}
\frac{ \ts\valdesc\ra\vars }
     { \IB\ts\valspec\ra\vars\ \In\ \Int }\index{61.5}
\end{equation}

\begin{equation}        % exception specification
%\label{exceptionspec-dyn-rule}
\frac{ \ts\exndesc\ra\exns}
     { \IB\ts\exceptionspec\ra\exns\ \In\ \Int }
\end{equation}

\begin{equation}        % structure specification
\label{structurespec-dyn-rule}
\frac{ \IB\ts\strdesc\ra\IE }
     { \IB\ts\structurespec\ra\IE\ \In\ \Int }\index{62.1}
\end{equation}

\begin{equation}        % local specification
\label{localspec-dyn-rule}
\frac{ \IB\ts\spec_1\ra\I_1 \qquad
       \plusmap{\IB}{\of{\IE}{\I_1}}\ts\spec_2\ra\I_2 }
     { \IB\ts\localspec\ra\I_2 }
\end{equation}

\begin{equation}        % open specification
%\label{openspec-dyn-rule}
\frac{ \IB(\longstrid_1)=\I_1\quad\cdots\quad
       \IB(\longstrid_n)=\I_n }
     { \IB\ts\openspec\ra\I_1 + \cdots +\I_n }
\end{equation}

\begin{equation}        % include signature specification
%\label{inclspec-dyn-rule}
\frac{ \IB(\sigid_1)=\I_1 \quad\cdots\quad
       \IB(\sigid_n)=\I_n }
     { \IB\ts\inclspec\ra\I_1 + \cdots +\I_n }
\end{equation}

\begin{equation}        % empty specification
%\label{emptyspec-dyn-rule}
\frac{}
     { \IB\ts\emptyspec\ra\emptymap{\rm\ in}\ \Int }
\end{equation}

\begin{equation}        % sequential specification
\label{seqspec-dyn-rule}
\frac{ \IB\ts\spec_1\ra\I_1
       \qquad \plusmap{\IB}{\of{\IE}{\I_1}}\ts\spec_2\ra\I_2 }
     { \IB\ts\seqspec\ra\plusmap{\I_1}{\I_2} }
\end{equation}
\comments
\begin{description}
\item{(\ref{localspec-dyn-rule}),(\ref{seqspec-dyn-rule})}
  Note that $\of{\vars}{\I_1}$ and $\of{\exns}{\I_1}$ are
not needed for the evaluation of $\spec_2$.
\end{description}

                         % Descriptions

\rulesec{Value Descriptions}{\ts\valdesc\ra\vars}
\begin{equation}         % value description
%\label{valdesc-dyn-rule}
\frac{ \langle\ts\valdesc\ra\vars\rangle }
     { \ts\var\ \langle\AND\ \valdesc\rangle\ra
       \{\var\}\ \langle\cup\ \vars\rangle }\index{62.2}
\end{equation}

\rulesec{Exception Descriptions}{\ts\exndesc\ra\exns}
\begin{equation}         % exception description
%\label{exndesc-dyn-rule}
\frac{ \langle\ts\exndesc\ra\exns\rangle }
     { \ts\exn\ \langle\exndesc\rangle\ra\{\exn\}\ \langle\cup\ \exns\rangle }\index{62.3}
\end{equation}

\rulesec{Structure Descriptions}{\IB\ts\strdesc\ra\IE}
\begin{equation}
%\label{strdesc-dyn-rule}
\frac{ \IB\ts\sigexp\ra\I\qquad\langle\IB\ts\strdesc\ra\IE\rangle }
     { \IB\ts\strdescription\ra\{\strid\mapsto\I\}\ \langle +\ \IE\rangle }\index{62.4}
\end{equation}

%    			Functor and Program rules
%
\rulesec{Functor Bindings}{\B\ts\funbind\ra\F}
\begin{equation}	% functor binding
%\label{funbind-dyn-rule}
\frac{
      \begin{array}{c}
      \Inter\B\ts\sigexp\ra\I\qquad
      \langle\Inter\B+\{\strid\mapsto\I\} \ts\sigexp'\ra\I'\rangle \\
       \langle\langle\B\ts\funbind\ra\F\rangle\rangle
      \end{array}
     }
     {
      \begin{array}{c}
       \B\ts\funstrbinder\ \optfunbind\ra\\
       \qquad\qquad \qquad
              \{\funid\mapsto(\strid:\I,\strexp\langle:\I'\rangle,\B)\}
              \ \langle\langle +\ \F\rangle\rangle
      \end{array}
     }\index{62.5}
\end{equation}

\rulesec{Functor Declarations}{\B\ts\fundec\ra\F}
\begin{equation}        % single functor declaration
%\label{singfundec-dyn-rule}
\frac{ \B\ts\funbind\ra\F }
     { \B\ts\singfundec\ra\F }\index{63.1}
\end{equation}

\vspace{6pt}
\begin{equation}        % empty functor declaration
%\label{emptyfundec-dyn-rule}
\frac{}
     { \B\ts\emptyfundec\ra\emptymap }
\end{equation}

\vspace{6pt}
\begin{equation}        % sequential functor declaration
%\label{seqfundec-dyn-rule}
\frac{ \B\ts\fundec_1\ra\F_1\qquad
       \B+\F_1\ts\fundec_2\ra\F_2 }
     { \B\ts\seqfundec\ra\plusmap{\F_1}{\F_2} }
\end{equation}

\rulesec{Top-level Declarations}{\B\ts\topdec\ra\B'/\p}
\begin{equation}	% structure-level declaration
%\label{strdectopdec-dyn-rule}
\frac{\B\ts\strdec\ra\E}
     {\B\ts\strdec\ra\E\ \In\ \Basis
     }\index{63.2}
\end{equation}

\vspace{6pt}
\begin{equation}	% signature declaration
%\label{sigdectopdec-dyn-rule}
\frac{\Inter\B\ts\sigdec\ra\G}
     {\B\ts\sigdec\ra\G\ \In\ \Basis
     }
\end{equation}

\vspace{6pt}
\begin{equation}	% functor declaration
%\label{fundectopdec-dyn-rule}
\frac{\B\ts\fundec\ra\F}
     {\B\ts\fundec\ra\F\ \In\ \Basis
     }
\end{equation}

%from version 1
%\vspace{6pt}
%\begin{equation}	% sequential program
%\label{seqprog-dyn-rule}
%\frac{\B\ts\program_1\ra\B_1\qquad\
%      \plusmap{\B}{\B_1}\ts\program_2\ra\B_2
%     }
%     {\B\ts\seqprog\ra\plusmap{\B_1}{\B_2}}
%\end{equation}




\include{prog}
\appendix
\section{Appendix: Derived Forms}
\label{derived-forms-app}
Several derived\index{66.1} grammatical forms are provided in the Core; they are presented
in Figures~\ref{der-exp}, \ref{der-pat} and \ref{der-dec}. Each derived form is
given with its equivalent form. Thus, each row of the tables should be
considered as a rewriting rule
\[ \mbox{Derived form \ $\Longrightarrow$\  Equivalent form} \]
and these rules may be applied repeatedly to a phrase until it is transformed
into a phrase of the bare language.
See Appendix~\ref{core-gram-app} for the full Core grammar, including all the
derived forms.

In the derived forms for tuples, in terms of records, we use $\overline{n}$ to
mean the ML numeral which stands for the natural number $n$.

Note that a new phrase class ~FvalBind~ of function-value bindings is introduced,
accompanied by a new declaration form ~\FUN\ \fvalbind~. The mixed forms
~\VAL\ \REC\ \fvalbind~, ~\VAL\ \fvalbind~ and ~\FUN\ \valbind~ are not
allowed -- though the first form arises during translation into the bare
language.

The following notes refer to Figure~\ref{der-dec}:
\begin{itemize}
%\item      In the equivalent form for a function-value binding, the
%           variables ~$\var_1$, $\cdots$, $\var_n$~ must be chosen not to
%           occur in the derived form.  The condition $m,n\geq 1$ applies.
\item      There is a version of the derived form for function-value binding
	   which allows the function identifier to be infixed;
	   see Figure~\ref{dec-gram} in Appendix~\ref{core-gram-app}.
\item      In the two forms involving ~\WITHTYPE~, the identifiers bound
           by ~\datbind~ and by ~\typbind~ must be distinct. Then the
           transformed binding ~\datbind$\/'$~ in the equivalent form is
           obtained from ~\datbind~ by expanding out all the definitions
           made by ~\typbind.  More precisely, if ~\typbind~ is
           \[ \tyvarseq_1\ \tycon_1\ \mbox{\ml{=}} \ty_1\ \ \AND
              \ \cdots\ \AND
            \ \ \tyvarseq_n\ \tycon_n\ \mbox{\ml{=}} \ty_n\ \]
           then ~\datbind$\/'$~ is the result of simultaneous replacement
           (in ~\datbind~) of every type expression ~$\tyseq_i\ \tycon_i$~
           ($1\leq i\leq n$)
           by the corresponding defining expression
           \[  \ty_i\{\tyseq_i/\tyvarseq_i\}\]
%\item      The abbreviation of ~\VAL\ {\tt it =} \exp~ to ~\exp~ is only
%           permitted at top-level, i.e. as a ~\program~.
\end{itemize}

Figure~\ref{functor-der-forms-fig} shows derived forms for functors.
They allow functors to take, say, a single type or value as a parameter,
in cases where it would seem clumsy to ``wrap up'' the argument as a
structure expression.
These forms are currently more experimental than the bare syntax of modules, 
but we recommend implementers to include them so that they can be
tested in practice.
In the derived forms for functor bindings and functor signature expressions,
$\strid$ is a new structure identifier and
the form of $\sigexp'$ depends
on the form of $\sigexp$ as follows. 
If $\sigexp$ is simply a signature identifier
$\sigid$, then $\sigexp'$ is also $\sigid$; otherwise $\sigexp$ must take
the form  ~$\SIG\ \spec_1\ \END$~,
and then $\sigexp'$ is
$\mbox{\SIG\ \LOCAL\ \OPEN\ \strid\ \IN\ $\spec_1$\ \END\ \END}$.
%(where $\strid$ is new).

\begin{figure}

\begin{tabular}{|l|l|l}
\multicolumn{1}{c}{Derived Form} & \multicolumn{1}{c}{Equivalent Form} &
\multicolumn{1}{c}{}\\
\multicolumn{3}{c}{}\\
\multicolumn{2}{l}{{\bf Expressions} \exp}\\
%\multicolumn{2}{l}{EXPRESSIONS \exp}\\
\cline{1-2}
\ml{()}         & \ml{\lttbrace\ \rttbrace} \\
\cline{1-2}
\ml{(}$\exp_1$ \ml{,} $\cdots$ \ml{,} $\exp_\n$\ml{)}
            & \ml{\lttbrace 1=}$\exp_1$\ml{,}\ $\cdots$\ml{,}\
                             $\overline{n}$\ml{=}$\exp_\n$\ml{\rttbrace}
                                                           & $(\n\geq 2)$\\
\cline{1-2}
\ml{\#}\ \lab      & \FN\ \ml{\lttbrace}\lab\ml{=}\var\ml{,...\rttbrace\  => }\var
                                                           & (\var\ new)\\
%\cline{1-2}
%\RAISE\ \longexn    & \RAISE\ \longexn\ \WITH\ \ml{()} \\
\cline{1-2}
\CASE\ \exp\ \OF\ \match
                & \ml{(}\FN\ \match\ml{)(}\exp\ml{)} \\
\cline{1-2}
\IF\ $\exp_1$\ \THEN\ $\exp_2$\ \ELSE\ $\exp_3$
                & \CASE\ $\exp_1$\ \OF\ \TRUE\ \ml{=>}\ \exp$_2$\\
                & \ \ \qquad\qquad\ml{|}\ \FALSE\ \ml{=>}\ \exp$_3$ \\
\cline{1-2}
\exp$_1$\ \ORELSE\ \exp$_2$
                & \IF\ \exp$_1$\ \THEN\ \TRUE\ \ELSE\ \exp$_2$ \\
\cline{1-2}
\exp$_1$\ \ANDALSO\ \exp$_2$
                & \IF\ \exp$_1$\ \THEN\ \exp$_2$\ \ELSE\ \FALSE \\
\cline{1-2}
\ml{(}$\exp_1$ \ml{;} $\cdots$ \ml{;} $\exp_\n$ \ml{;} \exp\ml{)}\
                & \CASE\ \exp$_1$\ \OF\ \ml{(\_) =>}
                                                           & $(\n\geq 1)$ \\
                & \qquad$\cdots$ \\
                & \CASE\ \exp$_n$\ \OF\ \ml{(\_) =>}\ \exp \\
\cline{1-2}
\LET\ \dec\ \IN
                & \LET\ \dec\ \IN                          & $(\n\geq 2)$ \\
\qquad$\exp_1$ \ml{;} $\cdots$ \ml{;} $\exp_\n$ \END
                & \ \ \ml{(}$\exp_1$ \ml{;} $\cdots$ \ml{;} $\exp_\n$\ml{)}\
                                                                         \END\\
\cline{1-2}
\WHILE\ \exp$_1$\ \DO\ \exp$_2$
                & \LET\ \VAL\ \REC\ \var\ \ml{=}\ \FN\ \ml{() =>}
                                                           & (\var\ new)\\
                & \ \ \IF\ \exp$_1$\ \THEN\
                    \ml{(}\exp$_2$\ml{;}\var\ml{())}\ \ELSE\ \ml{()} \\
                & \ \ \IN\ \var\ml{()}\ \END\\
\cline{1-2}
\ml{[}$\exp_1$ \ml{,} $\cdots$ \ml{,} $\exp_\n$\ml{]}
                & \exp$_1$\ \ml{::}\ $\cdots$\ \ml{::}\ \exp$_n$\
                            \ml{::}\ \NIL                 & $(n\geq 0)$ \\
\cline{1-2}
\multicolumn{3}{c}{}\\
\end{tabular}
\caption{Derived forms of Expressions\index{67.1}}
\label{der-exp}
\end{figure}

\begin{figure}

\begin{tabular}{|l|l|l}
\multicolumn{1}{c}{Derived Form} & \multicolumn{1}{c}{Equivalent Form} &
\multicolumn{1}{c}{}\\
\multicolumn{3}{c}{}\\
\multicolumn{2}{l}{{\bf Patterns} \pat}\\
%\multicolumn{2}{l}{PATTERNS \pat}\\
\cline{1-2}
\ml{()}         & \ml{\lttbrace\ \rttbrace} \\
\cline{1-2}
\ml{(}$\pat_1$ \ml{,} $\cdots$ \ml{,} $\pat_\n$\ml{)}
            & \ml{\lttbrace 1=}$\pat_1$\ml{,}\ $\cdots$ \ml{,}\
                             $\overline{n}$\ml{=}$\pat_\n$\ml{\rttbrace}
                                                           & $(\n\geq 2)$ \\
\cline{1-2}
\ml{[}$\pat_1$ \ml{,} $\cdots$ \ml{,} $\pat_\n$\ml{]}
                & \pat$_1$\ \ml{::}\ $\cdots$\ \ml{::}\ \pat$_n$\
                            \ml{::}\ \NIL                 & $(n\geq 0)$ \\
\cline{1-2}
\multicolumn{3}{c}{}\\
\multicolumn{2}{l}{{\bf Pattern Rows} \labpats}\\
%\multicolumn{2}{l}{PATTERN ROWS \labpats}\\
\cline{1-2}
\id$\langle$\ml{:}\ty$\rangle
    \ \langle\AS\ \pat\rangle
    \ \langle$\ml{,} \labpats$\rangle$
                & \id\ml{ = }\id$\langle$\ml{:}\ty$\rangle
                                 \ \langle\AS\ \pat\rangle
                                 \ \langle$\ml{,} \labpats$\rangle$ \\
\cline{1-2}
\multicolumn{3}{c}{}\\
\multicolumn{2}{l}{{\bf Type Expressions} \ty}\\
%\multicolumn{2}{l}{TYPE EXPRESSIONS \ty}\\
\cline{1-2}
$\ty_1$ \ml{*} $\cdots$ \ml{*} $\ty_\n$
            & \ml{\lttbrace 1:}$\ty_1$\ml{,}\ $\cdots$ \ml{,}\
                             $\overline{n}$\ml{:}$\ty_\n$\ml{\rttbrace}
                                                           & $(\n\geq 2)$ \\
\cline{1-2}
\multicolumn{3}{c}{}\\
\end{tabular}
\caption{Derived forms of Patterns and Type Expressions\index{67.2}}
\label{der-pat}
\end{figure}

\begin{figure}

\begin{tabular}{|l|l|}
\multicolumn{1}{c}{Derived Form} & \multicolumn{1}{c}{Equivalent Form}\\
\multicolumn{2}{c}{}\\
\multicolumn{2}{l}{{\bf Function-value Bindings} \fvalbind}\\
%\multicolumn{2}{l}{FUNCTION-VALUE BINDINGS \fvalbind}\\
\hline
               & $\langle\OP\rangle$\var\ \ml{=} \FN\ \var$_1$\ml{=>} $\cdots$
                              \FN\ \var$_n$\ml{=>} \\
               & \CASE\
                 \ml{(}\var$_1$\ml{,} $\cdots$ \ml{,} \var$_n$\ml{)} \OF \\
\ \ $\langle\OP\rangle\var\ \atpat_{11}\cdots\atpat_{1n}
                                              \langle$\ml{:}\ty$\rangle$
                                              \ml{=} \exp$_1$
               & \ \ \ml{(}\atpat$_{11}$\ml{,}$\cdots$\ml{,}\atpat$_{1n}$
                             \ml{)=>}\exp$_1\langle$\ml{:}\ty$\rangle$\\
\ml{|}$\langle\OP\rangle\var\ \atpat_{21}\cdots\atpat_{2n}
                                              \langle$\ml{:}\ty$\rangle$
                                              \ml{=} \exp$_2$
               & \ml{|(}\atpat$_{21}$\ml{,}$\cdots$\ml{,}\atpat$_{2n}$
                             \ml{)=>}\exp$_2\langle$\ml{:}\ty$\rangle$\\
\ml{|}\qquad$\cdots\qquad\cdots$
               & \ml{|}\qquad$\cdots\qquad\cdots$\\
\ml{|}$\langle\OP\rangle\var\ \atpat_{m1}\cdots\atpat_{mn}
                                              \langle$\ml{:}\ty$\rangle$
                                              \ml{=} \exp$_m$
               & \ml{|(}\atpat$_{m1}$\ml{,}$\cdots$\ml{,}\atpat$_{mn}$
                             \ml{)=>}\exp$_m\langle$\ml{:}\ty$\rangle$\\
\qquad\qquad\qquad$\langle\AND\ \fvalbind\rangle$
               & \qquad\qquad\qquad$\langle\AND\ \fvalbind\rangle$\\
\hline
\multicolumn{2}{r}{($m,n\geq1$; $\var_1,\cdots,\var_n$ distinct and new)}\\
\multicolumn{2}{c}{}\\
\multicolumn{2}{l}{{\bf Declarations} \dec}\\
%\multicolumn{2}{l}{DECLARATIONS \dec}\\
\hline
\FUN\ \fvalbind
               & \VAL\ \REC\ \fvalbind  \\
\hline
\DATATYPE\ \datbind\ \WITHTYPE\ \typbind
               & \DATATYPE\ \datbind$\/'$\ \ml{;}\ \TYPE\ \typbind \\
\hline
\ABSTYPE\ \datbind\ \WITHTYPE\ \typbind
               & \ABSTYPE\ \datbind$\/'$ \\
\qquad\qquad\WITH\ \dec\ \END
               & \qquad\WITH\ \TYPE\ \typbind\ \ml{;}\ \dec\ \END\\
\hline
\multicolumn{2}{r}{(see note in text concerning \datbind$\/'$)}\\
\multicolumn{2}{c}{}\\
\end{tabular}
\caption{Derived forms of Function-value Bindings and Declarations\index{68.1}}
\label{der-dec}
\end{figure}

%               Derived forms of functors:

\begin{figure}
\begin{tabular}{|l|l|}
\multicolumn{1}{c}{Derived Form} & \multicolumn{1}{c}{Equivalent Form} \\
\multicolumn{2}{c}{}\\
\multicolumn{2}{l}{{\bf Structure  Expressions} \strexp}\\
%\multicolumn{2}{l}{STRUCTURE EXPRESSIONS \strexp}\\
\cline{1-2}
\funappdec & \mbox{\funid\ \ml{(} \STRUCT\ \strdec\ \END\ \ml{)}}\\
\cline{1-2}
\multicolumn{2}{c}{}\\
%\multicolumn{2}{l}{FUNCTOR BINDINGS \funbind}\\
\multicolumn{2}{l}{{\bf Functor Bindings} \funbind}\\
\cline{1-2}        
\mbox{\funid\ \ml{(}\ \spec\ \ml{)}\ $\langle$\ml{:}\ \sigexp$\rangle$\ \ml{=}}&
\mbox{\funid\ \ml{(}\ \strid\ \ml{:} \SIG\ \spec\ \END\ \ml{)} 
              $\langle$\ml{:}\ $\sigexp'\rangle$\ \ml{=}}\\
\mbox{\ \ \strexp\ $\langle$\AND\ \funbind$\rangle$} &
  \mbox{\ \ \LET\ \OPEN\ \strid\ \IN\ \strexp\ \END\ $\langle$\AND\ \funbind$\rangle$} \\
\cline{1-2}
\multicolumn{2}{r}{($\strid$ new; see note in text concerning $\sigexp'$)}\\
\multicolumn{2}{c}{}\\
\multicolumn{2}{l}{{\bf Functor Signature Expressions} \funsigexp}\\
%\multicolumn{2}{l}{FUNCTOR SIGNATURES \funsigexp}\\
\cline{1-2}
\longfunsigexp & \mbox{\ml{(} \strid\ \ml{:}\ \SIG\ \spec\ \END\ \ml{)}
                \ml{:}\ \sigexp$'$} \\
\cline{1-2}
\multicolumn{2}{r}{($\strid$ new; see note in text concerning $\sigexp'$)}\\
\multicolumn{2}{c}{}\\
\multicolumn{2}{l}{{\bf Top-level Declarations} \topdec}\\
\cline{1-2}
\exp           & \VAL\ \ml{it =} \exp  \\
\cline{1-2}
\multicolumn{2}{c}{}\\
\end{tabular}
\caption{Derived forms of Functors and Top-level Declarations\index{68.2}}
\label{functor-der-forms-fig}
\end{figure}




\section{Appendix: Full Grammar}
\label{core-gram-app}
%In\index{69.1} this Appendix, the full Core 
%grammar is given for reference purposes.
The full grammar of programs is exactly as given at the start of 
Section~\ref{prog-sec}.

The\index{69.1} full grammar of Modules consists of the grammar of 
Figures \ref{mod-phr}--\ref{prog-syn} in Section~\ref{syn-mod-sec},
together with the derived forms of Figure~\ref{functor-der-forms-fig}
in Appendix~\ref{derived-forms-app}.

The remainder of this Appendix is devoted to the full grammar of the
Core. 
Roughly, it consists of the grammar of Section~\ref{syn-core-sec} augmented by
the derived forms of Appendix~\ref{derived-forms-app}.  But there is a further
difference: two additional subclasses of the phrase class ~Exp~ are introduced,
namely ~AppExp~ (application expressions) and ~InfExp~ (infix expressions).
The inclusion relation among the four classes is as follows:
\[ {\rm AtExp}\ \subset\ {\rm AppExp}\
                \subset\ {\rm InfExp}\ \subset\ {\rm Exp} \]
The effect is that certain phrases, such as
``\ml{2 + while $\cdots$ do $\cdots$ }'', are now disallowed.

The grammatical rules are displayed in Figures~\ref{exp-gram},
\ref{dec-gram}, \ref{pat-gram} and \ref{typ-gram}.
The grammatical conventions are exactly as in
Section~\ref{syn-core-sec}, namely:
\begin{itemize}
  \item The brackets ~$\langle\ \rangle$~ enclose optional phrases.
  \item For\index{69.3} any syntax class X (over which $x$ ranges)
we define the syntax class Xseq (over which {\it xseq} ranges) as follows:
    \begin{quote}
    \begin{tabular}{rcll}
       {\it xseq} & $::=$ & $x$ & (singleton sequence)\\
                  &       &     & (empty sequence)\\
                  &       & \ml{(}$x_1$\ml{,}$\cdots$\ml{,}$x_n$\ml{)}
                                & (sequence,~$n\geq 1$) \\
    \end{tabular}
    \end{quote}
(Note that the ``$\cdots$'' used here, a meta-symbol indicating syntactic
repetition, must not be
confused with ``$\wildrec$'' which is a reserved word of the language.)
  \item Alternative\index{69.4} forms for each phrase class are in order of decreasing
        precedence. This precedence resolves ambiguity in parsing in
the following way. Suppose that a phrase class --- we take $\exp$ as
an example --- has two alternative forms $F_1$ and $F_2$, such that $F_1$ ends
with an $\exp$ and $F_2$ starts with an $\exp$. A specific case is
\begin{tabbing}
\qquad\=$F_1$:\quad\=\IF\ $\exp_1$\ \THEN\ $\exp_2$\ \ELSE\ $\exp_3$\+\\
        $F_2$:     \>\handlexp
\end{tabbing}
It will be enough to see how ambiguity is resolved in this specific case.

Suppose that the lexical sequence
\[\cdots\ \cdots\ \IF\ \cdots\ \THEN\ \cdots\ \ELSE\ \exp\ \HANDLE\ \cdots\ \cdots\]
is to be parsed, where $\exp$ stands for a lexical sequence which 
is already determined as a subphrase (if necessary by applying the 
precedence rule).
Then the higher precedence of $F_2$ (in this case) dictates that $\exp$
associates to the right, i.e. that the correct parse takes the form
\[\cdots\ \cdots\ \IF\ \cdots\ \THEN\ \cdots\ \ELSE\ (\exp\ \HANDLE\ \cdots)\ \cdots\]
not the form
\[\cdots\ (\cdots\ \IF\ \cdots\ \THEN\ \cdots\ \ELSE\ \exp)\ \HANDLE\ \cdots\ \cdots\]
Note particularly that the use of precedence does not decrease the class
of admissible phrases; it merely rejects alternative ways of parsing certain
phrases. In particular, the purpose is not to prevent a phrase,
which is an instance of a form with higher precedence, having a constituent
which is an instance of a form with lower precedence. Thus for example
\[\IF\ \cdots\ \THEN\ \WHILE\ \cdots\ \DO\ \cdots\ \ELSE\ \WHILE\ \cdots\ \DO\ \cdots\]
is quite admissible, and will be parsed as
\[\IF\ \cdots\ \THEN\ (\WHILE\ \cdots\ \DO\ \cdots)\ \ELSE\ (\WHILE\ \cdots\ \DO\ \cdots)\]
  \item L (resp. R)\index{69.5} means left (resp. right) association.

\item The syntax of types binds more tightly than that of expressions.

\item Each\index{69.7} iterated construct (e.g. $\match$,  $\cdots$ )
extends as far
right as possible; thus, parentheses may be needed around an expression which
terminates with a match, e.g. ``$\FN\ \match$'', if this occurs within a
larger
match.
\end{itemize}

\begin{figure}[h]
\vspace{4pt}
\makeatletter{}
\tabskip\@centering
\halign to\textwidth
{#\hfil\tabskip1em&\hfil$#$\hfil&#\hfil&#\hfil\tabskip\@centering\cr
  \atexp& ::=	& \scon 	& special constant\cr
        & 	& \opp\longvar	& value variable\cr
	&	& \opp\longcon	& value constructor\cr
        &       & \opp\longexn  & exception constructor\cr
	&	& \verb+{ +\recexp\verb+ }+	& record\cr
        &       & \ml{\#}\ \lab   & record selector\cr
        &       & \ml{()}       & 0-tuple\cr
        &       & \ml{(}$\exp_1$ \ml{,} $\cdots$ \ml{,} $\exp_\n$\ml{)}
                                & $n$-tuple, $n\geq 2$\cr
        &       & \ml{[}$\exp_1$ \ml{,} $\cdots$ \ml{,} $\exp_\n$\ml{]}
                                & list, $n\geq 0$\cr
        &       & \ml{(}$\exp_1$ \ml{;} $\cdots$ \ml{;} $\exp_\n$\ml{)}
                                & sequence, $n\geq 2$\cr
	&	& \LET\ \dec\ \IN\
                  $\exp_1$ \ml{;} $\cdots$ \ml{;} $\exp_\n$ \END
	                        & local declaration, $n\geq 1$\cr
	&	& \parexp	& \cr
\noalign{\vspace{6pt}}
\labexps& ::=	& \longlabexps	& expression row\cr
\noalign{\vspace{6pt}}
 \apexp & ::=	& \atexp	& \cr
        &   	& \apexp\ \atexp& application expression\cr
\noalign{\vspace{6pt}}
\inexp & ::=	& \apexp	& \cr
        &   	& $\inexp_1$\ \id\ $\inexp_2$
                                & infix expression\cr
\noalign{\vspace{6pt}}
  \exp  & ::=	& \inexp 	& \cr
	&	& \typedexp	& typed (L)\cr
        &       & $\exp_1$\ \ANDALSO\ $\exp_2$
                                & conjunction\cr
        &       & $\exp_1$\ \ORELSE\ $\exp_2$
                                & disjunction\cr
	&	& \handlexp	& handle exception\cr
	&	& \raisexp     	& raise exception\cr
        &       & \IF\ $\exp_1$\ \THEN\ $\exp_2$\ \ELSE\ $\exp_3$
                                & conditional\cr
        &       & \WHILE\ \exp$_1$\ \DO\ \exp$_2$
                                & iteration\cr
        &       & \CASE\ \exp\ \OF\ \match
                                & case analysis\cr
	&	& \fnexp        & function\cr
\noalign{\vspace{6pt}}
\match  & ::=	& \longmatch    & \cr
\noalign{\vspace{6pt}}
\mrule	& ::=	& \longmrule	& \cr
\noalign{\vspace{6pt}}
}
\makeatother
\vspace{3pt}
\caption{Grammar: Expressions and Matches\index{70}}
\label{exp-gram}
\end{figure}

\begin{figure}[h]
\vspace{4pt}
\makeatletter{}
\tabskip\@centering
\halign to\textwidth
{#\hfil\tabskip1em&\hfil$#$\hfil&#\hfil&#\hfil\tabskip\@centering\cr
  \dec  & ::=	& \valdec	& value declaration\cr
	&	& \FUN\ \fvalbind
	                        & function declaration\cr
	&	& \typedec	& type declaration\cr
	&	& \datatypedeca & datatype declaration\cr
	&	& \abstypedeca  & abstype declaration\cr
        &       & \qquad\WITH\ \dec\ \END
                                & \cr
	&	& \exceptiondec & exception declaration\cr
	&	& \localdec	& local declaration\cr
        &       & \openstrdec   & open declaration, $n\geq 1$\cr
	&	& \emptydec	& empty declaration\cr
	&	& \seqdec	& sequential declaration\cr
        &       & \longinfix    & infix (L) directive, $n\geq 1$\cr
        &       & \longinfixr   & infix (R) directive, $n\geq 1$\cr
        &       & \longnonfix   & nonfix directive, $n\geq 1$\cr
%        &       & \exp          & expression (top-level only)\cr
\noalign{\vspace{6pt}}
\valbind& ::=   & \longvalbind   & \cr
	&	& \recvalbind	& \cr
\noalign{\vspace{6pt}}
\fvalbind& ::=  & \ \ $\langle\OP\rangle\var\ \atpat_{11}\cdots\atpat_{1n}
                  \langle$\ml{:}\ty$\rangle$\ml{=}\exp$_1$ & $m,n\geq 1$\cr
        &       & \ml{|}$\langle\OP\rangle\var\ \atpat_{21}\cdots\atpat_{2n}
                  \langle$\ml{:}\ty$\rangle$\ml{=}\exp$_2$ & See also note
                                                                     below\cr
        &       & \ml{|}\qquad$\cdots\qquad\cdots$ &\cr
        &       & \ml{|}$\langle\OP\rangle\var\ \atpat_{m1}\cdots\atpat_{mn}
                  \langle$\ml{:}\ty$\rangle$\ml{=}\exp$_m$ &\cr
        &       & \qquad\qquad\qquad$\langle\AND\ \fvalbind\rangle$ &\cr
\noalign{\vspace{6pt}}
\typbind& ::=	& \longtypbind	& \cr
\noalign{\vspace{6pt}}
\datbind& ::=	& \longdatbind	& \cr
\noalign{\vspace{6pt}}
\constrs& ::=	& \opp\longconstrs & \cr
\noalign{\vspace{6pt}}
\exnbind& ::=	& \generativeexnbind	& \cr
        &       & \eqexnbind   & \cr
\noalign{\vspace{6pt}}
}
\makeatother
\vspace{3pt}
Note: In the $\fvalbind$ form, if $\var$ has infix status then either
~\OP~ must be present, or $\var$ must be infixed.  Thus, at the start of
any clause, ``~\OP\ \var\ \ml{(}\atpat\ml{,}\atpat$'$\ml{)} $\cdots$'' may be
written
``\ml{(}\atpat\ \var\ \atpat$'$\ml{)} $\cdots$''; the parentheses may also be
dropped if ``\ml{:}\ty'' or ``\ml{=}'' follows immediately.
\caption{Grammar: Declarations and Bindings\index{71}}
\label{dec-gram}
\end{figure}

\begin{figure}[h]
\vspace{4pt}
\makeatletter{}
\tabskip\@centering
\halign to\textwidth
{#\hfil\tabskip1em&\hfil$#$\hfil&#\hfil&#\hfil\tabskip\@centering\cr
  \atpat& ::=	& \wildpat	& wildcard\cr
  	&	& \scon  	& special constant\cr
  	&	& \opp\var  	& variable\cr
	&	& \opp\longcon  & constant\cr
        &       & \opp\longexn  & exception constant\cr
	&	& \verb+{ +\recpat\verb+ }+       & record\cr
        &       & \ml{()}       & 0-tuple\cr
        &       & \ml{(}$\pat_1$ \ml{,} $\cdots$ \ml{,} $\pat_\n$\ml{)}
                                & $n$-tuple, $n\geq 2$\cr
        &       & \ml{[}$\pat_1$ \ml{,} $\cdots$ \ml{,} $\pat_\n$\ml{]}
                                & list, $n\geq 0$\cr
	&	& \parpat       & \cr
\noalign{\vspace{6pt}}
\labpats& ::=	& \wildrec	& wildcard\cr
  	&	& \longlabpats 	& pattern row\cr
        &       & \id$\langle$\ml{:}\ty$\rangle
                  \ \langle\AS\ \pat\rangle
                  \ \langle$\ml{,} \labpats$\rangle$
                                & label as variable\cr
\noalign{\vspace{6pt}}
  \pat	& ::=	& \atpat	& atomic\cr
	&	& \opp\conpat	& value construction\cr
        &       & \opp\exconpat  & exception construction\cr
	&	& \infpat       & infixed value construction\cr
        &       & \infexpat     & infixed exception construction\cr
	&	& \typedpat	& typed\cr
	&	& \opp\layeredpat	& layered\cr
\noalign{\vspace{6pt}}
}
\makeatother
\vspace{3pt}
\caption{Grammar: Patterns\index{72.1}}
\label{pat-gram}
\end{figure}

\begin{figure}[h]
\vspace{4pt}
\makeatletter{}
\tabskip\@centering
\halign to\textwidth
{#\hfil\tabskip1em&\hfil$#$\hfil&#\hfil&#\hfil\tabskip\@centering\cr
  \ty   & ::=	& \tyvar        & type variable\cr
	&	& \verb+{ +\rectype\verb+ }+      & record type expression\cr
	&	& \constype 	& type construction\cr
        &       & $\ty_1$ \ml{*} $\cdots$ \ml{*} $\ty_\n$
                                & tuple type, $\n\geq 2$ \cr
	&	& \funtype      & function type expression\cr
	&	& \partype      & \cr
\noalign{\vspace{6pt}}
\labtys & ::=	& \longlabtys   & type-expression row\cr
\noalign{\vspace{6pt}}
}
\makeatother
\vspace{3pt}
\caption{Grammar: Type expressions\index{72.2}}
\label{typ-gram}
\end{figure}

%\appendix
\section{Appendix: The Initial Static Basis}
\label{init-stat-bas-app}
We\index{73.1} shall indicate components of the initial basis by the subscript 0.
The initial static basis is
\[ \B_0\ =\ (\M_0,\T_0),\F_0,\G_0,\E_0\]
where
\begin{itemize}
\item $\M_0\ =\ \emptyset$
\item $\T_0\ =\ \{\BOOL,\INT,\REAL,\STRING,\LIST,\REF,\EXCN,\INSTREAM,\OUTSTREAM\}$
\item $\F_0\ =\ \emptymap$
\item $\G_0\ =\ \emptymap$
\item $\E_0\ =\ \longE{0}$
\end{itemize}
The members of $\T_0$ are type names, not type constructors; for convenience
we have used type-constructor identifiers
to stand also for the type names which are bound to them in the initial
static type environment $\TE_0$.  Of these type names,
~\LIST~ and ~\REF~
have arity 1, the rest have arity 0;  all except \EXCN, \INSTREAM~ 
and ~\OUTSTREAM~ admit equality.

The components of $\E_0$ are as follows:
\begin{itemize}
\item $\SE_0\ =\ \emptymap$
\item $\VE_0$ is shown in Figures~\ref{stat-ve} and \ref{stat-veio}. Note that
      $\Dom\VE_0$  contains those identifiers ({\tt true},{\tt false},{\tt nil},
      \verb+::+) which are basic value constructors,
      for reasons discussed in Section~\ref{stat-proj}. 
      $\VE_0$ also includes $\EE_0$, for the same reasons.
\item $\TE_0$ is shown in Figure~\ref{stat-te}. Note that the type
      structures in $\TE_0$ contain the type schemes of all basic value
      constructors.
\item $\Dom\EE_0\ =\ \BasExc$~, the set of basic exception names listed in
Section~\ref{bas-exc}.
In each case the associated type is ~\EXCN~, except that
~$\EE_0({\tt Io})=\STRING\rightarrow\EXCN$.
\end{itemize}

\begin{figure}
\begin{tabular}{|rl|rl|}
\multicolumn{2}{c}{NONFIX}&     \multicolumn{2}{c}{INFIX}\\
\hline
$\var$     & $\mapsto\ \tych$    
                          & $\var$ & $\mapsto\ \tych$\\
\hline
{\tt map}  & $\mapsto\ \forall\atyvar\ \btyvar.\ (\atyvar\to\btyvar)\to$
                          &     \multicolumn{2}{l|}{Precedence 7 :} \\
           & \qquad$\atyvar\ \LIST\to\btyvar\ \LIST$
                          & \verb+/+    & $\mapsto\ \REAL\ \ast\ \REAL
                                                    \to\REAL$\\
{\tt rev}  & $\mapsto\ \forall\atyvar.\ \atyvar\ \LIST\to\atyvar\ \LIST$
                          & {\tt div}   & $\mapsto\ \INT\ \ast\ \INT\to\INT$\\
{\tt not}  & $\mapsto\ \BOOL\to\BOOL$
                          & {\tt mod}   & $\mapsto\ \INT\ \ast\ \INT\to\INT$\\
\verb+~+   & $\mapsto\ \NUM\to\NUM$
                          & \verb+*+    & $\mapsto\ \NUM\ \ast\ \NUM\to\NUM$\\
{\tt abs}  & $\mapsto\ \NUM\to\NUM$
                          &    \multicolumn{2}{l|}{Precedence 6 :} \\
{\tt floor}& $\mapsto\ \REAL\to\INT$
                          & \verb-+-    & $\mapsto\ \NUM\ \ast\ \NUM\to\NUM$\\
{\tt real} & $\mapsto\ \INT\to\REAL$
                          & \verb+-+    & $\mapsto\ \NUM\ \ast\ \NUM\to\NUM$\\
{\tt sqrt} & $\mapsto\ \REAL\to\REAL$
                          & \verb+^+    & $\mapsto\ \STRING\ \ast\ \STRING
                                                    \to\STRING$\\
{\tt sin}  & $\mapsto\ \REAL\to\REAL$
                          &     \multicolumn{2}{l|}{Precedence 5 :} \\
{\tt cos}  & $\mapsto\ \REAL\to\REAL$
                          & \verb+::+   & $\mapsto\ \forall\atyvar.
                                          \atyvar\;{\ast}\;\atyvar\;\LIST
                                          \to\atyvar\;\LIST$\\
{\tt arctan}
           & $\mapsto\ \REAL\to\REAL$
                          & \verb+@+    & $\mapsto\ \forall\atyvar.\
                                          \atyvar\ \LIST\ $\\
{\tt exp}  & $\mapsto\ \REAL\to\REAL$
                          &             & $\qquad\ast\ \atyvar\ \LIST\to
                                           \atyvar\ \LIST$\\
{\tt ln}   & $\mapsto\ \REAL\to\REAL$
                          & \multicolumn{2}{l|}{Precedence 4 :}\\
{\tt size} & $\mapsto\ \STRING\to\INT$
                          & \verb+=+    & $\mapsto\ \forall\aetyvar.\
                                          \aetyvar\ \ast\ \aetyvar\to\BOOL$\\
{\tt chr}  & $\mapsto\ \INT\to\STRING$
                          & \verb+<>+   & $\mapsto\ \forall\aetyvar.\
                                          \aetyvar\ \ast\ \aetyvar\to\BOOL$\\
{\tt ord}  & $\mapsto\ \STRING\to\INT$
                          & \verb+<+    & $\mapsto\ \NUM\ \ast\ \NUM
                                                                  \to\BOOL$\\
{\tt explode}
           & $\mapsto\ \STRING\to\STRING\ \LIST$
                          & \verb+>+    & $\mapsto\ \NUM\ \ast\ \NUM
                                                                  \to\BOOL$\\
{\tt implode}
           & $\mapsto\ \STRING\ \LIST\to\STRING$
                          & \verb+<=+   & $\mapsto\ \NUM\ \ast\ \NUM
                                                                  \to\BOOL$\\
\verb+!+   & $\mapsto\ \forall\atyvar.\ \atyvar\ \REF\to\atyvar$
                          & \verb+>=+   & $\mapsto\ \NUM\ \ast\ \NUM
                                                                  \to\BOOL$  \\
\REF       & $\mapsto\ \forall\ \aityvar\ .\ \aityvar\to\aityvar\ \REF$
                          & \multicolumn{2}{l|}{Precedence 3 :} \\
{\tt true} & $\mapsto\ \BOOL$
                          & \verb+:=+   & $\mapsto\ \forall\atyvar.\
                                          \atyvar\ \REF\ \ast\ \atyvar\to\UNIT$\\
{\tt false}& $\mapsto\ \BOOL$
                          & {\tt o}     & $\mapsto\ \forall\atyvar\ \btyvar\ 
                                             \ctyvar.\ (\btyvar\to\ctyvar)$\\
{\tt nil}  & $\mapsto\ \forall\atyvar.\ \atyvar\ \LIST$ 
                          &             & \qquad
                                          $\ast\ (\atyvar\to\btyvar)\to(\atyvar\to\ctyvar) $ \\
\hline
\end{tabular}
\vspace{3pt}

Notes:
\begin{itemize}
\item In type schemes we have taken the liberty of writing
$\ty_1\ast\ty_2$ in place of
$\{\mbox{\tt 1}\mapsto\ty_1,\mbox{\tt 2}\mapsto\ty_2\}$.

\item  An identifier with type involving ~\NUM~ stands for two functions --
one in which ~\NUM~ is replaced by ~\INT~ in its type,
and another in which ~\NUM~ is replaced by ~\REAL~ in its type. 
Sometimes an explicit type constraint will be needed if the
surrounding text does not determine the type of a particular
occurrence of \verb-+- (for example). For this purpose, the
surrounding text is no larger than the enclosing top-level
declaration; an implementation may require that a smaller
context determines the type.
%In the case
%that both types can be inferred for an occurrence of the identifier, an
%explicit type constraint is needed to determine which type is intended.
%version 2: \item The type schemes associated with special
%constants are given in  Figure~\ref{stat-te} which shows the initial
%static type environment.
\end{itemize}
\caption{Static $\VE_0$ (except for Input/Output and $\EE_0$)\index{74}}
\label{stat-ve}
\end{figure}
 
\begin{figure}
\begin{center}
\begin{tabular}{|rl|}
\hline
$\var$            & $\mapsto\ \tych$\\
\hline
{\tt std\_in}     & $\mapsto\ \INSTREAM$\\
{\tt open\_in}    & $\mapsto\ \STRING\to\INSTREAM$\\
{\tt input}       & $\mapsto\ \INSTREAM\ \ast\ \INT\to\STRING$\\
{\tt lookahead}   & $\mapsto\ \INSTREAM\to\STRING$\\
{\tt close\_in}   & $\mapsto\ \INSTREAM\to\UNIT$\\
{\tt end\_of\_stream}
                  & $\mapsto\ \INSTREAM\to\BOOL$\\
\multicolumn{2}{|c|}{}\\
{\tt std\_out}    & $\mapsto\ \OUTSTREAM$\\
{\tt open\_out}   & $\mapsto\ \STRING\to\OUTSTREAM$\\
{\tt output}      & $\mapsto\ \OUTSTREAM\ \ast\ \STRING\to\UNIT$\\
{\tt close\_out}  & $\mapsto\ \OUTSTREAM\to\UNIT$\\
\hline
\end{tabular}
\end{center}
\vspace{3pt}
\caption{Static $\VE_0$ (Input/Output)\index{75.1}}
\label{stat-veio}
\end{figure}
 
\begin{figure}
\begin{center}
\begin{tabular}{|rll|}
\hline
$\tycon$   & $\mapsto\ \{\ \typefcn$, & $\{\con_1\mapsto\tych_1,\ldots,\con_n\mapsto\tych_n\}\ \}\quad (n\geq0)$\\
\hline
\UNIT      & $\mapsto\ \{\ \Lambda().\{ \}$,
                                      & $\emptymap\ \}$ \\
\BOOL      & $\mapsto\ \{\ \BOOL$,    & $\{\TRUE\mapsto\BOOL,
                                         \ \FALSE\mapsto\BOOL\}\ \}$\\
\INT       & $\mapsto\ \{\ \INT$,     & $\{\}\ \}$\\
\REAL      & $\mapsto\ \{\ \REAL$,    & $\{\}\ \}$\\
\STRING    & $\mapsto\ \{\ \STRING$,  & $\{\}\ \}$\\
%version 2: \INT       & $\mapsto\ \{\ \INT$,     & $\{i\mapsto\INT\ ;\
%                                           i$ an integer constant$\}\ \}$\\
%\REAL      & $\mapsto\ \{\ \REAL$,    & $\{r\mapsto\REAL\ ;\
%                                           r$ a real constant$\}\ \}$\\
%\STRING    & $\mapsto\ \{\ \STRING$,  & $\{s\mapsto\STRING\ ;\
%                                           s$ a string constant$\}\ \}$\\
\LIST      & $\mapsto\ \{\ \LIST$,    & $\{\NIL\mapsto\forall\atyvar\ .\ \atyvar\ \LIST$,\\
           &                          & $\ \mbox{\texttt+::+}\ \mapsto\forall\atyvar\ .
                                           \ \atyvar\ast\atyvar\ \LIST
                                           \to\atyvar\ \LIST\}\ \}$\\
%\LIST      & \multicolumn{2}{l|}{$\mapsto\ \{\ \LIST,    \{\NIL\mapsto
%                                         \forall\alpha.\alpha\LIST,\
%                                           \ml{::}\mapsto\forall\alpha.
%                                           \alpha\ast\alpha\LIST
%                                           \to\alpha\LIST\}\ \}$
%                               }\\
\REF       & $\mapsto\ \{\ \REF$,     & $\{\REF\mapsto\forall\ \aityvar\ .\ 
                                           \aityvar\to\aityvar\ \REF\}\ \}$\\
\EXCN      & $\mapsto\ \{\ \EXCN$,     & $\emptymap\ \}$\\
\INSTREAM  & $\mapsto\ \{\ \INSTREAM$,& $\emptymap\ \}$ \\
\OUTSTREAM & $\mapsto\ \{\ \OUTSTREAM$,& $\emptymap\ \}$ \\
\hline
\end{tabular}
\end{center}
\caption{Static $\TE_0$\index{75.2}}
\label{stat-te}
\end{figure}

\section{Appendix: The Initial Dynamic Basis}
\label{init-dyn-bas-app}
We\index{76.1} shall indicate components of the initial basis by the subscript 0.
The initial dynamic basis is
\[ \B_0\ =\ \F_0,\G_0,\E_0\]
where
\begin{itemize}
\item $\F_0\ =\ \emptymap$
\item $\G_0\ =\ \emptymap$
\item $\E_0\ =\ \E_0'+\E_0''$
\end{itemize}
$\E_0'$ contains bindings of identifiers to the basic values BasVal and
basic exception names \BasExc; in fact
~$\E_0'\ =\ \SE_0',\VE_0',\EE_0'$~, where:
\begin{itemize}
\item $\SE_0'\ =\ \emptymap$
\item $\VE_0'\ =\ \{\id\mapsto\id\ ;\ \id\in$ BasVal$\}
                             \cup\{$\verb+:=+ $\mapsto$ \verb+:=+$\}$
\item $\EE_0'\ =\ \{\id\mapsto\id\ ;\ \id\in$ \BasExc$\}$
\end{itemize}
Note that $\VE_0'$ is the identity function on BasVal; this is because
we have chosen to denote these values by the names of variables 
to which they are initially bound.
The semantics of these basic values (most of which are functions)
lies principally in their behaviour under APPLY, which we describe below.
On the other hand the semantics of \verb+:=+ is provided by a special
semantic rule, rule~\ref{assapp-dyn-rule}.
Similarly, $\EE_0'$ is the identity function on \BasExc, the set of
basic exception names, because we have also chosen
these names to be just those exception constructors to which they
are initially bound.
These exceptions are raised by APPLY as described below.

 $\E_0''$ contains initial variable bindings which, unlike BasVal, are
 definable in ML; it is the result of evaluating
 the following declaration in the basis $\F_0,\G_0,\E_0'$.  For convenience,
 we have also included all basic infix directives in this declaration.
 \begin{verbatim}
                infix  3  o
                infix  4  = <> < > <= >=
                infix  5  @
                infixr 5  ::
                infix  6  + - ^
                infix  7  div mod / *

                fun (F o G)x = F(G x)

                fun nil @ M = M
                  | (x::L) @ M = x::(L @ M)

                fun s ^ s' = implode((explode s) @ (explode s'))

                fun map F nil = nil
                  | map F (x::L) = (F x)::(map F L)

                fun rev nil = nil
                  | rev (x::L) = (rev L) @ [x]

                fun not true = false
                  | not false = true

                fun ! (ref x) = x
 \end{verbatim}

We now\index{77.1} describe the effect of APPLY upon each value
$b\in\BasVal$.  For special values, we shall normally use $i$, $r$,
$n$, $s$ to range over integers, reals, numbers (integer or real),
strings respectively.  We also take the liberty of abbreviating
``APPLY(\mbox{{\tt abs}}, $r$)'' to ``\mbox{{\tt abs}}($r$)'',
``APPLY({\tt mod}, $\{\verb+1+\mapsto i,\verb+2+\mapsto d\}$)'' to
``$i\ {\tt mod}\ d$'', etc. .
\begin{itemize}
\item \verb+~+($n$)~ returns the negation of $n$, or the 
      packet ~[{\tt Neg}]~ if the result is out of range.
\item \mbox{{\tt abs}}($n$)~ returns the absolute value of $n$, or
      the packet ~[{\tt Abs}]~ if the result is out of range.
\item {\tt floor}($r$)~ returns the largest integer $i$ not greater than $r$;
      it returns the packet ~[{\tt Floor}]~ if $i$ is out of range.
\item {\tt real}($i$)~ returns the real value equal to $i$.
\item {\tt sqrt}($r$)~ returns the square root of $r$, or the packet
      ~[{\tt Sqrt}]~ if r is negative.
\item {\tt sin}($r$)~,  {\tt cos}($r$)~ return the result of the appropriate
      trigonometric functions.
\item {\tt arctan}($r$)~ returns the result of the appropriate
      trigonometric function in the range $\underline{+}\pi/2$.
\item {\tt exp}($r$)~, {\tt ln}($r$)~ return respectively the exponential
      and the natural logarithm of $r$, or an exception packet
      ~[{\tt Exp}]~ or ~[{\tt Ln}]~ if the result is out of range.
\item {\tt size}($s$)~ returns the number of characters in $s$.
\item {\tt chr}($i$)~ returns the character numbered $i$ (see Section~\ref{cr:speccon}) if $i$ is in the interval $[0,255]$, and the packet
      ~[{\tt Chr}]~ otherwise.
\item {\tt ord}($s$)~ returns\index{78.0} the number of the first character
      in $s$ (an integer in the interval $[0,255]$, see Section~\ref{cr:speccon}), 
      or the packet ~[{\tt Ord}]~ if $s$ is empty.
\item {\tt explode}($s$)~ returns the list of characters (as single-character
      strings) of which $s$ consists.
\item {\tt implode}($L$)~ returns the string formed by concatenating all members
      of the list $L$ of strings.
\item The arithmetic\index{78.1} functions ~\verb+/+,\verb+*+,\verb-+-,\verb+-+~ all
      return the results of the usual
      arithmetic operations, or exception packets respectively
      [{\tt Quot}], [{\tt Prod}], [{\tt Sum}], [{\tt Diff}]
      if the result is undefined or out of range.
\item $i\ {\tt mod}\ d$~,~$i\ {\tt div}\ d$~ return integers $r,q$
      (remainder, quotient) determined by the equation $d\times q +r=i$,
      where either $0\leq r<d$ or $d<r\leq 0$.  Thus the remainder has the
      same sign as the divisor $d$. The packet [{\tt Mod}] or
      [{\tt Div}] is returned if $d=0$.
\item The order relations ~\verb+<+,\verb+>+,\verb+<=+,\verb+>=+~ return
      boolean values in accord with their usual meanings.
\item $v_1$\verb+ = +$v_2$~ returns {\TRUE} or {\FALSE} according as
      the values $v_1$ and $v_2$ are, or are not, identical.
      The type discipline (in particular, the fact that function types
      do not admit equality) ensures that equality is only ever applied
      to special values, nullary constructors, addresses, and values
      built out of such by record formation and constructor application.
%version 2:\item $v_1$\verb+ = +$v_2$~ returns the boolean value of $v_1=v_2$,
%      where the equality of values (=) is defined recursively as follows:
%      \begin{itemize}
%      \item If $v_1,v_2$ are constants (including nullary constructors) or
%      addresses, then $v_1=v_2$ iff $v_1$ and $v_2$ are identical.
%      \item $(\con_1,v_1)=(\con_2,v_2)$ iff $\con_1,\con_2$ are identical and
%            $v_1=v_2$.
%      \item $r_1=r_2$ (for records $r_1,r_2$) iff $\Dom r_1=\Dom r_2$ and, for
%            each $\lab\in\Dom r_1$, $r_1(\lab)=r_2(\lab)$.
%      \end{itemize}
%      The type discipline (in particular, the fact that function types
%      do not admit equality) makes it unnecessary to specify equality in
%      any other cases.
\item $v_1$\verb+ <> +$v_2$ returns the opposite boolean value to
      $v_1$\verb+ = +$v_2$.
\end{itemize}
It remains to define the effect of APPLY upon basic values concerned with
input/output; we therefore proceed to describe the ML input/output system.
 
Input/Output in ML uses the concept of a {\sl stream}. A stream is a finite or
infinite sequence of characters; if finite, it may or may not be terminated.
(It may be convenient to think of a special end-of-stream character
signifying termination, provided one realises that this ``character'' is
never treated as data).
Input streams -- or {\sl instreams} --
are of type ~\INSTREAM~ and will be denoted by ~$is$~;
output streams -- or {\sl outstreams} -- are of type ~\OUTSTREAM~ and will
be denoted by ~$os$~. Both these types of stream are {\sl abstract}, in the
sense that streams may only be manipulated by the functions provided in
BasVal.

Associated with an instream is a {\sl producer}, normally an I/O device or
file; similarly an outstream is associated with a {\sl consumer}.  After this
association has been established -- either initially or by the ~{\tt open\_in}~
or ~{\tt open\_out}~ function -- the stream acts as a vehicle for character
transmission from producer to program, or from program to consumer.
The association can be broken by the ~{\tt close\_in}~ or ~{\tt close\_out}~
function.\index{79.1}
A closed stream permits no further character transmission; a closed
instream is equivalent to one which is empty and terminated.

There\index{79.1.1} are two streams in BasVal:
\begin{itemize}
\item  {\tt std\_in}: an instream whose producer is the terminal.
\item  {\tt std\_out}: an outstream whose consumer is the terminal.
\end{itemize}
The other basic values concerned with Input/Output are all functional, and
the effect of APPLY upon each of them given below. We take the
liberty of abbreviating ``APPLY({\tt open\_in}, $s$)'' to 
``{\tt open\_in}($s$)''
etc., and
we shall use ~$s$~ and ~$n$~ to range over strings and integers
respectively.
\begin{itemize}
\item  {\tt open\_in}($s$)~ returns a new instream ~$is$~, whose producer is
       the external file named ~$s$~. It returns exception packet
       \begin{quote}
       [$(${\tt Io},\verb+"Cannot open +$s$\verb+"+$)$]
       \end{quote}
       if file ~$s$~ does not exist or does not provide read access.
\item  {\tt open\_out}($s$)~ returns a new outstream ~$os$~, whose consumer is
       the
       external file named ~$s$~. If file $s$ is non-existent, it is taken to
       be initially empty.
\item  {\tt input}($is,n$)~ returns a string ~$s$~ containing the first $n$
       characters of ~$is$~, also removing them from ~$is$~.  If only $k<n$
       characters are available on ~$is$~, then
       \begin{itemize}
       \item If ~$is$~ is terminated after these $k$ characters, the
             returned string
            ~$s$~ contains them alone, and they are removed from ~$is$~.
       \item Otherwise no result is returned until the producer of ~$is$~
             either supplies $n$ characters or terminates the stream.
       \end{itemize}
\item  {\tt lookahead}($is$)~ returns a single-character string ~$s$~ containing
       the next character of ~$is$~, without removing it. If no character is
       available on ~$is$~ then
       \begin{itemize}
       \item If ~$is$~ is closed, the empty string is returned.
       \item Otherwise no result is returned until the producer of ~$is$~
             either supplies a character or closes the stream.
       \end{itemize}
\item  {\tt close\_in}($is$)~ empties and terminates the instream ~$is$~ .
\item  {\tt end\_of\_stream}($is$)~ returns {\TRUE} if 
       ~{\tt lookahead}($is$)~ returns the empty string, {\FALSE} otherwise; 
        it detects the end of the instream ~$is$~.
\item {\tt output}($os,s$)~ writes the characters of ~$s$~  to the outstream
       ~$os$~, unless ~$os$~ is closed, in which case it returns the exception
       packet
       \begin{quote}
       [$(${\tt Io},\verb+"Output stream is closed"+$)$]
       \end{quote}
\item  {\tt close\_out}($os$)~ terminates the outstream ~$os$~.\index{79.2}
\end{itemize}
%\end{document}

%              Constructors
%\newcommand{\FALSE}{\mbox{\tt false}}
%\newcommand{\TRUE}{\mbox{\tt true}}
%\newcommand{\NIL}{\mbox{\tt nil}}
%\newcommand{\REF}{\mbox{\tt ref}}
%\newcommand{\UNIT}{\mbox{\tt unit}}

%              Basic Values BasVal
%\newcommand{\STDIN}{\mbox{\tt std\_in}}
%\newcommand{\STDOUT}{\mbox{\tt std\_out}}

\section{Appendix: The Development of ML}
\label{story-app}

This Appendix records the main stages in the development
of ML, and the people principally involved.  The main emphasis is upon
the design of the language; there is also a section devoted to 
implementation. On the other hand, no attempt is made to record work on 
implementation environments, or on applications of the language.
 
\subsection*{Origins}

ML and its semantic description have evolved over a period of about
fourteen years.  It is a fusion of many ideas from many people;  in this
appendix we try to record and to acknowledge the important precursors
of its ideas, the important influences upon it, and the important
contributions to its design, implementation and semantic description.

ML, which stands for {\sl meta language}, was conceived as a medium for
finding and performing proofs in a formal logical system.  This
application was the focus of the initial design effort, by Robin Milner
in collaboration first with Malcolm Newey and Lockwood Morris, then
with Michael Gordon and Christopher Wadsworth \cite{GMMNW}. The intended application
to proof affected the design considerably.  Higher order functions 
in full generality seemed necessary for programming proof tactics and
strategies, and also a robust type system (see below).  At the same time,
imperative features were important for practical reasons; no-one had experience
of large useful programs written in a pure functional style. In particular,
an exception-raising mechanism was highly desirable for the natural presentation
of tactics.

The full definition of this first version of ML was included in a book 
\cite{GMW}
which describes LCF, the proof system which ML was designed to support. 
The details of how the proof application exerted an influence on design is
reported by Milner \cite{Mil2}.
Other early influences were the applicative languages already in use
in Artificial Intelligence, principally LISP \cite{McC}, ISWIM \cite{Lan} and 
POP2 \cite{BP}.

\subsection*{Polymorphic types}

The polymorphic type discipline and the associated type-assignment algorithm
were prompted by the need for security;  it is vital to know that when
a program produces an object which it claims to be a theorem, then it
is indeed a theorem.  A type discipline provides the security, but a
polymorphic discipline also permits considerable flexibility.

The key ideas of the type discipline were evolved in
combinatory logic by Haskell Curry and Roger Hindley, who arrived at
different but equivalent algorithms for computing principal type schemes.
Curry's \cite{Cur} algorithm was by equation-solving; Hindley \cite{Hin} used
the unification algorithm of Alan Robinson \cite{Rob} and  also presented the 
precursor of our type inference system.
James Morris \cite{Mor} independently gave an equation-solving algorithm very
similar to Curry's.  The idea of an algorithm for finding principal 
type schemes is very natural and may well have been known earlier. I am
grateful to Roger Hindley for pointing out that Carew Meredith's inference 
rule for propositional logic called Condensed Detachment, defined
in the early 1950s, clearly suggests that he knew such an algorithm \cite{Mer}. 

Milner \cite{Mil1}, during the design of ML,
rediscovered principal types and their calculation by unification, for a
language (slightly richer than combinatory logic) containing local
declarations.  He and Damas \cite{DM} presented the ML type inference systems
following Hindley's style.
Damas \cite{Dam}, using ideas from Michael Gordon, also devised the first
mathematical treatment of polymorphism in the presence of references and
assignment.  Tofte \cite{Tof-a} produced a different scheme, which
has been adopted in the language.

\subsection*{Refinement of the Core Language}

Two movements led to the re-design of ML.  One was the work of Rod
Burstall and his group on specifications, crystallised in the specification
language CLEAR \cite{BG} and in the functional programming language HOPE
\cite{BMS};  the latter was for expressing executable specifications.  The
outcome of this work which is relevant here was twofold.  First, there were
elegant programming features in HOPE, particularly pattern matching
and clausal function definitions;  second, there were ideas on modular
construction of specifications, using signatures in the 
interfaces.  A smaller but significant movement was by Luca Cardelli,
who extended the data-type repertoire in ML by adding named records and
variant types.

In 1983, Milner (prompted by Bernard Sufrin) wrote the first draft of a
standard form of ML attempting to unite these ideas;  over the next three
years it evolved into the Standard ML Core Language.  Notable here was
the harmony found among polymorphism, HOPE patterns and Cardelli records,
and the nice generalisations of ML exceptions due to ideas from Alan
Mycroft, Brian Monahan and Don Sannella.  A simple stream-based I/O mechanism
was developed from ideas of Cardelli by Milner and Harper.  The Standard ML
Core Language is described in detail in a composite report \cite{HMM} which also
contains a description of the I/O mechanism and MacQueen's proposal
for program modules (see later for discussion of this). Since then only
few changes to the Core Language have occurred.  Milner proposed
equality types, and these were added, together with a few minor
adjustments \cite{Mil3}.   The latest and final development has been in the
exception mechanism, by MacQueen using an idea from Burstall \cite{AMMT};  
it unites the ideas of exception and data type construction.

\subsection*{Modules}

Besides contributory ideas to the Core Language, HOPE \cite{BMS} contained
a simple notion of program module.  The most important and original
feature of ML Modules, however, stems from the work on parameterised
specifications in CLEAR \cite{BG}.  MacQueen, who was a member of Burstall's
group at the time, designed \cite{Mac} a new parametric module feature for HOPE
inspired by the CLEAR work.
He later extended the parameterisation ideas by a novel method
of specifying sharing of components among the structure parameters of
a functor, and produced a draft design which accommodated features already
present in ML -- in particular the polymorphic type system.  This design
was discussed in detail at Edinburgh, leading to MacQueen's first
report on Modules \cite{HMM}.

Thereafter, the design came under close scrutiny through a draft operational
static semantics and prototype
implementation of it by Harper, through Kevin Mitchell's implementation of
the evaluation, through a denotational semantics written by
Don Sannella, and then through further work on operational semantics by Milner
and Tofte.  (More is said about this in the later section on Semantics.)
In all of this work the central ideas withstood scrutiny, while it
also became clear that there were gaps in the design and ambiguities in
interpretation.  (An example of a gap was the inability to specify sharing
between a functor argument structure and its result structure;  an
example of an ambiguity was the question of whether sharing exists in a
structure over and above what is specified in the signature expression
which accompanies its declaration.)

Much discussion ensued;  it was possible for a wider group to comment on
Modules through using Harper's prototype implementation, while
Harper, Milner and Tofte gained understanding during development of this
semantics.  In parallel, Sannella and Tarlecki explored the implications
of Modules for the methodology of program development \cite{ST}.  Tofte, in his 
thesis \cite{Tof}, proved several technical
properties of Modules in a skeletal language, which generated considerable
confidence in this design. A key point in this development was the proof
of the existence of principal signatures, and, in the careful distinction
between the notion of {\it enrichment} of structures, which allows more
polymorphism and more components, and {\it realisation} which allows more
sharing.

At a meeting in Edinburgh in 1987 a choice of two designs was presented,
hinging upon whether or not a functor application should coerce its
actual argument to its argument signature.  The meeting chose coercion,
and thereafter the production of Section~\ref{statmod-sec} of this report -- the
Static Semantics of Modules -- was a matter of detailed care.  That
section is undoubtedly the most original and demanding part of this
semantics, just as the ideas of MacQueen upon which it is based are the
most far-reaching extension to the original design of ML.

\subsection*{Implementation}

The first implementation of ML was by Malcolm Newey, Lockwood Morris
and Robin Milner in 1974, for the DEC10.  Later Mike Gordon and Chris
Wadsworth joined;  their work was mainly in specialising ML towards
machine-assisted reasoning.  Around 1980 Luca Cardelli implemented
a version on VAX;  his work was later extended by Alan Mycroft, Kevin
Mitchell and John Scott.  This version contained one or two new data-type 
features, and was based upon the Functional Abstract Machine (FAM), a virtual
machine which has been a considerable stimulus to later implementation.
By providing a reasonably efficient implementation, this work enabled the 
language to be taught to students; this, in turn, prompted
the idea that it could become a useful general purpose language.

In Gothenburg, an implementation was developed by Lennart Augustsson and 
Thomas Johnsson in 1982, using lazy 
evaluation rather than call-by-value; the result was called Lazy ML
and is reported in \cite{Aug}.  This work is part of continuing research
in many places on implementation of lazy evaluation in pure functional 
languages.  But for ML, which includes exceptions and assignment, the 
emphasis has been mainly upon strict evaluation (call-by-value).

In Cambridge, in the early 1980s, Larry Paulson made considerable improvements
to the Edinburgh ML compiler, as part of his wider programme of improving
Edinburgh LCF to become Cambridge LCF \cite{Pau}.  This system has supported 
larger proofs than the Edinburgh system, and with greater convenience; in 
particular, the compiled ML code ran four to five times faster.

Around the same time G\'{e}rard Huet at INRIA (Versailles) adapted ML to 
Maclisp on Multics, again for use in machine-assisted proof.  There was
close collaboration between INRIA and Cambridge in this period.
ML has undergone a separate development in the group at INRIA, arriving at a 
language and implementation known as CAML \cite{CCM}; this is close to the core 
language of Standard ML, but does not include the Modules. 

The first implementation of the Standard ML core language was by Mitchell, 
Mycroft and John Scott of Edinburgh, around 1984, and this was shortly followed
by an implementation by David Matthews at Cambridge, carried out 
in his language Poly.  

The prototype implementation of Modules, before that part
of the language settled down, was done in 1985-6;  Mitchell dealt with
evaluation, while Harper tackled the elaboration (or `signature checking')
which raised problems of a kind not previously encountered.  The Edinburgh
implementation continues to play the role of a test-bed for language
development.  

Meanwhile Matthews' Cambridge implementation also advanced to embrace
Modules. This implementation has 
supported applications of considerable size, both for machine-assisted proof
and for hardware design. 

In 1986, as the Modules definition was settling down, David MacQueen began 
an implementation at Bell AT\&T Laboratories, joined later by Andrew Appel
and Trevor Jim who are particularly interested in compilation into high quality
machine code.

The Bell and Cambridge implementations, the former led by MacQueen and Appel,
the latter by Matthews, are currently the most complete and highly engineered.
Other currently active implementations are by Michael Hedlund at the
Rutherford-Appleton Laboratory, by Robert Duncan, Simon Nichols and
Aaron Sloman at the University of Sussex (POPLOG) and by Malcolm Newey
and his group at the Australian National University.
 
\subsection*{Semantics}

The description of the first version of ML \cite{GMW} was informal, and in an 
operational style;  around the same time a denotational semantics was written,
but never published, by Mike Gordon and Robin Milner.  Meanwhile structured
operational semantics, presented as an inference system, was gaining
credence as a tractable medium.  This originates with the reduction rules
of $\lambda$-calculus, but was developed more widely through the work
of Plotkin \cite{Plo}, and also by Milner.   This was at
first only used for dynamic semantics, but later the
benefit of using inference systems for both static and dynamic
semantics became apparent.  This advantage was realised when Gilles Kahn
and his group at INRIA were able to execute early versions of both forms of 
semantics for the ML Core Language using their Typol system \cite{Des}.  
The static and dynamic
semantics of the Core reached a final form mostly through work by
Mads Tofte and Robin Milner.

The modules of ML presented little difficulty as far as dynamic semantics
is concerned, but the static semantics of Modules was a concerted effort
by several people.  MacQueen's original informal description \cite{HMM}
was the starting point;  Sannella wrote a denotational semantics for
several versions, which showed that several issues had not been settled
by the informal description.  Robert Harper, while writing the first
implementation of Modules, made the first draft of the static semantics.
Harper's version made clear the importance of structure names; 
work by Milner and Tofte introduced further ideas including realisation;
thereafter a concerted effort by all three led to several suggestions
for modification of the language, and a small range of alternative
interpretations;  these were assessed in discussion with MacQueen, and
more widely with the principal users of the language, and an agreed form
was reached.


There is no doubt that the interaction between design and semantic
description of  Modules has been one of the most striking phases in the
entire language development, leading (in the opinion of those involved)
to a high degree of confidence both in the language and in the semantic
method.

\subsection*{Literature} 
The present document is the definition of Standard ML; further versions
of it will be produced as the language develops (but the intention is
to minimise the number of versions).  An informal definition, consistent
with Version~2 of this document as far as the Core Language is concerned,
is provided by \cite{HMM}, as modified by \cite{Mil3} and \cite{AMMT}. An elementary
textbook covering the Core language has been recently published, 
written by {\AA}ke Wikstr\"{o}m \cite{Wik}.  
Robert Harper \cite{Har} has written a shorter
introduction which also includes material on Modules.

\subsection*{Further acknowledgments}
Apart from the people mentioned above we also acknowledge the following,
all of whom have contributed in some way to the evolution of ML:
Guy Cousineau, Simon Finn, Jim Hook,
Gerard Huet, Gilles Kahn,
Brian Monahan, 
Peter Mosses, Alan Mycroft,  David Park,
David Rydeheard, 
David Schmidt,  Stefan Sokolowski, Bernard Sufrin,
Philip Wadler.

\newpage
%\subsection*{References}
\addcontentsline{toc}{section}{\protect\numberline{}{\vrule width0pt height2cm depth0pt References}}
\label{references-sec}
\begin{thebibliography}{99}

\bibitem{AMMT} Appel, A., MacQueen, D.B., Milner, R. and Tofte, M., 
{\em Unifying Exceptions with Constructors in Standard ML}, Report ECS-LFCS-88-55,
Laboratory for Foundations of Computer Science, 
Computer Science Dept, Edinburgh University, 1988.

\bibitem{Aug} Augustsson L. and Johnsson, T., {\em Lazy ML User's Manual}, 
Dept.\ of Computer Sciences, Chalmers University of Technology,
Gothenburg, 1987.

\bibitem{BMS} Burstall R.M., MacQueen, D.B. and Sannella, D.T., 
{\em HOPE: An Experimental
Applicative Language}, Report CSR-62-80, 
Computer Science Dept, Edinburgh University, 1980.

\bibitem{BG} Burstall, R.M. and Goguen, J.A., {\em Putting Theories together to
make Specifications}, Proc Fifth Annual Joint Conference on Artificial
Intelligence, Cambridge, Mass., 1977, pp 1045--1058.
\bibitem{BP} Burstall, R.M. and Popplestone, R., {\em POP-2 Reference Manual},
Machine Intelligence 2, ed Dale and Michie, Oliver and Boyd, 1968.

\bibitem{CCM} Cousineau, G., Curien, P.L. and Mauny, M., {\em The Categorical
Abstract Machine}, in Functional Programming Languages and Computer 
Architecture, ed Jouannaud, Lecture Notes in Computer Science Vol 201,
Springer Verlag, 1985, pp 50--64.

\bibitem{Cur} Curry, H.B., {\em Modified Basic Functionality in Combinatory
Logic}, Dialectica 23, 1969, pp 83--92.

\bibitem{Dam} Damas, L., {\em Type Assignment in Programming Languages}, PhD
thesis, CST-33-85, Computer Science Department, Edinburgh University, 1985.

\bibitem{DM} Damas, L. and Milner, R., {\em Principal Type-schemes for
Functional Programs}, Proc 9th annual symposium on Principles of Programming
Languages, ACM, 1982.

\bibitem{Des} Despeyroux, T., {\em Executable Specification of Static Semantics},
Proc Symposium on Semantics of Data Types, Sophia Antipolis,
Springer-Verlag Lecture Notes in Computer Science, Vol.173, 1984.

\bibitem{GMMNW} Gordon, M.J.C., Milner, R., Morris,L., Newey, M.C. and 
Wadsworth, C.P., {\em A Metalanguage for Interactive Proof in LCF}, 
Proc 5th ACM Symposium on Principles of
Programming Languages, Tucson, 1978.
 
\bibitem{GMW} Gordon, M.J.C., Milner, R. and Wadsworth, C.P., 
{\em Edinburgh LCF: a Machanised Logic of Computation},
Springer-Verlag Lecture Notes in Computer Science, Vol.78, 1979.

\bibitem{Har} Harper, R.W., {\em Introduction to Standard ML}, Report
ECS-LFCS-86-14, 
Laboratory for Foundations of Computer Science, 
Computer Science Department, Edinburgh University, 1986.

\bibitem{Hin} Hindley, R., {\em The Principal Type-scheme of an Object in
Combinatory Logic}, Transactions of AMS 146, pp29--60, 1969.

\bibitem{HMM} Harper, R.M., MacQueen, D.B. and Milner, R., 
{\em Standard ML},
Report ECS-LFCS-86-2, 
Laboratory for Foundations of Computer Science, 
Computer Science Department, Edinburgh University, 1986.

\bibitem{ML1} Harper, R.M., Milner, R., Tofte, M., 
{\em The Semantics of Standard ML, Version 1}
Report ECS-LFCS-87-36,
Laboratory for Foundations of Computer Science, 
Computer Science Department, Edinburgh University, 1987.
 
\bibitem{ML2} Harper, R.M., Milner, R., Tofte, M., 
{\em The Definition of Standard ML, Version 2}
Report ECS-LFCS-88-62,
Laboratory for Foundations of Computer Science, 
Computer Science Department, Edinburgh University, 1988.
 
\bibitem{ML3} Harper, R.M., Milner, R., Tofte, M., 
{\em The Definition of Standard ML, Version 3}
Report ECS-LFCS-89-81,
Laboratory for Foundations of Computer Science, 
Computer Science Department, Edinburgh University, 1989.
 
\bibitem{Lan} Landin, P.J., {\em The next 700 Programming Languages}, CACM, Vol.9,
 No.3, 1966, pp57--164.

\bibitem{Mac} MacQueen, D.D., {\em Structures and parameterisation in a typed 
functional language}, Proc.\ Symposium on Functional Programming and Computer
Architecture, Aspinas, Sweden, 1981.

\bibitem{McC} McCarthy, J. et al., {\em LISP 1.5 Programming Manual}, 
The MIT Press, Cambridge, Mass, 1956.

\bibitem{Mer} Meredith, D., {\em In memoriam Carew Arthur Meredith}, Notre Dame
J. Formal Logic, Vol 18, 1977, pp 513--516.

\bibitem{Mil1} Milner, R.,
 {\em A theory of type polymorphism in programming},
J. Comp. Sys.Sci, Vol 17, 1978, pp 348--375.

\bibitem{Mil2} Milner, R., {\em How ML Evolved}, 
Polymorphism (The ML/LCF/Hope Newsletter),
Vol.1, No.1, 1983.

\bibitem{Mil3} Milner, R., {\em Changes to the Standard ML Core Language},
Report ECS-LFCS-87-33,
Laboratory for Foundations of Computer Science, 
Computer Science Department, Edinburgh University, 1987.

\bibitem{Mor} Morris, J.H., {\em Lambda Calculus Models of Programming Languages},
MAC-TR-57 (Thesis), Project MAC, M.I.T., 1968.

\bibitem{Pau} Paulson, L.C., {\em Logic and Computation: Interactive Proof with
LCF}, Cambridge Tracts in Theoretical Computer Science 2, Cambridge University
Press, 1987.

\bibitem{Plo} Plotkin, G.D., {\em A Structural Approach to Operational Semantics},
Technical Report DAIMI FN-19, Computer Science Department, \AA rhus University,
1981.

\bibitem{Rob} Robinson, J.A., {\em A Machine-oriented Logic based upon the
Resolution Principle}, Journal of ACM, Vol 12, No 1, pp23-41, 1965. 

\bibitem{ST} Sannella, D.T. and Tarlecki, A., {\em Program Specification
and Development in Standard ML}, 
Proc 12th ACM Symposium on Principles of
Programming Languages, New Orleans, 1985.

\bibitem{Tof} Tofte, M., {\em Operational Semantics and Polymorphic Type
Inference}, PhD Thesis CST-52-88, 
Computer Science Department, Edinburgh University, 1988. (Also appears as
Report ECS-LFCS-88-54 of the Laboratory for Foundations of Computer Science.)

\bibitem{Tof-a} Tofte, M., {\em Type Inference for Polymorphic References}
(To appear in Information and Computation)

\bibitem{Wik} Wikstr\"{o}m, \AA., {\em Functional Programming using Standard
ML}, Prentice Hall, 1987.
\end{thebibliography}





\addcontentsline{toc}{section}{\protect\numberline{}{Index}}
\label{index-sec}
\begin{theindex}
\item \verb+()+ (0-tuple), 79, 80, 84, 86
\item \verb+(   )+, 3
\subitem in expression, 9, 10, 29, 61, 79, 83, 84
\subitem in pattern, 11, 34, 65, 80, 86
\subitem in sequence, 8, 82
\subitem in type expression, 11, 35, 86
\item \verb+[   ]+, 3, 79, 80, 84, 86
\item \verb+{   }+, 3
\subitem in atomic expression, 10, 29, 61, 84
\subitem in pattern, 11, 34, 65, 86
\subitem in record type expression, 11, 35, 86
\item \verb+(*  *)+ (comment brackets), 4, 6
\item \verb+,+ (comma), 3, 8, 79, 82, 84, 86
\item \verb+...+ (wildcard pattern row), 3, 11, 34, 36, 66, 86
\item \verb+_+ (underbar) 
\subitem wildcard pattern, 3, 34, 65, 86
\subitem in identifier, 4
\item \verb+|+, 3, 4, 84, 85
\item \verb+=+ (reserved word), 3
\item \verb+=+ (identifier and basic value), 4, 56, 90, 92
\item \verb+=>+, 3
\subitem in a match rule, 10, 84
\item \verb+->+, 3, 11, 35, 86
\item \verb+~+, 3, 4, 56, 88, 91
\item \verb+.+ (period) 
\subitem in real constants, 3
\subitem in long identifiers, 4
\item \verb+"+, 3
\item \verb+\+, 3, 4
\item \verb+!+, 4, 88, 91
\item \verb+%+, 4
\item \verb+&+, 4
\item \verb+$+, 4
\item \verb+#+, 3, 4, 79, 84
\item \verb(+(, 4, 56, 88, 90, 92
\item \verb+-+, 4, 56, 88, 90, 92
\item \verb+/+, 4, 56, 88, 90, 92
\item \verb+:+ (see also type constraint), 4
\item \verb+::+, 87--90
\item \verb+:=+ (assignment), 57, 61, 88, 90
\item \verb+<+, 4, 56, 88, 90, 92
\item \verb+>+, 4, 56, 88, 90, 92
\item \verb+<=+, 56, 88, 90, 92
\item \verb+>=+, 56, 88, 90, 92
\item \verb+<>+, 56, 88, 90, 92
\item \verb+?+, 4
\item \verb+@+, 4, 88, 90
\item \verb+'+, 4
\item \verb+^+, 3, 4, 88, 90
\item \verb+*+, 4, 5, 56, 80, 86, 88, 90, 92
\item $\emptymap$ (empty map), 22
\item $+$ (modification), 22, 60
\item $\oplus$, 24, 39
\item $\Lambda$ (in type function), 22, 24, 32
\item $\forall$ (in type scheme), 22, 25
\subitem see also generalisation 
\item $\alpha$ (see type variable) 
\item $\varrho$ (see record type) 
\item $\tau$ (see type) 
\item $\tauk$ (type vector), 23, 24
\item $\tych$ (type scheme), 23, 25, 26, 28, 33, 42, 50, 88, 89
\item $\longtych$ (see type scheme) 
\item $\rightarrow$ (function type), 23, 29, 35
\item $\downarrow$ (restriction), 69
\item $\typefcn$ (see type function) 
\item $(\theta,\CE)$ (see type structure) 
\item $\typefcnk$ (see type function) 
\item $\sig$ (see signature) 
\item $\longsig{}$ (see signature) 
\item $\funsig$ (see functor signature) 
\item $\longfunsig{}$ (see functor signature) 
\item $\tyrea$ (type realisation), 40
\item $\strrea$ (structure realisation), 41
\item $\rea$ (realisation), 41--43, 54
\item $\geq$ (see instance) 
\item $\succ$ (see generalisation and enrichment) 
\item $\ts$ (turnstile), 2, 28, 29, 45, 59, 70, 75
\item $\tsdyn$ (evaluation), 75
\item $\tsstat$ (elaboration), 75
\item $\ra$, 2, 28, 45, 59, 70, 75
\item $\langle\ \rangle$ (see options) 
\item $\langle'\rangle$, 47
\indexspace
\parbox{65mm}{\hfil{\large\bf A}\hfil}
\indexspace
\item $a$ (see address) 
\item $\Abs$ (abstype operation), 27, 31
\item {\tt abs}, 56, 88, 91
\item {\tt Abs}, 57, 91
\item $\ABSTYPE$, 3, 10, 27, 31, 80, 85
\item abstype declaration, 10, 27, 31, 85
\item addition of numbers (\ml{+}), 4, 56, 88, 90, 92
\item $\Addr$ (addresses), 55, 57
\item address ($\A$), 55
\subitem fresh, 61
\item admissibility, 40, 43
\item admit equality, 24, 27, 32, 40, 43, 49, 87, 92
\item $\AND$, 3, 17--19, 85
\item \ANDALSO, 3, 79, 84
\item appending lists (\verb+@+), 4, 88, 90
\item $\apexp$ (application expression), 82, 84
\item application, 10, 29
\subitem of basic value ($\APPLY$), 56, 62, 91
\subitem of (function) closure, 62
\subitem of value constructor, 61
\subitem of exception name, 61
\subitem of {\tt ref}, 61
\subitem of {\tt :=}, 61, 88
\subitem infixed, 10
\item application of functor (see functor application) 
\item application of type function, 24, 35
\item application expression, 82, 84
\item applicative type variable (see type variable) 
\item $\APPLY$ (see application) 
\item $\AppTyVar$ (applicative type variables), 5
\item $\apptyvars$ (free applicative type variables), 22
\item {\tt arctan}, 56, 88, 91
\item arity 
\subitem of type name, 21
\subitem of type function, 24, 50
\item arrow type (see function type expression) 
\item \AS, 3, 11, 35, 66, 86
\item assignment (\ml{:=}), 57, 61, 88, 90
\item $\atexp$ (atomic expression), 8, 10, 28, 60, 79, 84
\item atomic expression, 8, 10, 28, 60, 79, 84
\subitem as expression, 10, 29, 61
\item atomic pattern, 8, 11, 34, 65, 80, 86
\subitem as pattern, 11, 35, 66, 86
\item $\atpat$ (atomic pattern), 8, 11, 34, 65, 80, 86
\indexspace
\parbox{65mm}{\hfil{\large\bf B}\hfil}
\indexspace
\item $b$ (see basic value) 
\item $\B$ (see basis) 
\item $\B_0$ (initial basis) 
\subitem static, 87
\subitem dynamic, 90
\item bare language, 1
\item $\BasExc$ (basic exception names), 57, 90
\item basic value ($b$), 55, 56, 90--94
\item basis ($\B$), 1
\subitem static, 28, 38, 45, 75, 87
\subitem dynamic, 68, 75, 90
\subitem combined, 75
\item $\Basis$ (bases), 38, 68, 75
\item $\BasVal$ (basic values), 55, 56, 90--94
\item $\Bdyn$ (dynamic basis), 75
\item {\tt Bind} (exception), 57, 64
\item $\BOOL$, 87, 89
\item bound names, 38, 39, 41
\item $\Bstat$ (static basis), 75
\indexspace
\parbox{65mm}{\hfil{\large\bf C}\hfil}
\indexspace
\item $\C$ (context), 23--22, 28--36
\item ``{\tt Cannot open} $s$'', 93
\item \CASE, 3, 79, 84
\item $\CE$ (constructor environment), 23, 27, 33, 51
\item {\tt chr}, 56, 88, 91
\item {\tt Chr}, 57, 91
\item $\cl{}{}$ (closure of types etc.), 26, 31, 33, 49, 50
\item \verb+close_in+, 56, 89, 92, 93
\item \verb+close_out+, 56, 89, 93
\item $\Closure$ (function closures), 57
\subitem recursion, 58
\item closure rules (signatures and functors), 19, 48, 52
\item coercion of numbers (\ml{real}), 56, 88, 91
\item comments, 4, 6
\item composition of functions (\ml{o}), 88, 90
\item $\con$ (see value constructor) 
\item $\Con$ (value constructors), 4, 57
\item $\constrs$ (constructor binding), 8, 10, 33, 85
\item $\ConBind$ (constructor bindings), 8, 55
\item concatenating strings (\verb+^+), 4, 88, 90
\item $\condesc$ (constructor description), 15, 17, 18, 50, 68
\item ConDesc (constructor descriptions), 15, 68
\item $\ConEnv$ (constructor environments), 23
\item ``consing'' an element to a list (\ml{::}), 87--90
\item consistency 
\subitem of type structures, 39, 51
\subitem of semantic object, 39, 40, 51
\item constant (see also value constant and exception constant) 
\subitem special (see special constant) 
\item construction (see value construction and  exception construction) 
\item constructor binding ($\constrs$), 8, 10, 33, 85
\item constructor description, 15, 17, 18, 50, 68
\item constructor environment ($\CE$), 23, 27, 33, 51
\item $\ConsType$ (constructed types), 23
\item contents of (see dereferencing) 
\item context ($\C$), 23--22, 28--36
\item $\Context$ (contexts), 23
\item control character, 3
\item Core Language, 1
\subitem syntax, 3
\subitem static semantics, 21
\subitem dynamic semantics, 55
\item Core Language Programs, 77
\item {\tt cos}, 56, 88, 91
\item cover, 43
\item cycle-freedom, 40
\indexspace
\parbox{65mm}{\hfil{\large\bf D}\hfil}
\indexspace
\item \DATATYPE, 3, 10, 18, 31, 49, 68, 80, 85
\item datatype binding, 8, 10, 33, 85
\item datatype declaration, 10, 31, 85
\item datatype description, 15, 18, 50
\item datatype specification, 18, 49, 68
\item $\datbind$ (datatype binding), 8, 10, 33, 85
\item DatBind (datatype bindings), 8, 55
\item $\datdesc$ (datatype description), 15, 18, 50
\item DatDesc (datatype descriptions), 15, 68
\item $\dec$ (declaration), 8, 10, 31, 64, 80, 85
\item Dec (declarations), 8
\item declaration (Core), 8, 10, 31, 64, 80, 85
\subitem as structure-level declaration, 17, 46, 71
\item dereferencing (\ml{!}), 4, 88, 91
\item derived forms, 1, 7, 13, 78--81
\item {\tt Diff}, 57, 92
\item digit 
\subitem in identifier, 4
\subitem in integers and reals, 3
\item $\dir$ (fixity directive), 7, 10, 13
\item directive, 10
\item {\tt div}, 56, 88, 90, 92
\item {\tt Div}, 57, 92
\item division of reals (\ml{/}), 56, 88, 90, 92
\item \DO, 3, 79, 84
\item $\Dom$ (domain), 22
\item dynamic 
\subitem semantics (Core), 55
\subitem semantics (Modules), 68
\subitem basis (see basis) 
\indexspace
\parbox{65mm}{\hfil{\large\bf E}\hfil}
\indexspace
\item $\exval$ (exception value), 57
\item $[\exval]$ (see packet) 
\item \verb+E+\ (exponent), 3
\item $\E$ (environment) 
\subitem static, 23, 27, 28, 31, 32
\subitem dynamic, 57, 60--67, 69, 70
\item $\EE$ (see exception constructor environment) 
\item elaboration, 1, 2, 28, 45, 75
\item \ELSE, 3, 79, 84
\item empty 
\subitem declaration (Core), 10, 31, 64, 85
\subitem functor declaration, 19, 52, 73
\subitem functor specification, 19, 52
\subitem signature declaration, 17, 48, 72
\subitem specification, 18, 49, 72
\subitem structure-level declaration, 17, 46, 71
\item $\e$ (exception name), 55, 64
\item \END, 3, 10, 17, 18, 84, 85
\item \verb+end_of_stream+, 56, 89, 93
\item enrichment ($\succ$), 36, 42, 45, 47, 53
\item $\excs$ (exception name set), 57, 64
\item $\Env$ (environments), 23, 57
\item \EQTYPE, 13, 18, 49, 68
\item equality 
\subitem admit equality, 24, 27, 32, 40, 43, 49, 87, 92
\subitem maximise equality, 27, 31
\subitem on abstract types, 27
\subitem of structures (sharing), 51
\subitem of type functions (sharing), 24, 51
\subitem of type schemes, 25
\subitem of values, 24, 88, 90, 92
\subitem -principal, 43, 48, 52, 53
\subitem respect equality, 27, 32, 43
\item equality attribute 
\subitem of type name, 21, 24, 27, 40, 43, 49
\subitem of type variable, 5, 21, 24, 25
\item equality type, 24, 88
\item equality type function, 24
\item equality type specification, 18, 49, 68
\item equality type variable, 5, 21, 24
\item escape sequence, 3
\item evaluation, 1, 2, 59, 70, 75
\item $\exnbind$ (exception binding), 8, 10, 33, 64, 85
\item ExBind (exception bindings), 8
\item \EXCEPTION, 3, 10, 18, 31, 49, 64, 72, 85
\item exception binding, 8, 10, 33, 64, 85
\item exception constant ($\exn$ or $\longexn$) 
\subitem as atomic pattern, 11, 34, 65, 86
\item exception construction 
\subitem as pattern, 11, 35, 66, 86
\subitem infixed, as pattern, 7, 11, 86
\item exception constructor 
\subitem as atomic expression, 10, 28, 61, 84
\item exception constructor environment ($\EE$) 
\subitem static, 23--22, 33, 69
\subitem dynamic, 57, 64, 69
\item exception convention, 60--62, 76
\item exception declaration, 10, 31, 64, 85
\item exception description, 15, 18, 51, 73
\item exception name ($\e$), 55
\subitem fresh, 64
\item exception name set ($\excs$), 57, 64
\item exception packet (see packet) 
\item exception specification, 18, 49, 72
\item exception value ($\exval$), 57
\item $\exn$ (see exception constant or constructor) 
\item $\Exn$ (exception constructors), 4
\item $\ExnEnv$ (exception constructor environments), 23, 57
\item $\exns$ (exeption constructor set), 68, 73
\item $\exndesc$ (exception description), 15, 18, 51, 73
\item ExDesc (exception descriptions), 15
\item execution, 1, 75
\item exhaustive patterns, 36, 57
\item $\EXCN$, 29, 34, 35, 51, 87, 89
\item $\Exc$ (exception names), 55
\item $\ExcSet$ (exception name sets), 57
\item $\exp$ (expression), 8, 10, 29, 61, 79, 84
\item Exp (expressions), 8
\item {\tt exp} (exponential), 56, 88, 91
\item {\tt Exp}, 57, 91
\item expansive expression, 26
\item {\tt explode} (a string), 56, 88, 92
\item expression, 8, 10, 29, 61, 79, 84
\item expression row, 8, 10, 29, 61, 84
\item $\labexps$ (expression row), 8, 10, 29, 61, 84
\item ExpRow (expression rows), 8
\item $\ExVal$ (exception values), 57
\indexspace
\parbox{65mm}{\hfil{\large\bf F}\hfil}
\indexspace
\item $\F$ (functor environment), 38, 52, 53, 68, 73
\item $\FAIL$ (failure in pattern matching), 55, 60--63, 65--67
\item \FALSE, 87--89
\item $\finfun{}{}$ (finite map), 22
\item $\Fin$ (finite subset), 22
\item {\tt floor}, 56, 88, 91
\item {\tt Floor}, 57, 91
\item \FN, 3, 10, 11, 29, 62, 84
\item formatting character, 4
\item \FUN, 3, 78, 80, 85
\item $\funbind$ (functor binding), 15, 19, 53, 73, 78, 81
\item FunBind (functor bindings), 15
\item function ($\fnexp$), 10, 29, 62, 84
\item function declaration (see $\FUN$) 
\item function type ($\rightarrow$), 23, 29, 35
\item function type expression (\verb+->+), 11, 35, 86
\item function-value binding ($\fvalbind$), 36, 78, 80, 85
\item \FUNCTOR, 13, 19, 52, 73
\item functor application, 17, 45, 70, 81
\item functor binding, 15, 19, 53, 73, 78, 81
\item functor closure, 68, 70, 73
\item functor declaration, 15, 19, 52, 73
\subitem as top-level declaration, 19, 53, 74
\item functor description, 15, 19, 52
\item functor environment ($\F$), 38, 52, 53, 68, 73
\item functor identifier ($\funid$), 13, 17, 19
\item functor signature ($\funsig$), 38, 52--54
\item functor signature expression, 15, 19, 52, 81
\item functor signature matching, 15, 54
\item functor specification, 15, 19, 52
\item $\FunctorClosure$ (functor closures), 68
\item $\fundec$ (functor declaration), 15, 19, 52, 73
\item FunDec (functor declarations), 15
\item $\fundesc$ (functor description), 15, 19, 52
\item FunDesc (functor descriptions), 15
\item $\FunEnv$ (functor environments), 38, 68
\item $\funid$ (functor identifier), 13, 17, 19
\item $\FunId$ (functor identifiers), 13
\item $\funsigexp$ (functor signature expression), 15, 19, 52, 81
\item FunSigExp (functor signature expressions), 15
\item $\funspec$ (functor specification), 15, 19, 52
\item FunSpec (functor specifications), 15
\item $\FunType$ (function types), 23
\item $\fvalbind$ (function-value binding), 78, 80, 85
\subitem exhaustive, 36
\indexspace
\indexspace
\indexspace
\parbox{65mm}{\hfil{\large\bf G}\hfil}
\indexspace
\item $\G$ (signature environment), 38, 48, 68, 72
\item generalisation ($\succ$), 25, 28, 34--36, 42
\item generative signature expression, 17, 47, 71
\item generative structure expression, 17, 45, 70
\item grammar, 1
\subitem for the Core, 7, 82
\subitem for Modules, 14
\indexspace
\parbox{65mm}{\hfil{\large\bf H}\hfil}
\indexspace
\item \HANDLE, 3, 10, 29, 62, 84
\indexspace
\parbox{65mm}{\hfil{\large\bf I}\hfil}
\indexspace
\item $\I$ (interface), 68, 71, 72
\item $\IB$ (interface basis), 68, 69, 71--73
\item identifier ($\id$), 4, 13
\subitem alphanumeric, 4
\subitem long, 4, 77
\subitem qualified, 4
\subitem symbolic, 4
\item $\IE$ (interface environment), 68, 73
\item \IF, 3, 79, 84
\item imperative attribute, 21, 24, 25
\item imperative type, 24, 33
\item imperative type variable (see type variable) 
\item implementation, 1, 75
\item {\tt implode} (a string list), 56, 88, 92
\item $\ImpTyVar$ (imperative type variables), 5
\item $\imptyvars$ (free imperative type variables), 22, 53
\item $\In$ (injection), 22
\item \IN, 3, 10, 17, 18, 79, 84, 85
\item \INCLUDE, 13, 18, 49, 72
\item inference, 2
\item inference rules 
\subitem static semantics (Core), 28
\subitem static semantics (Modules), 45
\subitem dynamic semantics (Core), 58
\subitem dynamic semantics (Modules), 70
\item $\inexp$ (infix expression), 82, 84
\item InfExp (infix expressions), 82, 84
\item \INFIX, 3, 6, 10, 85
\item infix expression, 6, 10, 82, 84
\item infix pattern, 6, 11, 86
\item infixed identifiers, 6, 10, 13, 84--86, 88
\item \INFIXR, 3, 6, 10, 85
\item initial basis, 2, 87, 90
\item injection (\In), 22
\item {\tt input}, 56, 89, 93
\item input/output, 89, 92
\item instance ($\geq$) 
\subitem of signature, 41--43, 47, 49
\subitem of functor signature, 41, 45
\subitem in matching, 42, 43
\item $\INSTREAM$, 87, 89, 92
\item $\INT$, 87, 89
\item $\Int$ (interfaces), 68
\item $\IntBasis$ (interface bases), 68
\item integer constant, 3, 89
\item $\IntEnv$ (interface environments), 68
\item $\Inter$, 68, 71, 73
\item interaction, 1, 75
\item interface ($\I$), 68, 71, 72
\item interface basis ($\IB$), 68, 69, 71--73
\item interface environment ($\IE$), 68, 73
\item {\tt Interrupt}, 57, 76
\item {\tt Io}, 57, 93
\item irredundant patterns, 36, 57
\item {\tt it}, 80
\indexspace
\parbox{65mm}{\hfil{\large\bf L}\hfil}
\indexspace
\item L (left associative), 8, 83
\item $\lab$ (label), 4, 5
\item $\Lab$ (labels), 4, 5
\item \LET, 3
\subitem expression (Core), 10, 29, 61, 79, 84
\subitem expression (Modules), 17, 45, 70
\item letter in identifer, 4
\item lexical analysis, 6, 7
\item $\LIST$, 87, 89
\item list reversal (\ml{rev}), 88, 91
\item {\tt ln}, 56, 88, 91
\item {\tt Ln}, 57, 91
\item \LOCAL, 3
\subitem declaration (Core), 10, 31, 64, 85
\subitem declaration (Modules), 17, 46, 71
\subitem specification (Modules), 18, 49, 72
\item long identifiers (e.g. $\longexn$), 4, 77
\item {\tt lookahead}, 56, 89, 93
\indexspace
\parbox{65mm}{\hfil{\large\bf M}\hfil}
\indexspace
\item $\m$ (structure name), 21, 23--22, 38--42, 47, 51, 56
\subitem fresh, 45, 46
\item $\M$ (structure name set), 38, 45
\item \ml{map}, 88, 91
\item match ($\match$), 8, 10, 30, 63
\subitem irredundant, 36, 57
\subitem exhaustive, 36, 57
\subitem in closure, 57, 58
\item $\Match$, 8
\item {\tt Match} (exception), 57, 62
\item match rule, 8, 10, 30, 63
\item matching 
\subitem signatures (see signature matching) 
\subitem functor signatures (see functor signature matching) 
\item maximise equality, 27, 31
\item $\mem$ (memory), 57, 61, 66
\item $\Mem$ (memories), 57
\item memory ($\mem$), 57, 61, 66
\item {\tt mod}, 56, 88, 90, 92
\item {\tt Mod}, 57, 92
\item modification ($+$) 
\subitem of finite maps, 22
\subitem of environments, 22, 60
\item module, 15
\item Modules, 1
\item $\mrule$ (match rule), 8, 10, 30, 63
\item Mrule (match rules), 8
\item multiplication of numbers (\ml{*}), 56, 88, 90, 92
\indexspace
\parbox{65mm}{\hfil{\large\bf N}\hfil}
\indexspace
\item $\n$ (name, see structure name, type name and exception name) 
\item $\N$ (name set), 38, 45
\item $n$-tuple, 79, 80, 84, 86
\item name 
\subitem of structure ($\m$), 21, 23--22, 38--42, 45--47, 51, 56
\item name set ($\N$), 38, 45
\item $\NamesFcn$ (free names), 38, 39, 45, 53
\item $\NameSets$ (name sets), 38
\item Natural Semantics, 2
\item {\tt Neg}, 57, 91
\item negation of booleans (\ml{not}), 88, 91
\item negation of numbers (\verb+~+), 3, 56, 88, 91
\item \NIL, 79, 87--89
\item non-expansive expression, 26
\item \NONFIX, 3, 6, 10, 13, 85, 88
\item nonfix identifiers, 6, 10, 13, 85, 88
\item \ml{not}, 88, 91
\item \NUM, 88
\indexspace
\parbox{65mm}{\hfil{\large\bf O}\hfil}
\indexspace
\item \ml{o} (function composition), 88, 90
\item occurrence 
\subitem substructure, 38
\item $\of{}{}$ (projection), 22, 38
\item $\OF$, 3
\subitem in $\CASE$ expression, 79, 84
\subitem in constructor binding, 10
\subitem in exception binding, 10, 55
\subitem in exception description, 18, 68
\item \OP, 3, 6
\subitem on variable or constructor, 10, 11, 84--86
\subitem in constructor binding, 10, 85
\item \OPEN, 3, 10, 18, 31, 49, 64, 68, 72, 77, 81, 85
\item \verb+open_in+, 56, 89, 92, 93
\item \verb+open_out+, 56, 89, 92, 93
\item opening structures in declarations, 10, 31, 64, 85
\item opening structures in specifications, 18, 19, 49, 72
\item options, 8
\subitem first ($\langle\ \rangle$), 28, 47
\subitem second ($\langle\langle\ \rangle\rangle$), 28
\item {\tt ord} (of string), 56, 88, 92
\item {\tt Ord}, 57, 92
\item \ORELSE, 3, 79, 84
\item {\tt output}, 56, 89, 93
\item ``{\tt Output stream is closed}'' , 93
\item $\OUTSTREAM$, 87, 89, 92
\indexspace
\parbox{65mm}{\hfil{\large\bf P}\hfil}
\indexspace
\item $\p$ (see packet) 
\item $\Pack$ (packets), 57
\item packet ($\p$), 57, 60, 62, 70, 75, 76
\item parsing, 1, 75
\item $\pat$ (pattern), 8, 11, 35, 66, 80, 86
\item Pat (patterns), 8
\item $\labpats$ (pattern row), 8, 11, 34, 66, 80, 86
\item PatRow (pattern rows), 8
\item pattern, 8, 11, 35, 66, 80, 86
\subitem layered, 11, 35, 66, 86
\item pattern matching, 36, 55, 57, 66
\subitem with $\REF$, 66
\item pattern row, 8, 11, 34, 66, 80, 86
\item polymorphic 
\subitem functions, 28, 31, 34
\subitem references, 26, 31, 53, 88
\subitem exceptions, 26, 33, 51, 53
\item precedence, 8, 82
\item principal 
\subitem environment, 36, 46
\subitem equality-, 43, 48, 52, 53
\subitem signature, 43, 48, 52, 53
\item printable character, 3
\item {\tt Prod}, 57, 92
\item product type (\verb+*+), 80, 86
\item program ($\program$), 1, 75, 76
\item Program (programs), 75
\item projection ($\of{}{}$), 22, 38
\indexspace
\parbox{65mm}{\hfil{\large\bf Q}\hfil}
\indexspace
\item qualified identifier, 4
\item {\tt Quot}, 57, 92
\indexspace
\parbox{65mm}{\hfil{\large\bf R}\hfil}
\indexspace
\item $\r$ (record), 57, 61, 65, 66
\item R (right associative), 8, 83
\item \RAISE, 3, 10, 29, 30, 60, 62, 75, 84
\item $\Ran$ (range), 22
\item $\REAL$ 
\subitem the type, 87, 89
\subitem coercion, 56, 88, 91
\item real constant, 3, 89
\item realisation ($\rea$), 41--43, 54
\item $\REC$, 3, 10, 11, 32, 58, 64, 85
\item $\Rec$ (recursion operator), 58, 62, 64
\item record  
\subitem $\r$, 57, 61, 65, 66
\subitem as atomic expression, 10, 29, 61, 79, 84
\subitem as atomic pattern, 11, 34, 65, 80, 86
\subitem selector (\ml{\#}\ {\it lab}), 3, 79, 84
\subitem type expression, 11, 35, 86
\subitem type ($\varrho$), 23, 29, 34, 36
\item Record (records), 57
\item $\RecType$ (record types), 23
\item recursion (see $\REC$, $\Rec$, and $\FUN$) 
\item $\REF$ 
\subitem the type constructor, 87, 89
\subitem the type name, 24, 87--89
\subitem the value constructor, 55, 61, 66, 88, 89, 91
\item reserved words, 3, 13
\item respect equality (see equality) 
\item restrictions 
\subitem closure rules (see these) 
\subitem syntactic (Core), 11, 36
\subitem syntactic (Modules), 17
\item \ml{rev}, 88, 91
\indexspace
\parbox{65mm}{\hfil{\large\bf S}\hfil}
\indexspace
\item $\s$ (state), 57, 59, 61, 66, 70, 75, 76
\item $\S$ (structure), 23, 38, 39, 42, 45, 47, 56
\item {\SCon} (special constants), 4
\item {\scon} (see special constant) 
\item scope 
\subitem of constructor, 5, 22
\subitem of value variable, 5, 22
\subitem of fixity directive, 7, 13
\subitem of explicit type variable, 25, 31, 32
\item $\SE$ (structure environment) 
\subitem static, 23--22, 38, 42, 47, 51, 87
\subitem dynamic, 57, 69, 71, 90
\item semantic object, 2
\subitem simple (Static), 21
\subitem simple (Dynamic), 55
\subitem compound (Core, Static), 22, 23
\subitem compound (Core, Dynamic), 57
\subitem compound (Modules, Static), 38
\subitem compound (Modules, Dynamic), 68
\item sentence, 2, 28, 45, 59, 70, 75
\item separate compilation, 15, 19, 20, 54
\item sequential 
\subitem expression, 79, 84
\subitem declaration (Core), 10, 31, 64, 85
\subitem functor declarations, 19, 52, 73
\subitem functor specification, 19, 52
\subitem signature declaration, 17, 48, 72
\subitem specification, 18, 49, 72
\subitem structure-level declaration, 17, 46, 71
\item $\shareq$ (sharing equation), 15, 18, 51, 68
\item SharEq (sharing equations), 15, 68
\item sharing, 19, 20, 46, 47, 50, 51, 54
\subitem equations, 15, 18, 51, 68
\subitem specification, 18, 49
\subitem of structures, 18, 51
\subitem of types, 18, 51
\subitem multiple, 18, 51
\item \SHARING, 13, 18, 49
\item side-condition, 59, 70
\item side-effect, 70, 76
\item \SIG, 13, 17, 47, 71
\item $\Sig$ (signatures), 38
\item $\sigbind$ (signature binding), 15, 17, 48, 72
\item SigBind (signature bindings), 15
\item $\sigdec$ (signature declaration), 15, 17, 48, 72
\item SigDec (signature declarations), 15
\item $\SigEnv$ (signature environments), 38, 68
\item $\sigexp$ (signature expression), 15, 17, 47, 71
\item SigExp (signature expressions), 15
\item $\sigid$ (signature identifier), 13, 17, 47, 71
\item $\SigId$ (signature identifiers), 13
\item signature ($\sig$), 38, 39, 41--43, 47, 48, 52--54, 69
\item \SIGNATURE, 13, 17, 48, 72
\item signature binding, 15, 17, 48, 72
\item signature declaration, 15, 17, 48, 72
\subitem in top-level declaration, 19, 53, 73
\item signature environment ($\G$) 
\subitem static, 38, 48, 53
\subitem dynamic, 68, 69, 72, 74
\item signature expression, 15, 17, 47, 71
\item signature identifier, 13, 17, 47, 71
\item signature instantiation (see instance) 
\item signature matching, 42, 43, 45--47, 53
\item {\tt sin}, 56, 88, 91
\item {\tt size} (of strings), 56, 88, 91
\item $\spec$ (specification), 15, 18, 49, 72
\item Spec (specifications), 15
\item special constant (\scon), 3, 4, 22
\subitem as atomic expression, 10, 28, 60, 84
\subitem in pattern, 11, 34, 65, 86
\item special value ($\sv$), 55
\item specification, 15, 18, 49, 72
\item {\tt sqrt} (square root), 56, 88, 91
\item {\tt Sqrt}, 57, 91
\item state ($\s$), 57, 59, 61, 66, 70, 75, 76
\item $\State$, 57
\item state convention, 60, 61
\item static 
\subitem basis, 1, 28, 38, 45, 75, 87
\subitem semantics (Core), 21
\subitem semantics (Modules), 38
\item \verb+std_in+, 56, 89, 93
\item \verb+std_out+, 56, 89, 93
\item $\Str$ (structures), 23
\item $\strbind$ (structure binding), 15, 17, 47, 71
\item StrBind (structure bindings), 15
\item $\strdec$ (structure-level declaration), 15, 17, 46, 71, 77
\item StrDec (structure-level declarations), 15
\item $\strdesc$ (structure description), 15, 18, 51, 73
\item StrDesc (structure descriptions), 15
\item stream (input/output), 92
\item $\StrEnv$ (structure environments), 23, 57
\item $\strexp$ (structure expression), 15, 17, 45, 70, 81
\item $\StrExp$ (structure expressions), 15
\item $\strid$ (structure identifier), 4
\subitem as structure expression, 17, 45, 70
\item $\StrId$ (structure identifiers), 4
\item $\STRING$, 87, 89
\item string constant, 3, 89
\item $\StrNames$ (structure names), 21
\item $\StrNamesFcn$ (free structure names), 38
\item $\StrNameSets$ (structure name sets), 38
\item $\STRUCT$, 13, 17, 45, 70, 81
\item structure ($\S$ or $(\m,\E)$), 23, 38, 39, 42, 45, 47, 56
\item $\STRUCTURE$, 13, 17, 18, 46, 49, 71, 72
\item structure binding ($\strbind$), 15, 17, 47, 71
\item structure declaration, 17, 46, 71
\item structure description ($\strdesc$), 15, 18, 51, 73
\item structure environment ($\SE$) 
\subitem static, 23--22, 38, 42, 47, 51, 87
\subitem dynamic, 57, 69, 71, 90
\item structure expression ($\strexp$), 15, 17, 45, 70, 81
\item structure identifier ($\strid$), 4
\subitem as structure expression, 17, 45, 70
\item structure-level declaration ($\strdec$), 15, 17, 46, 71, 77
\subitem in top-level declaration, 19, 53, 73, 77
\item structure name ($\m$, see name) 
\item structure name set ($\M$), 38, 45
\item structure realisation ($\strrea$), 41
\item structure specification, 18, 49, 72
\item substructure, 38
\subitem proper, 38, 40
\item subtraction of numbers (\ml{-}), 56, 88, 90, 92
\item {\tt Sum}, 57, 92
\item {\SVal} (special values), 55
\item $\Supp$ (support), 40, 41
\item $\sv$ (special value), 55
\item symbol, 4
\item syntax, 3, 13, 55, 68, 82
\indexspace
\parbox{65mm}{\hfil{\large\bf T}\hfil}
\indexspace
\item $\t$ (type name), 21, 24, 27, 31, 33, 36, 38--40, 42, 50, 89
\item $\T$ (type name set), 23, 38
\item $\TE$ (type environment), 23, 27, 32, 33, 42, 50, 69
\item \THEN, 3, 79, 84
\item $\topdec$ (top-level declaration), 15, 19, 53, 73, 77
\subitem in program, 75, 76
\item TopDec (top-level declarations), 15
\item top-level declaration, 1, 15, 19, 53, 73, 77
\item $\TRUE$, 87--89
\item truncation of reals (\ml{floor}), 56, 88, 91
\item tuple, 79, 80, 84, 86
\item tuple type, 80, 86
\item $\ty$ (type expression), 8, 11, 35, 55, 80, 86
\item Ty (type expressions), 8, 11, 55
\item $\tycon$ (type constructor), 4, 10, 11, 18, 22, 27, 32, 33, 35, 39, 42, 50, 51, 89
\item $\TyCon$ (type constructors), 4
\item $\TyEnv$ (type environments), 23
\item $\TyNames$ (type names), 21
\item $\TyNamesFcn$ (free type names), 22, 45
\item $\TyNameSets$ (type name sets), 23
\item $\typbind$ (type binding), 8, 10, 32, 55, 85
\item TypBind (type bindings), 8, 55
\item $\typdesc$ (type description), 15, 18, 50, 68
\item TypDesc (type descriptions), 15, 68
\item type ($\tau$), 22, 24--26, 28--30, 34, 35
\item $\Type$ (types), 23
\item $\TYPE$, 3, 10, 18, 31, 49, 51, 55, 68, 85
\item $\scontype$ (function on special constants), 22, 28, 34
\item type binding, 8, 10, 32, 55, 85
\item type constraint (\verb+:+) 
\subitem in expression, 10, 29, 55, 84
\subitem in pattern, 11, 35, 55, 86
\item type construction, 11, 35
\item type constructor ($\tycon$), 4, 10, 11, 18, 22, 27, 32, 33, 35, 39, 42, 50, 51, 89
\item type constructor name (see type name) 
\item type declaration, 10, 31, 55, 85
\item type description ($\typdesc$), 15, 18, 50, 68
\item type environment ($\TE$), 23, 27, 32, 33, 42, 50, 69
\item type explication, 41, 42, 46, 48, 53
\item type-explicit signature (see type explication) 
\item type expression, 8, 11, 35, 55, 80, 86
\item type-expression row ($\labtys$), 8, 11, 36, 55, 86
\item type function ($\typefcn$), 23, 24, 27, 32, 39, 40, 42, 50, 51, 89
\item type name ($\t$), 21, 24, 27, 31, 33, 36, 38--40, 42, 50, 89
\item type name set, 23, 38
\item type realisation ($\tyrea$), 40
\item type scheme ($\tych$), 23, 25, 26, 28, 34, 42, 50, 88, 89
\item type specification, 18, 49, 68
\item type structure $(\theta,\CE)$, 23, 27, 31--33, 35, 39, 42, 49--51, 88, 89
\item type variable ($\tyvar$, $\alpha$), 5, 11, 21
\subitem in type expression, 11, 35, 86
\subitem equality, 5, 21, 24, 25
\subitem imperative, 5, 21, 22, 24--26, 31, 33, 36, 53
\subitem applicative, 5, 21, 22, 24--26, 31, 33
\subitem explicit, 25, 30, 31
\item type vector ($\tauk$), 23, 24
\item $\TypeFcn$ (type functions), 23
\item $\TypeScheme$ (type schemes), 23
\item $\labtys$ (type-expression row), 8, 11, 36, 55, 86
\item TyRow (type-expression rows), 8, 11, 55
\item $\TyStr$ (type structures), 23
\item $\tyvar$ (see type variable) 
\item $\TyVar$ (type variables), 4, 21
\item $\TyVarFcn$ (free type variables), 22
\item $\tyvarseq$ (type variable sequence), 8
\item $\TyVarSet$, 23
\indexspace
\parbox{65mm}{\hfil{\large\bf U}\hfil}
\indexspace
\item $\U$ (explicit type variables), 23--25, 31
\item $\UNIT$, 89
\item unguarded type variable, 25
\indexspace
\parbox{65mm}{\hfil{\large\bf V}\hfil}
\indexspace
\item $\V$ (value), 57, 60--63
\item $\sconval$ (function on special constants), 55, 60, 65
\item $\Val$ (values), 57
\item $\VAL$, 3, 10, 18, 31, 49, 64, 72, 85
\item $\valbind$ (value binding), 8, 10, 25, 26, 31, 32, 64, 85
\subitem simple, 10, 32, 64, 85
\subitem recursive, 10, 32, 64, 85
\item Valbind (value bindings), 8
\item $\valdesc$ (value description), 15, 18, 50, 73
\item ValDesc (value descriptions), 15
\item value binding ($\valbind$), 8, 10, 25, 26, 31, 32, 64, 85
\subitem simple, 10, 32, 64, 85
\subitem recursive, 10, 32, 64, 85
\item value constant ($\con$) 
\subitem in pattern, 11, 34, 65, 86
\item value constructor ($\con$), 4
\subitem as atomic expression, 10, 28, 61, 84
\subitem scope, 5, 22
\item value construction 
\subitem in pattern, 11, 35, 66, 86
\subitem infixed, in pattern, 11, 86
\item value declaration, 10, 25, 31, 64, 85
\item value description ($\valdesc$), 15, 18, 50, 73
\item value variable ($\var$), 4
\subitem as atomic expression, 10, 28, 60, 84
\subitem in pattern, 11, 34, 65, 86
\item value specification, 18, 49, 72
\item $\var$ (see value variable) 
\item $\Var$ (value variables), 4
\item $\VarEnv$ (variable environments), 23, 57
\item variable (see value variable) 
\item variable environment ($\VE$) 
\subitem static, 23--22, 26, 31--35, 42, 50, 69, 88, 89
\subitem dynamic, 57, 58, 64--66, 69, 90
\item $\vars$ (set of value variables), 68, 73
\item $\VE$ (see variable environment) 
\item via $\rea$, 42, 54
\item view of a structure, 47, 51, 68, 70, 71
\indexspace
\parbox{65mm}{\hfil{\large\bf W}\hfil}
\indexspace
\item well-formed 
\subitem assembly, 39, 40
\subitem functor signature, 39
\subitem signature, 39
\subitem type structure, 27
\item \WHILE, 3, 79, 84
\item wildcard pattern (\verb+_+), 11, 34, 65, 86
\item wildcard pattern row (\verb+...+), 3, 11, 34, 36, 66, 86
\item \WITH, 3, 10, 85
\item \WITHTYPE, 3, 78, 80, 85
\indexspace
\parbox{65mm}{\hfil{\large\bf Y}\hfil}
\indexspace
\item $\Yield$, 41
\end{theindex}

\end{document}

 HOW TO REVISE THE INDEX 

The index is made using partly LaTex and partly an ML progam;
the latter is found on the file ``index.ml''. 

When LaTex is run on input ``root.tex'' it produces a file ``root.idx''.
Each line in this file is of the form

               \indexentry{idxkey}{pageref}

where idxkey is a key inserted precisely one place in the document 
(in the form of a LaTex command \index{idxkey})
and pageref is the page number of the page LaTex was printing
by the time it encountered the \index{idxkey} command.

A typical idxkey is 45.1 , which will occur in the TeX file 
somewhere near what produces page 45 in the final document. 
In fact, when the keys were first inserted in the TeX file, 
45.1 would be the first key on page 45, but as the document changes, 
one cannot get the pageref from the idxkey simple by taking the 
prefix of the idxkey up to the full stop.

There are many entries in the index that refer to the same
idxkey. Thus the number of idxkeys has been kept relatively
small, typically 2 or 3 pr page. The basic idea, then, is that
there is an ML program (on file ``index.ml'') which
associates entries in the final index with idxkeys by 
a sequence of expressions

              .......
              item ``handle'' [p``45.4'',``57.3''to``59.1'',p``78.2''];
              item ``{\\it happiness}'' [p''38.1''];
              ......
 
The program will first build a conversion table from idxkeys
(such as ``45.4'') to page references by reading thrugh
``root.idx''. Then it will evaluate all the item and subitem
expressions. The item function produces a line in the
latex file (``index.tex'') using the conversion table.
Thus the above lines may produce

               ...........
               \item ``handle'' 46, 57--58, 78
               \item {\it happiness} 38
               ...........

If insertion of more text in the source files result in 
new page splits, then one should manually check that
the item expressions in ``index.ml'' refer to the
right idxkeys. It may be necessary to change some of the
lists in the item expressions and it may be desirable to
insert new \index commands in the source text. However,
if we for simplicity assume that we simply insert new
material corresponding to a new chapter (not affecting
the page splits in other chapters) then one would proceed
as follows:
First add new item expressions in the index.ml file corre-
sponding to new entries in the index (one will have new
\index commands in the new input, of course). Then one
runs latex on ``root.tex'' to produce the ``root.idx'' file. 
Then one runs index.ml (enter ML and type use ``index.ml'').
Then one runs latex on root again, this time giving the 
correct index.





